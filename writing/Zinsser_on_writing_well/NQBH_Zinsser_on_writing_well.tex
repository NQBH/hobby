\documentclass{article}
\usepackage[backend=biber,natbib=true,style=alphabetic]{biblatex}
\addbibresource{/home/nqbh/reference/bib.bib}
\usepackage{tocloft}
\renewcommand{\cftsecleader}{\cftdotfill{\cftdotsep}}
\usepackage[colorlinks=true,linkcolor=blue,urlcolor=red,citecolor=magenta]{hyperref}
\usepackage{algorithm,algpseudocode,amsmath,amssymb,amsthm,float,graphicx,mathtools}
\usepackage{enumitem}
\setlist{leftmargin=4mm}
\allowdisplaybreaks
\usepackage[left=1cm,right=1cm,top=5mm,bottom=5mm,footskip=4mm]{geometry}
\def\labelitemii{$\circ$}

\title{On Writing Well: The Classic Guide to Writing Nonfiction}
\author{William Zinsser}
\date{\today}

\begin{document}
\maketitle
\tableofcontents
\subsection*{Reference{\tt/}Writing Skills}

\begin{quotation}
	``\textit{On Writing Well} \cite{Zinsser2001, Zinsser2016} is a bible for a generation of writers looking for clues to clean, compelling prose.'' -- \textit{New York Times}
\end{quotation}
``\textit{On Writing Well} has been praised for its sound avice, its clarity \& the warmth of its style. It is a book for everybody who wants to learn how to write or who needs to do some writing to get through the day, as almost everybody does in the age of e-mail \& the Internet.

Whether you want to write about people or places, science \& technology, business, sports, the arts or about yourself in the increasingly popular memoir genre, \textit{On Writing Well} offers you fundamental principles as well as the insights of a distinguished writer \& teacher. With $>10^6$ copies sold, this volume has stood the test of time \& remains a valuable resource for writers \& would-be writers.''
\begin{quotation}
	``Not since \textit{The Elements of Style} has there been a guide to writing as well presented \& readable as this one. A love \& respect for the language is evident on every page.'' -- \textit{Library Journal}
\end{quotation}
\textsc{Books by William Zinsser.} \textit{Any Old Place With You. Seen Any Good Movies Lately? The City Dwellers. Weekend Guests. The Haircurl Papers. Pop Goes America. The Paradise Bit. The Lunacy Boom. On Writing Well. Writing With a Word Processor. Willie \& Dwike} (republished as \textit{Mitchell \& Ruff}). \textit{Writing to Learn. Spring Training. American Places. Speaking of Journalism. Easy to Remember}.

\noindent\textsc{Audio Books by William Zinsser.} \textit{On Writing Well. How to Write a Memoir}.

\noindent\textsc{Books Edited by William Zinsser.} \textit{Extraordinary Lives: The Art \& Craft of American Biography. Inventing the Truth: The Art \& Craft of Memoir. Spiritual Quests: The Art \& Craft of Religious Writing. Paths of Resistance: The Art \& Craft of the Political Novel. Worlds of Childhood: The Art \& Craft of Writing for Children. They Went: The Art \& Craft of Travel Writing. Going on Faith: Writing as a Spiritual Quest}.

%------------------------------------------------------------------------------%

\section*{Introduction}
``When I 1st wrote this book, in 1976, the readers I had in mind were a relatively small segment of the population: students, writers, editors, \& people who wanted to learn to write. I wrote it on a typewriter, the highest technology then available. I had no inkling of the electronic marvels just around the corner that were about to revolutionize the act of writing. 1st came the word processor, in the 1980s, which made the computer an everyday tool for people who had never thought of themselves as writers. Then came the Internet \& e-mail, in the 1990s, which completed the revolution. Today everybody in the world is writing to everybody else, keeping in touch \& doing business across every border \& time zone.

To me this is nothing less than a miracle, curing overnight what appeared to be a deep American disorder. I've been repeated told by people in nonwriting occupations -- especially people in science, technology, medicine, business, \& finance -- that they hate writing \& can't write \& don't want to be made to write. 1 thing they particularly didn't want to write was letters. Just getting started on a letter loomed as a chore with so many formalities -- Where's the stationery? Where's the envelope? Where's the stamp? -- that they would keep putting it off, \& when they finally did sit down to write they would spend the entire 1st paragraph explaining why they hadn't written sooner. In the 2nd paragraph they would describe the weather in their part of the country -- a subject of no interest anywhere else. Only in the 3rd paragraph would they begin to relax \& say what they wanted to say.

Then along came e-mail \& all the formalities went away. E-mail has no etiquette. It doesn't require stationery, or neatness, or proper spelling, or preliminary chitchat. E-mail writers are like people who stop a friend on the sidewalk \& say, ``Did you see the game last night?'' WHAP! No amenities. They just start typing at full speed. So here's the miracle: All those people who said they hate writing \& can't write \& don't want to write \textit{can} write \& \textit{do} want to write. In fact, they can't be turned off. Never have so many Americans written so profusely \& with so few inhibitions. Which means that it wasn't a cognitive problem after all. It was a cultural problem, rooted in that old bugaboo of American education: fear.

Fear of writing gets planted in American schoolchildren at an early age, especially children of scientific or technical or mechanical bent. They are led to believe that writing is a special language owned by the English teacher, available only to the humanistic few who have ``a gift for words.'' But writing isn't a skill that some people are born with \& others aren't, like a gift for art or music. Writing is talking to someone else on paper. Anybody who can think clearly can write clearly, about any subject at all. That has always been the central premise of this book.

On 1 level, therefore, the new fluency created by e-mail is terrific news. Any invention that eliminates the fear of writing is up there with air conditioning \& the lightbulb. But, as always, there's a catch. Nobody told all the new e-mail writers that the essence of writing is rewriting. Just because they are writing with ease \& enjoyment doesn't mean they are writing well.

That condition was 1st revealed in the 1980s, when people began writing on word processors. 2 opposite things happened. The word processor made good writers better \& bad writers worse. Good writers know that very few sentences come out right the 1st time or even the 3rd time or the 5th time. For them the word processor was a rare gift, enabling them to fuss endlessly with their sentences -- cutting \& revising \& reshaping -- without the drudgery of retyping. Bad writers became even more verbose because writing was suddenly so easy \& their sentences looked so pretty on the screen. How could such beautiful sentences not be perfect?

E-mail pushed that verbosity to a new extreme: chatter unlimited. It's a spontaneous medium, not conductive to slowing down or looking back. That makes it ideal for the never-ending upkeep of personal life: maintaining contact with far-flung children \& grandchildren \& friends \& long-lost classmates. If the writing is often garrulous or disorganized or not quite clear, no real harm is done.

But e-mail is also where much of the world's business is now conducted. Millions of e-mail messages every day give people the information they need to do their job, \& a badly written message can cause a lot of damage. Employers have begun to realize that they literally cannot afford to hire men \& women who can't write sentences that are tight \& logical \& clear. The new information age, for all its high-tech gadgetry, is, finally, writing-based. E-mail, the Internet \& the fax are all forms of writing, \& writing is, finally, a craft, with its own set of tools, which are words. Like all tools, they have to be used right.

\textit{On Writing Well} is a craft book. That's what I set out to write 25 years ago -- a book that would teach the craft of writing warmly \& clearly -- \& its principles have never changed; they are as valid in the digital age as they were in the age of the typewriter. I don't mean that the book itself hasn't changed. I've revised \& expanded it 5 times since 1976 to keep pace with new trends in the language \& in society: a far greater interest in memoir-writing, e.g., \& in writing about business \& science \& sports, \& in nonfiction writing by women \& by newcomers to the United States from other cultural traditions.

I'm also not the same person I was 25 years ago. Books that teach, if they have a long life, should reflect who the writer has become at later stages of his own long life -- what he has been doing \& thinking about. \textit{On Writing Well} \& I have grown older \& wiser together. In each of the 5 new editions the new material consisted of things I had learned since the previous edition by continuing to wrestle with the craft as a writer. As a teacher, I've become far more preoccupied with the intangibles of the craft -- the attitudes \& values, like enjoyment \& confidence \& intention, that keep us going \& produce our best work. But it wasn't until the 6th edition that I knew enough to write the 2 chapters (21 \& 22) that deal at proper length with those attitudes \& values.

Ultimately, however, good writing rests on craft \& always will. I don't know what still newer electronic marvels are waiting just around the corner to make writing twice as easy \& twice as fast in the next 25 years. But I do know they won't make writing twice as good. That will still require plain old hard work -- clear thinking -- \& the plain old tools of the English language. \textsc{William Zinsser}, Sep 2001.'' -- \cite[pp. ix--xii]{Zinsser2001}

``1 of the pictures hanging in my office in mid-Manhattan is a photograph of the writer E. B. White. It was taken by Jill Krementz when White was 77 years old, at his home in North Brooklin, Maine. A white-haired man is sitting on a plain wooden bench at a plain wooden table -- 3 boards nailed to 4 legs -- in a small boathouse. The window is open to a view across the water. White is typing on a manual typewriter, \& the only other objects are an ashtray \& a nail keg. The keg, I don't have to be told, is his wastebasket.

Many people from many corners of my life -- writers \& aspiring writers, students \& former students -- have seen that picture. They come to talk through a writing problem or to catch me up on their lives. But usually it doesn't take more than a few minutes for their eye to be drawn to the old man sitting at the typewriter. What gets their attention is the simplicity of the process. White has everything he needs: a writing implement, a piece of paper, \& a receptable for all the sentences that didn't come out the way he wanted them to.

Since then writing has gone electronic. Computers have replaced the typewriter, the delete key has replaced the wastebasket, \& various other keys insert, move, \& rearrange whole chunks of text. But nothing has replaced the writer. He or she is still stuck with the same old job of saying something that other people will want to read. That's the point of the photograph, \& it's still the point -- 30 years later -- of this book.

I 1st wrote \textit{On Writing Well} in an outbuilding in Connecticut that was as small \& as crude as White's boathouse. My tools were a dangling lightbulb, an Underwood standard typewriter, a ream of yellow copy paper \& a wire wastebasket. I had then been teaching my nonfiction writing course at Yale for 5 years, \& I wanted to use the summer of 1975 to try to put the course into a book.

E. B. White, as it happened, was very much on my mind. I had long considered him my model as a writer. His was the seemingly effortless style -- achieved, I knew, with great effort -- that I wanted to emulate, \& whenever I began a new project I would 1st read some White to get his cadences into my ear. But now I also had a pedagogical interest: White was the reigning champ of the arena I was trying to enter. \textit{The Elements of Style}, his updating of the book that had most influenced \textit{him}, written in 1919 by his English professor at Cornell, William Strunk, Jr., was the dominant how-to manual for writers. Tough competition.

Instead of competing with the Strunk \& White book I decided to complement it. \textit{The Elements of Style} was a book of pointers \& admonitions: do this, don't do that. What it \textit{didn't} address was how to apply those principles to the various forms that nonfiction writing \& journalism can take. That's what I taught in my course, \& it's what I would teach in my book: how to write about people \& places, science \& technology, history \& medicine, business \& education, sports \& that arts \& everything else under the sun that's waiting to be written about.

So \textit{On Writing Well} was born, in 1976, \& it's now in its 3rd generation of readers, its sales well over a million. Today I often meet young newspaper reporters who were given the book by the editor who hired them, just as those editors were 1st given the book by the editor who hired \textit{them}. I also often meet gray-haired matrons who remember being assigned the book in college \& not finding it the horrible medicine they expected. Sometimes they bring that early edition for me to sign, its sentences highlighted in yellow. They apologize for the mess. I love the mess.

As America has steadily changed in 30 years, so has the book. I've revised it 6 times to keep pace with new social trends (more interest in memoir, business, science, \& sports), new literary trends (more women writing nonfiction), new demographic patterns (more writers from other cultural traditions), new technologies (the computer) \& new words \& usages. I've also incorporated lessons I learned by continuing to wrestle with the craft myself, writing books on subjects I hadn't tried before: baseball \& music \& American history. My purpose is to make myself \& my experience available. If readers connect with my book it's because they don't think they're hearing from an English professor. They're hearing from a working writer.

My concerns as a teacher have also shifted. I'm more interested in the intangibles that produce good writing -- confidence, enjoyment, intention, integrity -- \& I've written new chapters on those values. Since the 1990s I've also taught an adult course on memoir \& family history at the New School. My students are men \& women who want to use writing to try to understand who they are \& what heritage they were born into. Year after year their stories take me deeply into their lives \& into their yearning to leave a record of what they have done \& thought \& felt. Half the people in America, it seems, are writing a memoir.

The bad news is that most of them are paralyzed by the size of the task. How can they begin to impose a coherent shape on the past -- that vast sprawl of half-remembered people \& events \& emotions? Many are near despair. To offer some help \& comfort I wrote a book in 2004 called \textit{Writing About Your Life}. It's a memoir of various events in my own life, but it's also a teaching book: along the way I explain the writing decisions I made. They are the same decisions that confront every writer going in search of his or her past: matters of selection, reduction, organization, \& tone. Now, for this 17th edition, I've put the lessons I learned into a new chapter called ``Writing Family History \& Memoir.''

When I 1st wrote \textit{On Writing Well}, the readers I had in mind were a small segment of the population: students, writers, editors, teachers, \& people who wanted to learn how to write. I had no inkling of the electronic marvels that would soon revolutionize the act of writing. 1st came the word processor, in the 1980s, which made the computer an everyday  tool for people who had never thought of themselves as writers. Then came the Internet \& e-mail, in the 1990s, which continued the revolution. Today everybody in the world is writing to everybody else, making instant contact across every border \& across every time zone. Bloggers are saturating the globe.

On 1 level the new torrent is good news. Any invention that reduces the fear of writing is up there with air-conditioning \& the lightbulb. But, as always, there's a catch. Nobody told all the new computer writers that the essence of writing is rewriting. Just because they're writing fluently doesn't mean they're writing well.

That condition was 1st revealed with the arrival of the word processor. 2 opposite things happened: good writers go better \& bad writers got worse. Good writers welcomed the gift of being able to fuss endlessly with their sentences -- pruning \& revising \& reshaping -- without the drudgery of retyping. Bad writers became even more verbose because writing was suddenly so easy \& their sentences looked so pretty on the screen. How could such beautiful sentences not be perfect?

E-mail is an impromptu medium, not conducive to slowing down or looking back. It's ideal for the never-ending upkeep of daily life. If the writing is disorderly, no real harm is done. But e-mail is also where much of the world's business is now conducted. Millions of e-mail messages every day give people the information they need to do their job, \& a badly written message can do a lot of damage. So can a badly written Web site. The new age, for all its electronic wizardry, is still writing-based.

\textit{On Writing Well} is a craft book, \& its principles haven't changed since it was written 30 years ago. I don't know what still newer marvels will make writing twice as easy in the next 30 years. But I do know they won't make writing twice as good. That will still require plain old hard thinking -- what E. B. White was doing in his boathouse -- \& the plain old tools of the English language. \textsc{William Zinsser}, Apr 2006.'' -- \cite[pp. 5--8]{Zinsser2016}

%------------------------------------------------------------------------------%

\begin{center}\huge
	Part I: Principles
\end{center}

%------------------------------------------------------------------------------%

\section{The Transaction}
``A school in Connecticut once held ``a day devoted to the arts,'' \& I was asked if I would come \& talk about writing as a vocation. When I arrived I found that a 2nd speaker had been invited -- Dr. Brock (as I'll call him), a surgeon who had recently begun to write \& had sold some stories to magazines. He was going to talk about writing as an avocation. That made us a panel, \& we sat down to face a crowd of students \& teachers \& parents, all eager to learn the secrets of our glamorous work.

Dr. Brock was dressed in a bright red jacket, looking vaguely bohemian, as authors are supposed to look, \& the 1st question went to him. What was it like to be a writer?

He said it was tremendous fun. Coming home from an arduous day at the hospital, he would go straight to his yellow pad \& write his tensions away. The words just flowed. It was easy. I then said that writing wasn't easy \& wasn't fun. It was hard \& lonely, \& the words seldom just flowed.

Next Dr. Brock was asked if it was important to rewrite. Absolutely not, he said. ``Let it all hang out,'' he told us, \& whatever form the sentences take will reflect the writer at his most natural. I then said that rewriting is the essence of writing. I pointed out that professional writers rewrite their sentences over \& over \& then rewrite what they have rewritten.

``What do you do on days when it isn't going well?'' Dr. Brock was asked. He said he just stopped writing \& put the work aside for a day when it would go better. I then said that the professional writer must establish a daily schedule \& stick to it. I said that writing is a craft, not an art, \& that the man who runs away from his craft because he lacks inspiration is fooling himself. He is also going broke.

``What if you're feeling depressed or unhappy?'' a student asked. ``Won't that affect your writing?''

Probably it will, Dr. Brock replied. Go fishing. Take a walk. Probably it won't, I said. If your job is to write every day, you learn to do it like any other job.

A student asked if we found it useful to circulate in the literary world. Dr. Brock said he was greatly enjoying his new life as a man of letters, \& he told several stories of being taken to lunch by his publisher \& his agent at Manhattan restaurants where writers \& editors gather. I said that professional writers are solitary drudges who seldom see other writers.

``Do you put symbolism in your writing?'' a student asked me.

``Not if I can help it,'' I replied. I have an unbroken record of missing the deeper meaning in any story, play or movie, \& as for dance \& mime, I have never had any idea of what is being conveyed.

``I \textit{love} symbols!'' Dr. Brock exclaimed, \& he described with gusto the joys of weaving them through his work.

So the morning went, \& it was a revelation to all of us. At the end Dr. Brock told me he was enormously interested in my answers -- it had never occurred to him that writing could be hard. I told him I was just as interested in \textit{his} answers -- it had never occurred to me that writing could be easy. Maybe I should take up surgery on the side.

As for the students, anyone might think we left them bewildered. But in fact we gave them a broader glimpse of the writing process than if only 1 of us had talked. For there isn't any ``right'' way to do such personal work. There are all kinds of writers \& all kinds of methods, \& any method that helps you to say what you want to say is the right method for you. Some people write by day, others by night. Some people need silence, others turn on the radio. Some write by hand, some by computer, some by talking into a tape recorder. Some people write their 1st draft in 1 long burst \& then revise; others can't write the 2nd paragraph until they have fiddled endlessly with the 1st.

But all of them are vulnerable \& all of them are tense. They are driven by a compulsion to put some part of themselves on paper, \& yet they don't just write what comes naturally. They sit down to commit an act of literature, \& the self who emerges on paper is far stiffer than the person who sat down to write. The problem is to find the real man or woman behind the tension.

Ultimately the product that any writer has to sell is not the subject being written about, but who he or she is. I often find myself reading with interest about a topic I never thought would interest me -- some scientific quest, perhaps. What holds me is the enthusiasm of the writer for his field. How was he drawn into it? What emotional baggage did he bring along? How did it change his life? It's not necessary to want to spend a year alone at Walden Pond to become involved with a writer who did.

This is the personal transaction that's at the heart of good nonfiction writing. Out of it come 2 of the most important qualities that this book will go in search of: humanity \& warmth. Good writing has an aliveness that keeps the reader reading from 1 paragraph to the next, \& it's not a question of gimmicks to ``personalize'' the author. It's a question of using the English language in a way that will achieve the greatest clarity \& strength.

Can such principles be taught? Maybe not. But most of them can be learned.'' -- \cite[pp. 12--13]{Zinsser2016}

%------------------------------------------------------------------------------%

\section{Simplicity}
``Clutter is the disease of American writing. We are a society strangling in unnecessary words, circular constructions, pompous frills \& meaningless jargon.

Who can understand the clotted language of everyday American commerce: the memo, the corporation report, the business letter, the notice from the bank explaining its latest ``simplified'' statement? What member of an insurance or medical plan can decipher the brochure explaining his costs \& benefits? What father or mother can put together a child's toy from the instructions on the box? Our national tendency is to inflate \& thereby sound important. The airline pilot who announces that he is presently anticipating experiencing considerable precipitation wouldn't think of saying it may rain. The sentence is too simple -- there must be something wrong with it.

But the secret of good writing is to strip every sentence to its cleanest components. Every word that serves no function, every long word that could be a short word, every adverb that carries the same meaning that's already in the verb, every passive construction that leaves the reader unsure of who is doing what -- these are the 1001 adulterants that weaken the strength of a sentence. \& they usually occur in proportion to education \& rank.

During the 1960s the president of my university wrote a letter to mollify the alumni after a spell of campus unrest. ``You are probably aware,'' he began, ``that we have been experiencing very considerable potentially explosive expressions of dissatisfaction on issues only partially related.'' He meant that the students had been hassling them about different things. I was far more upset by the president's English than by the students' potentially explosive expressions of dissatisfaction. I would have preferred the presidential approach taken by Franklin D. Roosevelt when he tried to convert into English his own government's memos, such as this blackout order of 1942:
\begin{quotation}
	Such preparations shall be made as will completely obscure all Federal buildings \& non-Federal buildings occupied by the Federal government during an air raid for any period of time from visibility by reason of internal or external illumination.
\end{quotation}
``Tell them,'' Roosevelt said, ``that in buildings where they have to keep the work going to put something across the windows.''

Simplify, simplify. Thoreau said it, as we are so often reminded, \& no American writer more consistently practiced what he preached. Open \textit{Walden} to any page \& you will find a man saying in a plain \& orderly way what is on his mind:
\begin{quotation}
	I went to the woods because I wished to live deliberately, to front only the essential facts of life, \& see if I could not learn what it had to teach, \& not, when I came to die, discover that I had not lived.
\end{quotation}
How can the rest of us achieve such enviable freedom from clutter? The answer is to clear our heads of clutter. Clear thinking becomes clear writing; one can't exist without the other. It's impossible for a muddy thinker to write good English. He may get away with it for a paragraph or 2, but soon the reader will be lost, \& there's no sin so grave, for the reader will not easily be lured back.

Who is this elusive creative, the reader? The reader is someone with an attention span of about 30 s -- a person assailed by many forces competing for attention. At 1 time those forces were relatively few: newspapers, magazines, radio, spouse, children, pets. Today they also include a galaxy of electronic devices for receiving entertainment \& information -- television, VCRs, DVDs, CDs, video games, the Internet, e-mail, cell phones, BlackBerries, iPods -- as well as a fitness program, a pool, a lawn \& that most potent of competitors, sleep. The man or woman snoozing in a chair with a magazine or a book is a person who was being given too much unnecessary trouble by the writer. 

It won't do to say that the reader is too dumb or too lazy to keep pace with the train of thought. If the reader is lost, it's usually because the writer hasn't been careful enough. That carelessness can take any number of forms. Perhaps a sentence is so excessively cluttered that the reader, hacking through the verbiage, simply doesn't know what it means. Perhaps a sentence has been so shoddily constructed that the reader could read it in several ways. Perhaps the writer has switched pronouns in midsentence, or has switched tenses, so the reader loses track of who is talking or when the action took place. Perhaps Sentence B is not a logical sequel to Sentence A; the writer, in whose head the connection is clear, hasn't bothered to provide the missing link. Perhaps the writer has used a word incorrectly by not taking the trouble to look it up.

Faced with such obstacles, readers are at 1st tenacious. They blame themselves -- they obviously missed something, \& they go back over the mystifying sentence, or over the whole paragraph, piecing it out like an ancient rule, making guesses \& moving on. But they won't do that for long. The writer is making them work too hard, \& they will look for one who is better at the craft.

Writers must therefore constantly ask: what am I trying to say? Surprising often they don't know. Then they must look at what they have written \& ask: have I said it? Is it clear to someone encountering the subject for the 1st time? If it's not, some fuzz worked its way into the machinery. The clear writer is someone clearheaded enough to see this stuff for what it is: fuzz.

I don't mean that some people are born clearheaded \& are therefore natural writers, whereas others are naturally fuzzy \& will never write well. Thinking clearly is a conscious act that writers must force on themselves, as if they were working an any other project that requires logic: making a shopping list or doing an algebra problem. Good writing doesn't come naturally, though most people seem to think it does. Professional writers are constantly bearded by people who say they'd like to ``try a little writing someone'' -- meaning when they retire from their real profession, like insurance or real estate, which is hard. Or they say, ``I could write a book about that.'' I doubt it.

Writing is hard work. A clear sentence is no accident. Very few sentences come out right the 1st time, or even the 3rd time. Remember this in moments of despair. If you find that writing is hard, it's because it \textit{is} hard.

2 pages of the final manuscript of this chapter from the 1st Edition of \textit{On Writing Well}. Although they look like a 1st draft, they had already been rewritten \& retyped -- like almost every other page -- 4 or 5 times. With each rewrite I try to make what I have written tighter, stronger \& more precise, eliminating every element that's not doing useful work. Then I go over it once more, reading it aloud, \& am always amazed at how much clutter can still be cut. (In later editions I eliminated the sexist pronoun ``he'' denoting ``the writer'' \& ``the reader.'')'' -- \cite[pp. 15--17]{Zinsser2016}

%------------------------------------------------------------------------------%

\section{Clutter}
``Fighting clutter is like fighting weeds -- the writer is always slightly behind. New varieties sprout overnight, \& by noon they are part of American speech. Consider what President Nixon's aide John Dean accomplished in just 1 day of testimony on television during the Watergate hearings. The next day everyone in America was saying ``at this point in time'' instead of ``now.''

Consider all the prepositions that are draped onto verbs that don't need any help. We no longer head committees. We head them up. We don't face problems anymore. We face up to them when we can free up a few minutes. A small detail, you may say -- not worth bothering about. It \textit{is} worth bothering about. Writing improves in direct ratio to the number of things we can keep out of it that shouldn't be there. ``Up'' in ``free up'' shouldn't be there. Examine every word you put on paper. You'll find a surprising number that don't serve any purpose.

Take the adjective ``personal,'' as in ``a personal friend of mine,'' ``his personal feeling'' or ``her personal physician.'' It's typical of hundreds of words that can be eliminated. The personal friends has come into the language to distinguish him or her from the business friend, thereby debasing both language \& friendship. Someone's feeling \textit{is} that person's personal feeling -- that's what ``his'' means. As for the personal physician, that's the man or woman summoned to the dressing room of a stricken actress so she won't have to be treated by the impersonal physician assigned to the theater. Someday I'd like to see that person identified as ``her doctor.'' Physicians are physicians, friends are friends. The rest is clutter.

Clutter is the laborious phrase that has pushed out the short word that means the same thing. Even before John Dean, people \& businesses had stopped saying ``now.'' They were saying ``currently'' (``all our operators are currently assisting other customers''), or ``at the present time,'' or ``presently'' (which means ``soon'). Yet the idea can always be expressed by ``now'' to mean the immediate moment (``Now I can see him''), or by ``today'' to mean the historical present (``Today prices are high''), or simply by the verb ``to be'' (``It is raining''). There's no need to say, ``At the present time we are experiencing precipitation.''

``Experiencing'' is 1 of the worst clutterers. Even your dentist will ask if you are experiencing any pain. If he had his own kid in the chair he would say, ``Does it hurt?'' He would, in short, be himself. By using a more pompous phrase in his professional role he not only sounds more important; he blunts the painful edge of truth. It's the language of the flight attendant demonstrating the oxygen mask that will drop down if the plane should run out of air. ``In the unlikely possibility that the aircraft should experience such an eventuality,'' she begins -- a phrase so oxygen-depriving in itself that we are prepared for any disaster.

Clutter is the ponderous euphemism that turns a slum into a depressed socioeconomic area, garbage collectors into waste-disposal personnel \& the town dump into the volume reduction unit. I think of Bill Mauldin's cartoon of 2 hoboes riding a freight car. 1 of them says, ``I started as a simple bum, but now I'm hard-core unemployed.'' Clutter is political correctness gone amok. I saw an ad for a boys' camp designed to provide ``individual attention for the minimally exceptional.''

Clutter is the official language used by corporations to hide their mistakes. When the Digital Equipment Corporation eliminated 3000 jobs its statement didn't mention layoffs; those were ``involuntary methodologies.'' When an Air Force missile crashed, it ``impacted with the ground prematurely.'' When General Motors had a plant shutdown, that was a ``volume-related production-schedule adjustment.'' Companies that go belly-up have ``a negative cash-flow position.''

Clutter is the language of the Pentagon calling an invasion a ``reinforced protective reaction strike'' \& justifying its vast budgets on the need for ``counterforce deterrence.'' As George Orwell pointed out in ``Politics \& the English Language,'' an essay written in 1946 but often cited during the wars in Cambodia, Vietnam, \& Iraq, ``political speech \& writing are largely the defense of the indefensible $\ldots$ Thus political language has to consist largely of euphemism, question-begging \& sheer cloudy vagueness.'' Orwell's warning that clutter is not just a nuisance but a deadly tool has come true in the recent decades of American military adventurism. It was during George W. Bush's presidency that ``civilian casualities'' in Iraq became ``collateral damage.''

Verbal camouflage reached new heights during General Alexander Haig's tenure as President Reagan's secretary of state. Before Haig nobody had thought of saying ``at this juncture of maturization'' to mean ``now.'' He told the American people that terrorism could be fought with ``meaningful sanctionary teeth'' \& that intermediate nuclear missiles were ``at the vortex of cruciality.'' As for any worries that the public might harbor, his message was ``leave it to AI,'' though what he actually said was: ``We must push this to a lower decibel of public fixation. I don't think there's much of a learning curve to be achieved in this area of content.''

I could go on quoting examples from various fields -- every profession has its growing arsenal of jargon to throw dust in the eyes of the populace. But the list would be tedious. The point of raising it now is to serve notice that clutter is the enemy. Beware, then, of the long word that's no better than the short word: ``assistance'' (help), ``numerous'' (many), ``facilitate'' (ease), ``individual'' (man or woman), ``remainder'' (rest), ``initial'' (1st), ``implement'' (do), ``sufficient'' (enough), ``attempt'' (try), ``referred to as'' (called) \& hundreds more. Beware of all the slippery new fad words: paradigm \& parameter, prioritize \& potentialize. They are all weeds that will smother what you write. Don't dialogue with someone you can talk to. Don't interface with anybody.

Just as insidious are all the word clusters with which we explain how we propose to go about our explaining: ``I might add,'' ``It should be pointed out,'' ``It is interesting to note.'' If you might add, add it. If it should be pointed out, point it out. If it is interesting to note, \textit{make} it interesting; are we not all stupefied by what follows when someone says, ``This will interest you''? Don't inflate what needs no inflating: ``with the possible exception of'' (except), ``due to the fact that'' (because), ``he totally lacked the ability to'' (he couldn't), ``until such time as'' (until), ``for the purpose of'' (for).

Is there any way to recognize clutter at a glance? Here's a device my students at Yale found helpful. I would put brackets around every component in a piece of writing that wasn't doing useful work. Often just 1 word got bracketed: the unnecessary preposition appended to a verb (``order up''), or the adverb that carries the same meaning as the verb (``smile happily''), or the adjective that states a known fact (``tall skyscraper''). Often my brackets surrounded the little qualifiers that weaken any sentence they inhabit (``a bit,'' ``sort of''), or phrases like ``in a sense,'' which don't mean anything. Sometimes my brackets surrounded an entire sentence -- the one that essentially repeats what the previous sentence said, or that says something readers don't  need to know or can figure out for themselves. Most 1st drafts can be cut by 50\% without losing any information or losing the author's voice.

My reason for bracketing the students' superfluous words, instead of crossing them out, was to avoid violating their sacred prose. I wanted to leave the sentence intact for them to analyze. I was saying, ``I may be wrong, but I think this can be deleted \& the meaning won't be affected. But \textit{you} decide. Read the sentence without the bracketed material \& see if it works.'' In early weeks of the term I handed back papers that were festooned with brackets. Entire paragraphs were bracketed. But soon the students learned to put mental brackets around their own clutter, \& by the end of the term their papers were almost clean. Today many of those students are professional writers, \& they tell me, ``I still see your brackets -- they're following me through life.''

You can develop the same eye. Look for the clutter in your writing \& prune it ruthlessly. Be grateful for everything you can throw away. Reexamine each sentence you put on paper. Is every word doing new work? Can any thought be expressed with more economy? Is anything pompous or pretentious or faddish? Are you hanging on to something unless just because you think it's beautiful?

Simplify, simplify.'' -- \cite[pp. 19--22]{Zinsser2016}

%------------------------------------------------------------------------------%

\section{Style}
``So much for early warnings about the bloated monsters that lie in ambush for the writer trying to put together a clean English sentence.

``But,'' you may say, ``if I eliminate everything you think is clutter \& if I strip every sentence to its barest bones, will there by anything left of me?'' The question is a fair one; simplicity carried to an extreme might seem to point to a style little more sophisticated than ``Dick likes Jane'' \& ``See Spot run.''

I'll answer the question 1st on the level of carpentry. Then I'll get to the larger issue of who the writer is \& how to preserve his or her identity.

Few people realize how badly they write. Nobody has shown them how much excess or murkiness has crept into their style \& how it obstructs what they are trying to say. If you give me an 8-page article \& I tell you to cut it to 4 pages, you'll howl \& say it can't be done. Then you'll go home \& do it, \& it will be much better. After that comes the hard part: cutting it to 3.

The point is that you have to strip your writing down before you can build it back up. You must know what the essential tools are \& what job they were designed to do. Extending the metaphor of carpentry, it's 1st necessary to be able to saw wood neatly \& to drive nails. Later you can bevel the edges or add elegant finials, if that's your taste. But you can never forget that you are practicing a craft that's based on certain principles. If the nails are weak, your house will collapse. If your verbs are weak \& your syntax is rickety, your sentences will fall apart.

I'll admit that certain nonfiction writers, like Tom Wolfe \& Norman Mailer, have built some remarkable houses. But these are writers who spent years learning their crafts, \& when at last they raised their fanciful turrets \& hanging gardens, to the surprise of all of us who never dreamed of such ornamentation, they knew what they were doing. Nobody becomes Tom Wolfe overnight, not even Tom Wolfe.

1st, then, learn to hammer the nails, \& if what you build is sturdy \& serviceable, take satisfaction in its plain strength.

But you will be impatient to find a ``style'' -- to embellish the plain words so that readers will recognize you as someone special. You will reach for gaudy similes \& tinseled adjectives, as if ``style'' were something you could buy at the style store \& drape onto your words in bright decorator colors. (Decorator colors are the colors that decorators come in.) There is no style store; style is organic to the person doing the writing, as much a part of him as his hair, or, if he is bald, his lack of it. Trying to add style is like adding a toupee. At 1st glance the formerly bald man looks young \& even handsome. But at 2nd glance -- \& with a toupee there's always a 2nd glance -- he doesn't look quite right. The problem is not that he doesn't look well groomed; he does, \& we can only admire the wigmaker's skill. The point is that he doesn't look like himself.

This is the problem of writers who set out deliberately to garnish their prose. You lose whatever it is that makes you unique. The reader will notice if you are putting on airs. Readers want the person who is talking to them to sound genuine. Therefore a fundamental rule is: be yourself.

No rule, however, is harder to follow. It requires writers to do 2 things that by their metabolism are impossible. They must relax, \& they must have confidence.

Telling a writer to relax is like telling a man to relax while being examined for a hernia, \& as for confidence, see how stiffly he sits, glaring at the screen that awaits his words. See how often he gets up to look for something to eat or drink. A writer will do anything to avoid the act of writing. I can testify from my newspaper days that the number of trips to the water cooler per reporter-hour far exceeds the body's need for fluids.

What can be done to put the writer out of these miseries? Unfortunately, no cure has been found. I can only offer the consoling thought that you are not alone. Some days will go better than others. Some will go so badly that you'll despair of ever writing again. We have all had many of those days \& will have many more.

Still, it would be nice to keep the bad days to a minimum, which brings me back to the problem of trying to relax.

Assume that you are the writer sitting down to write. You think your article must be of a certain length or it won't seem important. You think how august it will look in print. You think of all the people who will read it. You think that it must have the solid weight of authority. You think that its style must dazzle. No wonder you tighten; you are so busy thinking of your awesome responsibility to the finished article that you can't even start. Yet you vow to be worthy of the task, \&, casting about for grand phrases that wouldn't occur to you if you weren't trying so hard to make an impression, you plunge in.

Paragraph 1 is a disaster -- a tissue of generalities that seem to have come out of a machine. No \textit{person} could have written them. Paragraphs 2 isn't much better. But Paragraph 3 begins to have a somewhat human quality, \& by Paragraph 4 you begin to sound like yourself. You've started to relax. It's amazing how often an editor can throw away the 1st 3 or 4 paragraphs of an article, or even the 1st few pages, \& start with the paragraph where the writer begins to sound like himself or herself. Not only are those 1st paragraphs impersonal \& ornate; they don't say anything -- they are a self-conscious attempt at a fancy prologue. What I'm always looking for as an editor is a sentence that says something like ``I'll never forget the day when I $\ldots$'' I think, ``Aha! A person!''

Writers are obviously at their most natural when they write in the 1st person. Writing is an intimate transaction between 2 people, conducted on paper, \& it will go well to the extent that it retains its humanity. Therefore I urge people to write in the 1st person: to use ``I'' \& ``me'' \& ``we'' \& ``us.'' They put up a fight.

``Who am I to say what \textit{I} think?'' they ask. ``Or what \textit{I} fee?''

``Who are you \textit{not} to say what you think?'' I tell them. ``There's only 1 you. Nobody else thinks or feels in exactly the same way.''

``But nobody cares about my opinions,'' they say. ``It would make me feel conspicuous.''

``They'll care if you tell them something interesting,'' I say, ``\& tell them in words that come naturally.''

Nevertheless, getting writers to use ``I'' is seldom easy. They think they must earn the right to reveal their emotions or their thoughts. Or that it's egotistical. Or that it's undignified -- a fear that afflicts the academic world. Hence the professorial use of ``one'' (``One finds oneself not wholly in accord with Dr. Maltby's view of the human condition''), or of the impersonal ``it is'' (``It is to be hoped that Prof. Felt's monograph will find the wider audience it most assuredly deserves''). I don't want to meet ``one'' -- he's a boring guy. I want a professor with a passion for his subject to tell me why it fascinates \textit{him}.

I realize that there are vast regions of writing where ``I'' isn't allowed. Newspapers don't want ``I'' in their news stories; many magazines don't want it in their articles; businesses \& institutions don't want it in the reports they send so profusely into the American home; colleges don't want ``I'' in their term papers or dissertations, \& English teachers discourage any 1st-person pronoun except the literary ``we'' (``We see in Melville's symbolic use of the white whale $\ldots$''). Many of those prohibitions are valid; newspaper articles should consist of news, reported objectively. I also sympathize with teachers who don't want to give students an easy escape into opinion -- ``I think Hamlet was stupid'' -- before they have grappled with the discipline of assessing a work on its merits \& on external sources. ``I'' can be a self-indulgence \& a cop-out.

Still, we have become a society fearful of revealing who we are. The institutions that seek our support by sending us their brochures sound remarkably alike, though surely all of them -- hospitals, schools, libraries, museums, zoos -- were founded \& are still sustained by men \& women with different dreams \& visions. Where are these people? It's hard to glimpse them among all the impersonal passive sentences that say ``initiatives were undertaken'' \& ``priorities have been identified.''

Even when ``I'' isn't permitted, it's still possible to convey a sense of I-ness. The political columnist James Reston didn't use ``I'' in his columns; yet I had a good idea of what kind of person he was, \& I could say the same of many other essayists \& reporters. Good writers are visible just behind their words. If you aren't allowed to use ``I,'' at least think ``I'' while you write, or write the 1st draft in the 1st person \& they take the ``I''s out. It will warm up your impersonal style.

Style is tied to the psyche, \& writing has deep psychological roots. The reasons why we express ourselves as we do, or fail to express ourselves because of ``writer's block,'' are partly buried in the subconscious mind. There are as many kinds of writer's block as there are kinds of writers, \& I have no intention of trying to untangle them. This is a short book, \& my name isn't Sigmund Freud.

But I've also noticed a new reason for avoiding ``I'': Americans are unwilling to go out on a limb. A generalization ago our leaders told us where they stood \& what they believed. Today they perform strenuous verbal feats to escape that fate. Watch them wriggle through TV interviews without committing themselves. I remember President Ford assuring a group of visiting businessmen that his fiscal policies would work. He said: ``We see nothing but increasingly brighter clouds every month.'' I took this to mean that the clouds were still fairly dark. Ford's sentence was just vague enough to say nothing \& still sedate his constituents.

Later administrations brought no relief. Defense Secretary Caspar Weinberger, assessing a Polish crisis in 1984, said: ``There's continuing ground for serious concern \& the situation remains serious. The longer it remains serious, the more ground there is for serious concern.'' The 1st President Bush, questioned about his stand on assault rifles, said: ``There are various groups that think you can ban certain kinds of guns. I am not in that mode. I am in the mode of being deeply concerned.''

But my all-time champ is Elliot Richardson, who held 4 major cabinet positions in the 1970s. It's hard to know where to begin picking from his trove of equivocal statements, but consider this one: ``\& yet, on balance, affirmative action has, I think, been a qualified success.'' A 13-word sentence with 5 hedging words. I give it 1st prize as the most wishy-washy sentence in modern public discourse, though a rival would be his analysis of how to ease boredom among assembly-line workers: ``\& so, at last, I come to the 1 firm conviction that I mentioned at the beginning: it is that the subject is too new for final judgments.''

That's a firm conviction? Leaders who bob \& weave like aging boxers don't inspire confidence -- or deserve it. The same thing is true of writers. Sell yourself, \& your subject will exert its own appeal. Believe in your own identity \& your own opinions. Writing is an act of ego, \& you might as well admit it. Use its energy to keep yourself going.'' -- \cite[pp. 24--28]{Zinsser2016}

%------------------------------------------------------------------------------%

\section{The Audience}
``Soon after you confront the matter of preserving your identity, another question will occur to you: ``Who am I writing for?''

It's a fundamental question, \& it has a fundamental answer: You are writing for yourself. Don't try to visualize the great mass audience. There is no such audience -- every reader is a different person. Don't try to guess what sort of thing editors want to publish or what you think the country is in a mood to read. Editors \& readers don't know what they want to read until they read it. Besides, they're always looking for something new.

Don't worry about whether the reader will ``get it'' if you indulge a sudden impulse for humor. If it amuses you in the act of writing, put it in. (It can always be taken out, but only you can put it in.) You are writing primarily to please yourself, \& if you go about it with enjoyment you will also entertain the readers who are worth writing for. If you lose the dullards back in the dust, you don't want them anyway.

This may seem to be a paradox. Earlier I warned that the reader is an impatient bird, perched on the thin edge of distraction or sleep. Now I'm saying you must write for yourself \& not be gnawed by worry over whether the reader is tagging along.

I'm talking about 2 different issues. One is craft, the other is attitude. The 1st is a question of mastering a precise skill. The 2nd is a question of how you use that skill to express your personality.

In terms of craft, there's no excuse for losing readers through sloppy workmanship. If they doze off in the middle of your article because you have been careless about a technical detail, the fault is yours. But on the larger issue of whether the reader likes you, or likes what you are saying or how you are saying it, or agrees with it, or feels an affinity for your sense of humor or your vision of life, don't give him a moment's worry. You are who you are, he is who he is, \& either you'll get along or you won't.

Perhaps this still seems like a paradox. How can you think carefully about not losing the reader \& still be carefree about his opinion? I assure you that they are separate processes.

1st, work hard to master the tools. Simplify, prune, \& strive for order. Think of this as a mechanical act, \& soon your sentences will become cleaner. The act will never become as mechanical as, say, shaving or shampooing; you will always have to think about the various ways in which the tools can be used. But at least your sentences will be grounded in solid principles, \& your chances of losing the reader will be smaller.

Think of the other as a creative at: the expressing of who you are. Relax \& say what you want to say. \& since style is who you are, you only need to be true to yourself to find it gradually emerging from under the accumulated clutter \& debris, growing more distinctive every day. Perhaps the style won't solidify for years as \textit{your} style, \textit{your} voice. Just as it takes time to find yourself as a person, it takes time to find yourself as a stylist, \& even then your style will change as you grow older.

But whatever your age, be yourself when you write. Many old men still write with the zest they had in their 20s or 30s; obviously their ideas are still young. Other old writers ramble \& repeat themselves; their style is the tip-off that they have turned into garrulous bores. Many college students write as if they were desiccated alumni 30 years out. Never say anything in writing that you wouldn't comfortably say in conversation. If you're not a person who says ``indeed'' or ``moreover,'' or who calls someone an individual (``he's a fine individual''), \textit{please} don't write it.

Let's look at a few writers to see the pleasure with which they put on paper their passions \& their crotchets, not caring whether the reader shares them or not. The 1st excerpt is from ``The Hen (An Appreciation),'' written by E. B. White in 1944, at the height of World War II:
\begin{quotation}
	Chickens do not always enjoy an honorable position among city-bred people, although the egg, I notice, goes on \& on. Right now the hen is in favor. The var has deified her \& she is the darling of the home front, feted at conference tables, praised in every smoking car, her girlish ways \& curious habits the topic of many an excited husbandryman to whom yesterday she was a stranger without honor or allure.
	
	My own attachment to the hen dates from 1907, \& I have been faithful to her in good times \& bad. Ours has not always been an easy relationship to maintain. At 1st, as a boy in a carefully zoned suburb, I had neighbors \& police to reckon with; my chickens had to be as closely guarded as an underground newspaper. Later, as a man in the country, I had my old friends in town to reckon with, most of whom regarded the hen as a comic prop straight out of vaudeville $\ldots$ Their scorn only increased my devotion to the hen. I remained loyal, as a man would to a bride whom his family received with open ridicule. Now it is my turn to wear the smile, as I listen to the enthusiastic cackling of urbanites, who have suddenly taken up the hen socially \& who fill the air with their newfound ecstasy \& knowledge \& the relative charms of the New Hampshire Red \& the Laced Wyandotte. You would think, from their nervous cries of wonder \& praise, that the hen was hatched yesterday in the suburbs of New York, instead of in the remote past in the jungles of India.
	
	To a man who keeps hens, all poultry lore is exciting \& endlessly fascinating. Every spring I settle down with my farm journal \& read, with the same glazed expression on my face, the age-old story of how to prepare a brooder house $\ldots$
\end{quotation}
There's a man writing about a subject I have absolutely no interest in. Yet I enjoy this piece thoroughly. I like the simple beauty of its style. I like the rhythms, the unexpected but refreshing words (``deified,'' ``allure,'' ``cackling''), the specific details like the Laced Wyandotte \& the brooder house. But mainly what I like is that this is a man telling me unabashedly about a love affair with poultry that goes back to 1907. It's written with humanity \& warmth, \& after 3 paragraphs I know quite a lot about what sort of man this hen-lover is.

Or take a writer who is almost White's opposite in terms of style, who relishes the opulent word for its opulence \& doesn't deify the simple sentence. Yet they are brothers in holding firm opinions \& saying what they think. This is H. L. Mencken reporting on the notorious ``Monkey Trial'' -- the trial of John Scopes, a young teacher who taught the theory of evolution in his Tennessee classroom -- in the summer of 1925:
\begin{quotation}
	It was hot weather when they tried the infidel Scopes at Dayton, Tenn., but I went down there very willingly, for I was eager to see something of evangelical Christianity as a going concern. In the big cities of the Republic, despite the endless efforts of consecrated men, it is laid up with a wasting disease. They very Sunday-school superintendents, taking jazz from the stealthy radio, shake their fire-proof legs; their pupils, moving into adolescence, no longer respond to the proliferating hormones by enlisting for missionary service in Africa, but resort to necking instead. Even in Dayton, I found, though the mob was up to do execution on Scopes, there was a strong smell of antinomianism. The 9 churches of the village were all half empty on Sunday, \& weeds choked their yards. Only 2 or 3 of the resident pastors managed to sustain themselves by their ghostly science; the rest had to take orders for mail-order pantaloons or work in the adjacent strawberry fields; one, I heard, was a barber $\ldots$ Exactly 12 minutes after I reached the village I was taken in tow by a Christian man \& introduced to the favorite tipple of the Cumberland Range; half corn liquor \& half Coca-Cola. It seemed a dreadful dose to me, but I found that the Dayton illuminati got it down with gusto, rubbing their tummies \& rolling their eyes. They were all hot for Genesis, but their faces were too florid to belong to teetotalers, \& when a pretty girl came tripping down the main street, they reached for the places where their neckties should have been with all the amorous enterprise of movie stars $\ldots$
\end{quotation}
This is pure Mencken in its surging momentum \& its irreverence. At almost any page where you open his books he is saying something sure to outrage the professed pieties of his countrymen. The sanctity in which Americans bathed their heroes, their churches \& their edifying laws -- especially Prohibition -- was a well of hypocrisy for him that never dried up. Some of his heaviest ammunition he hurled at politicians \& Presidents -- his portrait of ``The Archangel Woodrow'' still scorches the pages -- \& as for Christian believers \& clerical folk, they turn up unfailingly as mountebanks \& boobs.

It may seem a miracle that Mencken could get away with such heresies in the 1920s, when hero worship was an American religion \& the self-righteous wrath of the Bible Belt oozed from coast to coast. Not only did he get away with it; he was the most revered \& influential journalist of his generation. The impact he made on subsequent writers of nonfiction is beyond measuring, \& even now his topical pieces seem as fresh as if they were written yesterday.

The secret of his popularity -- aside from his pyrotechnical use of the American language -- was that he was writing for himself \& didn't give a damn what the reader might think. It wasn't necessary to share his prejudices to enjoy seeing them expressed with such mirthful abandon. Mencken was never timid or evasive; he didn't kowtow to the reader or curry anyone's favor. It takes courage to be such a writer, but it is out of such courage that revered \& influential journalists are born.

Moving forward to our own time, here's an excerpt from \textit{How to Survive in Your Native Land}, a book by James Herndon describing his experiences as a teacher in a California junior high school. Of all the earnest books on education that have sprouted in America, Herndon's is -- for me -- the one that best captures how it really is in the classroom. His style is not quite like anybody else;s, but his voice is true. Here's how the book starts:
\begin{quotation}
	I might as well begin with Piston. Piston was, as a matter of description, a red-headed medium-sized chubby 8th-grader; his definitive characteristic was, however, stubbornness. Without going into a lot of detail, it became clear right away that what Piston didn't want to do, Piston didn't do; what Piston wanted to do, Piston did.
	
	It really wasn't much of a problem. Piston wanted mainly to paint, draw monsters, scratch designs on mimeograph blanks \& print them up, write an occasional horror story -- some kids referred to him as The Ghoul -- \& when he didn't want to do any of those, he wanted to roam the halls \& on occasion (we heard) investigate the girls' bathrooms.
	
	We had minor confrontations. Once I wanted everyone to sit down \& listen to what I had to say -- something about the way they had been acting in the halls. I was letting them come \& go freely \& it was up to them (I planned to point out) not to raise hell so that I had to hear about it from other teachers. Sitting down was the issue -- I was determined everyone was going to do it 1st, then I'd talk. Piston remained standing. I reordered. He paid no intention. I pointed out that I was talking to him. He indicated he heard me. I inquired then why in hell didn't he sit down. He said he didn't want to. I said I did want him to. He said that didn't matter to him. I said do it anyway. He said why? I said because I said so. He said he wouldn't. I said Look I want you to sit down \& listen to what I'm going to say. He said he \textit{was} listening. I'll listen but I won't sit down.
	
	Well, that's the way it goes sometimes in schools. You as teacher become obsessed with an issue -- I was the injured party, conferring, as usual, unheard-of freedoms, \& here they were as usual taking advantage. It ain't pleasant coming in the teachers' room for coffee \& having to hear somebody say that so-\&-so \& so-\&-so from \textit{your} class were out in the halls \textit{without a pass} \& \textit{making faces} \& \textit{giving the finger} to kids in \textit{my} class during the most \textit{important} part of \textit{my} lesson about \textit{Egypt} -- \& you ought to be allowed your tendentious speech, \& most everyone will allow it, sit down for it, but occasionally someone wises you up by refusing to submit where it isn't necessary $\ldots$ How did any of us get into this? we ought to be asking ourselves.
\end{quotation}
Any writer who uses ``ain't'' \& ``tendentious'' in the same sentences, who quotes without using quotation marks, knows what he's doing. This seemingly artless style, so full of art, is ideal for Herndon's purpose. It avoids the pretentiousness that infects so much writing by people doing worthy work, \& it allows for a rich vein of humor \& common sense. Herndon sounds like a good teacher \& a man whose company I would enjoy. But ultimately he is writing for himself: an audience of one.

``Who am I writing for?'' The question that begins this chapter has irked some readers. They want me to say ``Whom am I writing for?'' But I can't bring myself to say it. It's just not me.'' -- \cite[pp. 30--35]{Zinsser2016}

%------------------------------------------------------------------------------%

\section{Words}
``There is a kind of writing that might be called journalese, \& it's the death of freshness in anybody's style. It's the common currency of newspapers \& of magazines like \textit{People} -- a mixture of cheap words, made-up words \& clich\'es that have become so pervasive that a writer can hardly help using them. You must fight these phrases or you'll sound like every hack. You'll never make you mark as a writer unless you develop a respect for words \& a curiosity about their shades of meaning that is almost obsessive. The English language is rich in strong \& supple words. Take the time to root around \& find the ones you want.

What is ``journalese''? It's a quilt of instant words patched together out of other parts of speech. Adjectives are used as nouns (``greats,'' ``notables''). Nouns are used as verbs (``to host''), or they are chopped off to form verbs (``enthuse,'' ``emote''), or they are padded to form verbs (``beef up,'' ``put teeth into''). This is a world where eminent people are ``famed'' \& their associates are ``staffers,'' where the future is always ``up-coming'' \& someone is forever ``firing off'' a note. Nobody in America has sent a note or a memo or a telegram in years. Famed diplomat Condoleezza Rice, who hosts foreign notables to beef up the morale of top State Department staffers, sits down \& fires off a lot of notes. Notes that are fired off are always fired in anger \& from a sitting position. What the weapon is I've never found out.

Here's an article from a famed newsmagazine that is hard to match for fatigue:
\begin{quotation}
	Last Feb, Plainclothes Patrolman Frank Serpico knocked at the door of a suspected Brooklyn heroin pusher. When the door opened a crack, Serpico shouldered his way in only to be met by a .22-cal. pistol slug crashing into his face. Somehow he survived, although there are still buzzing fragments in his head, causing dizziness \& permanent deafness in his left ear. Almost as painful is the suspicion that he may well have been set up for the shooting by other policemen. For Serpico, 35, has been waging a lonely, 4-year war against the routine \& endemic corruption that he \& others claim is rife in the New York City police department. His efforts are now sending shock waves through the ranks of New York's finest $\ldots$ Though the impact of the commission's upcoming report has yet to be felt, Serpico has little hope that $\ldots$
\end{quotation}
The upcoming report has yet to be felt because it's still upcoming, \& as for the permanent deafness, it's a little early to tell. \& what makes those buzzing fragments buzz? By now only Serpico's head should be buzzing. But apart from these lazinesses of logic, what makes the story so tired is the failure of the writer to reach for anything but the nearest clich\'e. ``Shouldered his way,'' ``only to be met,'' ``crashing into his face,'' ``waging a lonely war,'' ``corruption that is rife,'' ``sending shock waves,'' ``New York's finest'' -- these dreary phrases constitute writing at its most banal. We know just what to expect. No surprise awaits us in the form of an unusual word, an oblique look. We are in the hands of a hack, \& we know it right away. We stop reading.

Don't let yourself get in this position. The only way to avoid it is to care deeply about words. If you find yourself writing that someone recently enjoyed a spell of illness, or that a business has been enjoying a slump, ask yourself how much they enjoyed it. Notice the decisions that other writers make in their choice of words \& be finicky about the ones you select from the vast supply. The race in writing is not to the swift but to the original.

Make a habit of reading what is being written today \& what was written by earlier masters. Writing is learned by imitation. If anyone asked me how I learned to write, I'd say I learned by reading the men \& women who were doing the kind of writing \textit{I} wanted to do \& trying to figure out how they did it. But cultivate the best models. Don't assume that because an article is in a newspaper or a magazine it must be good. Sloppy editing is common in newspapers, often for lack of time, \& writers who use clich\'es often work for editors who have seen so many clich\'es that they no longer even recognize them.

Also get in the habit of using dictionaries. My favorite for handy use is \textit{Webster's New World Dictionary}, 2nd College Edition, although, like all word freaks, I own bigger dictionaries that will reward me when I'm on some more specialized search. If you have any doubt of what a word means, look it up. Learn its etymology \& notice what curious branches its original root has put forth. See if it has any meanings you didn't know it had. Master the small gradations between words that seem to be synonyms. What's the difference between ``cajole,'' ``wheedle,'' ``blandish'', \& ``coax''? Get yourself a dictionary of synonyms.

\& don't scorn that bulging grab bag \textit{Roget's Thesaurus}. It's easy to regard the book as hilarious. Look up ``villain,'' e.g., \& you'll be awash in such rascality as only a lexicographer could conjure back from centuries of iniquity, obliquity, depravity, knavery, profligacy, frailty, flagrancy, infamy, immorality, corruption, wickedness, wrongdoing, backsliding, \& sin. You'll find ruffians \& riffraff, miscreants \& malefactors, reprobates \& rapscallions, hooligans \& hoodlums, scamps \& scapegraces, scoundrels \& scalawags, jezebels \& jades. You'll find adjectives to fit them all (foul \& fiendish, devilish \& diabolical), \& adverbs \& verbs to describe how the wrongdoers do their wrong, \& cross-references leading to still other thickets of venality \& vice. Still, there's no better friend to have around to nudge the memory than \textit{Roget}. It saves you the time of rummaging in your brain -- that network of overloaded grooves -- to find the word that's right on the tip of your tongue, where it doesn't do you any good. The \textit{Thesaurus} is to the writer what a rhyming dictionary is to the songwriter -- a reminder of all the choices -- \& you should use it with gratitude. If, having found the scalawag \& the scapegrace, you want to know how they differ, \textit{then} go to the dictionary.

Also bear in mind, when you're choosing words \& stringing them together, how they sound. This may seem absurd: readers read with their eyes. But in fact they hear what they are reading far more than you realize. Therefore such matters as rhythm \& alliteration are vital to every sentence. A typical example -- maybe not the best, but undeniably the nearest -- is the preceding paragraph. Obviously I enjoyed making a certain arrangement of my ruffians \& riffraff, my hooligans \& hoodlums, \& my readers enjoyed it too -- far more than if I had provided a mere list. They enjoyed not only the arrangement but the effort to entertain them. They weren't enjoying it, however, with their eyes. They were hearing the words in their inner ear.

E. B. White makes the case cogently in \textit{The Elements of Style}, a book every writer should read once a year, when he suggests trying to rearrange any phrase that has survived for a century or 2, such as Thomas Paine's ``These are the time that try men's souls'':
\begin{quote}
	Times like these try men's souls.\\How trying it is to live in these times!\\These are trying times for men's souls.\\Soulwise, these are trying times.
\end{quote}
Paine's phrase is like poetry \& the other 4 are like oatmeal -- which is the divine mystery of the creative process. Good writers of prose must be part poet, always listening to what they write. E. B. White is 1 of my favorite stylists because I'm conscious of being with a man who cares about the cadences \& sonorities of the language. I relish (in my ear) the pattern his words make as they fall into a sentence. I try to surmise how in rewriting the sentence he reassembled it to end with a phrase that will momentarily linger, or how he chose 1 word over another because he was after a certain emotional weight. It's the difference between, say, ``serene'' \& ``tranquil'' -- one so soft, the other strangely disturbing because of the unusual \textit{n} \& \textit{q}.

Such considerations of sound \& rhythm should go into everything you write. If all your sentences move at the same plodding gait, which even you recognize as deadly but don't know how to cure, read them aloud. (I write entirely by ear \& read everything aloud before letting it go out into the world.) You'll begin to hear where the trouble lies. See if you can gain variety by reversing the order of a sentence, or by substituting a word that has freshness or oddity, or by altering the length of your sentences so they don't all sound as if they came out of the same machine. An occasional short sentence can carry a tremendous punch. It stays in the reader's ear.

Remember that words are the only tools you've got. Learn to use them with originality \& care. \& also remember: somebody out there is listening.'' -- \cite[pp. 37--40]{Zinsser2016}

%------------------------------------------------------------------------------%

\section{Usage}
``All this talk about good words \& bad words brings us to a gray but important area called ``usage.'' What is good usage? What is good English? What newly minted words is it O.K. to use, \& who is to be the judge? Is it O.K. to use ``O.K.''?

Earlier I mentioned an incident of college students hassling the administration, \& in the last chapter I described myself as a word freak. Here are 2 fairly recent arrivals. ``Hassle'' is both a verb \& a noun, meaning to give somebody a hard time, or the act of being given a hard time, \& anyone who has ever been hassled for not properly filling out Form 35-BX will agree that the word sounds exactly right. ``Freak'' means an enthusiast, \& there's no missing the aura of obsession that goes with calling someone a jazz freak, or a chess freak, or a sun freak, though it would probably be pushing my luck to describe a man who compulsively visits circus sideshows as a freak freak.

Anyway, I accept these 2 usages gladly. I don't consider them slang, or put quotation marks around them to show that I'm mucking about in the argot of the youth culture \& really know better. They're good words \& we need them. But I won't accept ``notables'' \& ``greats'' \& ``upcoming'' \& many other newcomers. They are cheap words \& we \textit{don't} need them.

Why is 1 word good \& another word cheap? I can't give you an answer, because usage has no fixed boundaries. Language is a fabric that changes from 1 week to another, adding new strands \& dropping old ones, \& even word freaks fight over what is allowable, often reaching their decision on a wholly subjective basis such as taste (``notables'' is sleazy). Which still leaves the question of who our tastemakers are.

The question was confronted in the 1960s by the editors of a brand-new dictionary, \textit{The American Heritage Dictionary}. They assembled a ``Usage Panel'' to help them appraise the new words \& dubious constructions that had come knocking at the door. Which ones should be ushered in, which thrown out on their ear? The panel consisted of 104 men \& women -- mostly writers, poets, editors, \& teachers -- who were known for caring about the language \& trying to use it well. I was a member of the panel, \& over the next few years I kept getting questionnaires. Would I accept ``finalize'' \& ``escalate''? How did I feel about ``It's me''? Would I allow ``like'' to be used as a conjunction -- like so many people do? How about ``mighty,'' as in ``mighty fine''?

We were told that in the dictionary our opinions would be tabulated in a separate ``Usage Note,'' so that readers could see how we voted. The questionnaire also left room for any comments we might feel impelled to make -- an opportunity the panelists seized avidly, as we found when the dictionary was published \& our comments were released to the press. Passions ran high. ``Good God, no! Never!'' cried Barbara W. Tuchman, asked about the verb ``to author.'' Scholarship hath no fury like that of a language purist faced with sludge, \& I shared Tuchman's vow that ``author'' should never be authorized, just as I agreed with Lewis Mumford that the adverb ``good'' should be ``left as the exclusive property of Ernest Hemingway.''

But guardians of usage are doing only half their job if they merely keep the language from becoming sloppy. Any dolt can rule that the suffix ``wise,'' as in ``healthwise,'' is doltwise, or that being ``rather unique'' is no more possible than being rather pregnant. The other half of the job is to help the language grow by welcoming any immigrant that will bring strength or color. Therefore I was glad that 97\% of us voted to admit ``dropout,'' which is clean \& vivid, but that only 47\% would accept ``senior citizen,'' which is typical of the pudgy new intruders from the land of sociology, where an illegal alien is now an undocumented resident. I'm glad we accepted ``escalate,'' the kind of verbal contraption I generally dislike but which the Vietnam war endowed with a precise meaning, complete with overtones of blunder.

I'm glad we took into full membership all sorts of robust words that previous dictionaries derided as ``colloquial'': adjectives like ``rambunctious,'' verbs like ``trigger'' \& ``rile,'' nouns like ``shambles'' \& ``tycoon'', \& ``trek,'' the latter approved by 78\% to mean any difficult trip, as in ``the commuter's daily trek to Manhattan.'' Originally it was a Cape Dutch word applied to the Boers' arduous journey by ox wagon. But our panel evidently felt that the Manhattan commuter's daily trek is no less arduous.

Still, 22\% were unwilling to let ``trek'' slip into general usage. That was the virtue of revealing how our panel voted -- it put our opinions on display, \& writers in doubt can conduct themselves accordingly. Thus our 95\% vote against ``myself,'' as in ``He invited Mary \& myself to dinner,'' a word condemned as ``prissy,'' ``horrible'' \& ``genteelism,'' ought to warn off anyone who doesn't want to be prissy, horrible or genteel. As Red Smith put it, ```Myself' is the refuge of idiots taught early that `me' is a dirty word.''

On the other hand, only 66\% of our panel rejected the verb ``to contact,'' once regarded as tacky, \& only half opposed the split infinitive \& the verbs ``to fault'' \& ``to bus.'' So only 50\% of your readers will fault you if you decide to voluntarily call your school board \& to bus your children to another town. If you contact your school board you risk your reputation by another 16\%. Our apparent rule of thumb was stated by Theodore M. Bernstein, author of the excellent \textit{The Careful Writer}: ``We should apply the test of convenience. Does the word fill a real need? If it does, let's give it a franchise.''

All of this confirms what lexicographers have always known: that the laws of usage are relative, bending with the taste of the lawmaker. 1 of our panelists, Katherine Anne Porter, called ``O.K.'' a ``detestable vulgarity'' \& claimed she had never spoken the word in her life, whereas I freely admit that I have spoken the word ``O.K.'' ``Most,'' as in ``most everyone,'' was scorned as ``cute farmer talk'' by Isaac Asimov \& embraced as a ``good English idiom'' by Virgil Thomson. ``Regime,'' meaning any administration, as in ``the Truman regime,'' drew the approval of most everyone on the panel, as did ``dynasty.'' But they drew the wrath of Jacques Barzun, who said, ``These are technical terms, you blasted non-historians!'' Probably I gave my O.K. to ``regime.'' Now, chided by Barzun for imprecision, I think it looks like journalese. 1 of the words \textit{I} railed against was ``personality,'' as in a ``TV personality.'' But now I wonder if it isn't the only word for that vast swarm of people who are famous for being famous -- \& possibly nothing else. What did the Gabor sisters actually \textit{do}?

In the end it comes down to what is ``correct'' usage. We have no king to establish the King's English; we only have the President's English, which we don't want. \textit{Webster}, long a defender of the faith, muddied the waters in 1961 with its permissive 3rd Edition, which argued that almost anything goes as long as somebody uses it, noting that ``ain't'' is ``used orally in most parts of the U.S. by many cultivated speakers.''

Just where \textit{Webster} cultivated those speakers I ain't sure. Nevertheless it's true that the spoken language is looser than the written language, \& \textit{The American Heritage Dictionary} properly put its question to us in both forms. Often we allowed an oral idiom that we forbade in print as too informal, fully realizing, however, that ``the pen must at length comply with the tongue,'' as Samuel Johnson said, \& that today's spoken garbage may be tomorrow's written gold. The growing acceptance of the split infinitive, or of the preposition at the end of a sentence, proves that formal syntax can't hold the fort forever against a speaker's more comfortable way of getting the same thing said -- \& it shouldn't. I think a sentence is a fine thing to put a preposition at the end of.

Our panel recognized that correctness can even vary within a word. We voted heavily against ``cohort'' as a synonym for ``colleague,' except when the tone was jocular. Thus a professor would not be among his cohorts at a faculty meeting, but they would abound at his college reunion, wearing funny hats. We rejected ``too'' as a synonym for ``very,'' as in ``His health is not too good.'' Whose health is? But we approved it in sardonic or humorous use, as in ``He was not too happy when she ignored him.''

These may seem like picayune distinctions. They're not. They are signals to the reader that you are sensitive to the shadings of usage. ``Too'' when substituted for ``very'' is clutter: ``He didn't feel too much like going shopping.'' But the wry example in the previous paragraph is worthy of Ring Lardner. It adds a tinge of sarcasm that otherwise wouldn't be there.

Luckily, a pattern emerged from the deliberations of our panel, \& it offers a guideline that is still useful. We turned out to be liberal in accepting new words \& phrases, but conservative in grammar.

It would be foolish to reject a word as perfect as ``dropout,'' or to pretend that countless words \& phrases are not entering the gates of correct usage every day, borne on the winds of science \& technology, business \& sports \& social change: ``outsource,'' ``blog,'' ``laptop,'' ``mousepad,'' ``geek,'' ``boomer,'' ``Google,'' ``iPod,'' ``hedge fund,'' ``24{\tt/}7,'' ``multi-tasking,'' ``slam dunk'' \& hundreds of others. Nor should we forget all the short words invented by the counterculture in the 1960s as a way of lashing back at the self-important verbiage o the Establishment: ``trip,'' ``rap,'' ``crash,'' ``trash,'' ``funky,'' ``split,'' ``rip-off,'' ``vibes,'' ``downer,'' ``bummer.'' If brevity is a prize, these were winners. The only trouble with accepting words that entered the language overnight is that they often leave just as abruptly. The ``happenings'' of the late 1960s no longer happen, ``out of sight'' is out of sight, \& even ``awesome'' has begun to chill out. The writer who cares about usage must always know the quick from the dead.

As for the area where ur Usage Panel was conservative, we upheld most of the classic distinctions in grammar -- ``can'' \& ``may,'' ``fewer'' \& ``less,'' ``eldest'' \& ``oldest,'' etc. -- \& decried the classic errors, insisting that ``flout'' still doesn't mean ``flaunt,'' no matter how many writers flaunt their ignorance by flouting the rule, \& that ``fortuitous'' still means ``accidental,'' ``disinterested'' still means ``impartial,'' \& ``infer'' doesn't mean ``imply.'' Here we were motivated by our love of the language's beautiful precision. Incorrect usage will lose you the readers you would most like to win. Know the difference between a ``reference'' \& an ``allusion,'' between ``connive'' \& ``conspire,'' between ``compare with'' \& ``compare to.'' If you must use ``comprise,'' use it right. It means ``include''; dinner comprises meat, potatoes, salad, \& dessert.

``I choose always the grammatical form unless it sounds affected,'' Marianne Moore explained, \& that's finally where our panel took its stand. We were not pedants, so hung up on correctness that we didn't want the language to keep refreshing itself with phrases like ``hung up.'' But that didn't mean we had to accept every atrocity that comes lumbering in.

Meanwhile the battle continues. Today I still receive ballots from \textit{The American Heritage Dictionary} soliciting my opinion on new locutions: verbs like ``definitize'' (``Congress definitized a proposal''), nouns like ``affordables,'' colloquialisms like ``the bottom line'' \& strays like ``into'' (``He's into backgammon \& she's into jogging'').

It no longer takes a panel of experts to notice that jargon is flooding our daily life \& language. President Carter signed an executive order directing that federal regulations be written ``simply \& clearly.'' President Clinton's attorney general, Janet Reno, urged the nation's lawyers to replace ``a lot of legalese'' with ``small, old words that all people understand'' -- ``words like ``right'' \& ``wrong'' \& ``justice.'' Corporations have hired consultants to make their prose less opaque, \& even the insurance industry is trying to rewrite its policies to tell us in less disastrous English what redress will be ours when disaster strikes. Whether these efforts will do much good I wouldn't want to bet. Still, there's comfort in the sight of so many watchdogs standing Canute-like on the beach, trying to hold back the tide. That's where all careful writers ought to be -- looking at every new piece of flotsam that washes up \& asking ``Do we need it?''

I remember the 1st time somebody asked me, ``How does that impact you?'' I always thought ``impact'' was a noun, except in dentistry. Then I began to meet ``de-impact,'' usually in connection with programs to de-impact the effects of some adversity. Nouns now turn overnight into verbs. We target goals \& we access facts. Train conductors announce that the train won't platform. A sign on an airport door tells me that the door is alarmed. Companies are downsizing. It's part of an ongoing effort to grow the business. ``Ongoing'' is a jargon word whose main use is to raise morale. We face our daily job with more zest if the boss tells us it's an ongoing project; we give more willingly to institutions if they have targeted our funds for ongoing needs. Otherwise we might fall prey to disincentivization.

I could go on; I have enough examples to fill a book, but it's not a book I would want anyone to read. We're still left with the question: What is good usage? 1 helpful approach is to try to separate usage from jargon.

I would say, e.g., that ``prioritize'' is jargon -- a pompous new verb that sounds more important than ``rank'' -- \& that ``bottom line'' is usage, a metaphor borrowed from the world of bookkeeping that conveys an image we can picture. As every businessman knows, the bottom line is the one that matters. If someone says, ``The bottom line is that we just can't work together,'' we know what he means. I don't much like the phrase, but the bottom line is that it's here to stay.

new usages also arrive with new political events. Just as Vietnam gave us ``escalate,'' Watergate gave us a whole lexicon of words connoting obstruction \& deceit, including ``deep-6,'' ``launder,'' ``enemies list'' \& other ``gate''-suffix scandals (``Irangate''). It's a fitting irony that under Richard Nixon ``launder'' become a dirty word. Today when we hear that someone laundered his funds to hide the origin of the money \& the route it took, the word has a precise meaning. It's short, it's vivid, \& we need it. I accept ``launder'' \& ``stonewall''; I don't accept ``prioritize'' \& ``disincentive.''

I would suggest a similar guideline for separating good English from technical English. It's the difference between, say, ``printout'' \& ``input.'' A printout is a specific object that a computer emits. Before the advent of computers it wasn't needed; now it is. But it ha stayed where it belongs. Not so with ``input,'' which was coined to describe the information that's fed to a computer. Our input is sought on every subject, from diets to philosophical discourse (``I'd like your input on whether God really exists'').

I don't want to give somebody my input \& get his feedback, though I'd be glad to offer my ideas \& hear what he thinks of them. Good usage, to me, consists of using good words if they already exist -- as they almost always do -- to express myself clearly \& simply to someone else. You might say it's how I verbalize the interpersonal.'' -- \cite[pp. 42--48]{Zinsser2016}

%------------------------------------------------------------------------------%

\begin{center}\huge
	Part II: Methods
\end{center}

%------------------------------------------------------------------------------%

\section{Unity}
``You learn to write by writing. It's a truism, but what makes it a truism is that it's true. The only way to learn to write is to force yourself to produce a certain number of words on a regular basis.

If you went to work for a newspaper that required you to write 2 or 3 articles every day, you would be a better writer after 6 months. You wouldn't necessarily be writing well; your style might still be full of clutter \& clich\'es. But you would be exercising your powers of putting the English language on paper, gaining confidence \& identifying the most common problems.

All writing is ultimately a question of solving a problem. It may be a problem of where to obtain the facts or how to organize the material. It may be a problem of approach or attitude, tone or style. Whatever it is, it has to be confronted \& solved. Sometimes you will despair of finding the right solution -- or any solution. You'll think, ``If I live to be 90 I'll never get out of this mess.'' I've often thought it myself. But when I finally do solve the problem it's because I'm like a surgeon removing his 500th appendix; I've been there before.

Unity is the anchor of good writing. So, 1st, get your unities straight. Unity not only keeps the reader from straggling off in all directions; it satisfies your readers' subconscious need for order \& reassures them that all is well at the helm. Therefore choose from among the many variables \& stick to your choice.

1 choice is unity of pronoun. Are you going to write in the 1st person, as a participant, or in the 3rd person, as an observer? Or even in the 2nd person, that darling of sportswriters hung up on Hemingway? (``You knew this had to be the most spine-tingling clash of giants you'd ever seen from a pressbox seat, \& you weren't just some green kid who was still wet behind the ears.'')

Unity of tense is another choice. Most people write mainly in the past tense (``I went up to Boston the other day''), but some people write agreeably in the present (``I'm sitting in the dining car of the Yankee Limited \& we're pulling into Boston''). What is not agreeable is to switch back \& forth. I'm not saying you can't use $> 1$ tense; the whole purpose of tenses is to enable a writer to deal with time in its various gradations, from the past to the hypothetical future (``When I telephoned my mother from the Boston station, I realized that if I had written to tell her I would be coming she would have waited for me''). But you must choose the tense in which you are \textit{principally} going to address the reader, no matter how many glances you may take backward \& forward along the way.

Another choice is unity of mood. You might want to talk to the reader in the casual voice that \textit{The New Yorker} has strenuously refined. Or you might want to approach the reader with a certain formality to describe a serious event or to present a set of important facts. Both tones are acceptable. In fact, \textit{any} tone is acceptable. But don't mix 2 or 3.

Such fatal mixtures are common in writers who haven't learned control. Travel writing is a conspicuous example. ``My wife, Ann, \& I had always wanted to visit Hong Kong,'' the writer begins, his blood astir with reminiscence, ``\& 1 day last spring we found ourselves looking at an airline poster '\& I said, `Let's go!' The kids were grown up,'' he continues, \& he proceeds to describe in genial detail how he \& his wife stopped off in Hawaii \& had such a comical time changing their money at the Hong Kong airport \& finding their hotel. Fine. He is a real person taking us along on a real trip, \& we can identify with him \& Ann.

Suddenly he turns into a travel brochure. ``Hong Kong affords many fascinating experiences to the curious sightseer,'' he writes. ``One can ride the picturesque ferry from Kowloon \& gawk at the myriad sampans as they scuttle across the teeming harbor, or take a day's trip to browse in the alleys of fabled Macao with its colorful history as a den of smuggling \& intrigue. You will want to take the quaint funicular that climbs $\ldots$'' Then we get back to him \& Ann \& their efforts to eat at Chinese restaurants, \& again all is well. Everyone is interested in food, \& we are being told about a personal adventure.

Then suddenly the writer is a guidebook: ``To enter Hong Kong it is necessary to have a valid passport, but no visa is required. You should definitely be immunized against hepatitis \& you would also be well advised to consult your physician with regard to a possible inoculation for typhoid. The climate in Hong Kong is seasonable except in Jul \& Aug when $\ldots$'' Our writer is gone, \& so is Ann, \& so -- very soon -- are we.

It's not that the scuttling sampans \& the hepatitis shots shouldn't be included. What annoys us is that the writer never decided what kind of article he wanted to write or how he wanted to approach us. He comes at us in many guises, depending on what kind of material he is trying to purvey. Instead of controlling his material, his material is controlling him. That wouldn't happen if he took time to establish certain unities.

Therefore ask yourself some basic questions before you start. E.g.: ``In what capacity am I going to address the reader?'' (Reporter? Provider of information? Average man or woman?) ``What pronoun \& tense am I going to use?'' ``What style?'' (Impersonal reportorial? Personal but formal? Personal \& casual?) ``What attitude am I going to take toward the material?'' (Involved? Detached? Judgmental? Ironic? Amused?) ``How much do I want to cover?'' ``What 1 point do I want to make?''

The last 2 questions are especially important. Most nonfiction writers have a definitiveness complex. They feel that they are under some obligation -- to the subject, to their honor, to the gods of writing -- to make their article the last word. It's a commendable impulse, but there is no last word. What you think is definitive today will turn undefinitive by tonight, \& writers who doggedly pursue every last fact will find themselves pursuing the rainbow \& never settling down to write. Nobody can write a book or an article ``about'' something. Tolstoy couldn't write a book about war \& peace, or Melville a book about whaling. They made certain reductive decisions about time \& place \& about individual characters in that time \& place -- 1 man pursuing 1 whale. Every writing project must be reduced before you start to write.

Therefore think small. Decide what corner of your subject you're going to bite off, \& be content to cover it well \& stop. This is also a matter of energy \& morale. An unwieldy writing task is a drain on your enthusiasm. Enthusiasm is the force that keeps you going \& keeps the reader in your grip. When your zest begins to ebb, the reader is the 1st person to know it.

As for what point you want to make, every successful piece of nonfiction should leave the reader with 1 provocative thought that he or she didn't have before. Not 2 thoughts, or 5 -- just 1. So decided what single point you want to leave in the reader's mind. It will not only give you a better idea of what route you should follow \& what destination you hope to reach; it will affect your decision about tone \& attitude. Some points are best made by earnestness, some by dry understatement, some by humor.

Once you have your unities decided, there's no material you can't work into your frame. If the tourist in Hong Kong had chosen to write solely in the conversational vein about what he \& Ann did, he would have found a natural way to weave into his narrative whatever he wanted to tell us about the Kowloon ferry \& the local weather. His personality \& purpose would have been intact, \& his article would have held together.

Now it often happens that you'll make these prior decisions \& then discover that they weren't the right ones. The material begins to lead you in an unexpected direction, where you are more comfortable writing in a different tone. That's normal -- the act of writing generates some cluster of thoughts or memories that you didn't anticipate. Don't fight such a current if it feels right. Trust your material if it's taking you into terrain you didn't intend to enter but where the vibrations are good. Adjust your style accordingly \& proceed to whatever destination you reach. Don't become the prisoner of a preconceived plan. Writing is no respecter of blueprints.

If this happens, the 2nd part of your article will be badly out of joint with the 1st. But at least you know which part is truest to your instincts. Then it's just a matter of making repairs. Go back to the beginning \& rewrite it so that your mood \& your style are consistent from start to finish.

There's nothing in such a method to be ashamed of. Scissors \& paste -- or their equivalent on a computer -- are honorable writers' tools. Just remember that all the unities must be fitted into the edifice you finally put together, however backwardly they may be assembled, or it will soon come tumbling down.'' -- \cite[pp. 52--55]{Zinsser2016}

%------------------------------------------------------------------------------%

\section{The Lead \& the Ending}
``The most important sentence in any article is the 1st one. If it doesn't induce the reader to proceed to the 2nd sentence, your article is dead. \& if the 2nd sentence doesn't include him to continue to the 3rd sentence, it's equally dead. Of such a progression of sentence, each tugging the reader forward until he is hooked, a writer constructs that fateful unit, the ``lead.''

How long should the lead be? 1 or 2 paragraphs? 4 or 5? There's no pat answer. Some leads hook the reader with just a few well-baited sentences; others amble on for several pages, exerting a slow but steady pull. Every article poses a different problem, \& the only valid test is: does it work? Your lead may not be the best of all possible leads, but if it does the job it's supposed to do, be thankful \& proceed.

Sometimes the length may depend on the audience you're writing for. Readers of a literary review expect its writers to start somewhat discursively, \& they will stick with those writers for the pleasure of wondering where they will emerge as they move in leisurely circles toward the eventual point. But I urge you not to count on the reader to stick around. Readers want to know -- very soon -- what's in it for them.

Therefore your lead must capture the reader immediately \& force him to keep reading. It must cajole him with freshness, or novelty, or paradox, or humor, or surprise, or with an unusual idea, or an interesting fact, or a question. Anything will do, as long as it nudges his curiosity \& tugs at his sleeve.

Next the lead must do some real work. It must provide hard details that tell the reader why the piece was written \& why he ought to read it. But don't dwell on the reason. Coax the reader a little more; keep him inquisitive.

Continue to build. Every paragraph should amplify the one that preceded it. Give more thought to adding solid detail \& less to entertaining the reader. But take special care with the last sentence of each paragraph -- it's the crucial springboard to the next paragraph. Try to give that sentence an extra twist of humor or surprise, like the periodic ``snapper'' in the routine of a stand-up comic. Make the reader smile \& you've got him for at least 1 more paragraph.

Let's look at a few leads that vary in pace but are alike in maintaining pressure. I'll start with 2 columns of my own that 1st appeared in \textit{Life} \& \textit{Look} -- magazines which, judging by the comments of readers, found their consumers mainly in barbershops, hairdressing salons, airplanes \& doctors' offices (``I was getting a haircut the other day \& I saw your article''). I mention this as a reminder that far more periodical reading is done under the dryer than under the reading lamp, so there isn't much time for the writer to fool around.

The 1st is the lead of a piece called ``Block That Chickenfurter'':
\begin{quotation}
	I've often wondered what goes into a hot dog. Now I know \& I wish I didn't.
\end{quotation}
2 very short sentences. But it would be hard not to continue to the 2nd paragraph:
\begin{quotation}
	My trouble began when the Department of Agriculture published the hot dog's ingredients -- everything that may legally qualify -- because it was asked by the poultry industry to relax the conditions under which the ingredients might also include chicken. In other words, can a chickenfurter find happiness in the land of the frank?
\end{quotation}
1 sentence that explains the incident that the column is based on. Then a snapper to restore the easygoing tone.
\begin{quotation}
	Judging by the 1066 mainly hostile answers that the Department got when it sent out a questionnaire on this point, the very thought is unthinkable. The public mood was most felicitously caught by the woman who replied: ``I don't eat feather meat of no kind.''
\end{quotation}
Another fact \& another smile. Whenever you're lucky enough to get a quotation as funny as that one, find a way to use it. The article then specifies what the Department of Agriculture says may go into a hot dog -- a list that includes ``the edible part of the muscle of cattle, sheep, swine or goats, in the diaphragm, in the heart or in the esophagus $\ldots$ [but not including] the muscle found in the lips, snout or ears.''

From there it progresses -- not without an involuntary reflex around the esophagus -- into an account of the controversy between the poultry interests \& the frankfurter interests, which in turn leads to the point that Americans will eat anything that even remotely resembles a hot dog. Implicit at the end is the larger point that Americans don't know, or care, what goes into the food they eat. The style of the article has remained casual \& touched with humor. But its content turns out to be more serious than readers expected when they were drawn into it by a whimsical lead.

A slower lead, luring the reader more with curiosity than with humor, introduced a piece called ``Thank God for Nuts'':
\begin{quotation}
	By any reasonable standard, nobody would want to look twice -- or even once -- at the piece of slippery elm bark from Clear Lake, Wisc., birthplace of pitcher Burleigh Grimes, that is on display at the National Baseball Museum \& Hall of Fame in Cooperstown, N.Y. As the label explains, it is the kind of bark Grimes chewed during games ``to increase saliva for throwing the spitball. When wet, the ball sailed to the plate in deceptive fashion.'' This would seem to be 1 of the least interesting facts available in America today.
	
	But baseball fans can't be judged by any reasonable standard. We are obsessed by the minutiae of the game \& nagged for the rest of our lives by the memory of players we once saw play. No item is therefore too trivial that puts us back in touch with them. I am just old enough to remember Burleigh Grimes \& his well-moistened pitches sailing deceptively to the plate, \& when I found his bark I studied it as intently as if I had come upon the Rosetta Stone. ``So \textit{that's} how he did it,'' I thought, peering at the old botanical relic. ``Slippery elm! I'll be damned.''
	
	This was only 1 of several hundred encounters I had with my own boyhood as I prowled through the Museum. Probably no other museum is so personal a pilgrimage to our past $\ldots$
\end{quotation}
The reader is now safely hooked, \& the hardest part of the writer's job is over.

1 reason for citing this lead is to note that salvation often lies not in the writer's style but in some odd fact he or she was able to discover. I went up to Cooperstown \& spent a whole afternoon in the museum, taking notes. Jostled everywhere by nostalgia, I gazed with reverence at Lou Gehrig's locker \& Bobby Thomson's game-winning bat. I sat in a grandstand seat brought from the Polo Grounds, dug my unspiked soles into the home plate from Ebbets Field, \& dutifully copied all the labels \& captions that might be useful.

``These are the shoes that touched home plate as Ted finished his journey around the bases,'' said a label identifying the shoes worn by Ted Williams when he famously hit a home run on his last time at bat. The shoes were in much better shape than the pair -- rotted open at the sides -- that belonged to Walter Johnson. But the caption provided exactly the kind of justifying fact a baseball nut would want. ``My feet must be comfortable when I'm out there a-pitching,'' the great Walter said.

The museum closed at 5 \& I returned to my motel secure in my memories \& my research. But instinct told me to go back the next morning for 1 more tour, \& it was only then that I noticed Burleigh Grimes's slippery elm bark, which struck me as an ideal lead. It still does.

1 moral of this story is that you should always collect more material than you will use. Every article is strong in proportion to the surplus of details from which you can choose the few that will serve you best -- if you don't go on gathering facts forever. At some point you must stop researching \& start writing.

Another moral is to look for your material everywhere, not just by reading the obvious sources \& interviewing the obvious people. Look at signs \& at billboards \& at all the junk written along the American roadside. Read the labels on our packages \& the instructions on our toys, the claims on our medicines \& the graffiti on our walls. Read the fillers, so rich in self-esteem, that come spilling out of your monthly statement from the electric company \& the telephone company \& the bank. Read menus \& catalogues \& 2nd-class mail. Nose about in obscure crannies of the newspaper, like the Sunday real estate section -- you can tell the temper of a society by what patio accessories it wants. Our daily landscape is thick with absurd messages \& portents. Notice them. They not only have social significance; they are often just quirky enough to make a lead that's different from everybody else's.

Speaking of everybody else's lead, there are many categories I'd be glad never to see again. One is the future archaeologist: ``When some future archaeologist stumbles on the remains of our civilization, what will he make of the jukebox?'' I'm tired of him already \& he's not even here. I'm also tired of the visitor from Mars: ``If a creature from Mars landed on our planet he would be amazed to see hordes of scantily clad earthlings lying on the sand barbecuing their skins.'' I'm tired of the cute event that just happened to happen ``1 day not long ago'' or on a conveniently recent Saturday afternoon: ``1 day not long ago a small button-nosed boy was walking with his dog, Terry, in a field outside Paramus, N.J., when he saw something that looked strangely like a balloon rising out of the ground.'' \& I'm very tired of the have-in-common lead: ``What did Joseph Stalin, Douglas MacArthur, Ludwig Wittgenstein, Sherwood Anderson, Jorge Luis Borges \& Akira Kurosawa have in common? They all loved Westerns.'' Let's retire the future archaeologist \& the man from Mars \& the button-nosed boy. Try to give your lead a freshness of perception or detail.

Consider this lead, by Joan Didion, on a piece called ``7000 Romaine, Los Angeles 38'':
\begin{quotation}
	7000 Romaine Street is in that part of Los Angeles familiar to admirers of Raymond Chandler \& Dashiell Hammett: the underside of Hollywood, south of Sunset Boulevard, a middle-class slum of ``model studios'' \& warehouses \& 2-family bungalows. Because Paramount \& Columbia \& Desilu \& the Samuel Goldwyn studios are nearby, many of the people who live around here have some tenuous connection with the motion-picture industry. They once processed fan photographs, say, or knew Jean Harlow's manicurist. 7000 Romaine looks itself like a faded movie exterior, a pastel building with chipped \textit{art moderne} detailing, the windows now either boarded or paned with chicken-wire glass \&, at the entrance, among the dusty oleander, a rubber mat that reads \textsc{Welcome}.
	
	Actually no one is welcome, for 7000 Romaine belongs to Howard Hughes, \& the door is locked. That the Hughes ``communications center'' should lie here in the dull sunlight of Hammett-Chandler country is 1 of those circumstances that satisfy one's suspicion that life is indeed a scenario, for the Hughes empire has been in our time the only industrial complex in the world -- involving, over the years, machinery manufacture, foreign oil-tool subsidiaries, a brewery, 2 airlines, immense real-estate holdings, a major motion-picture studio, \& an electronics \& missile operation -- run by a man whose \textit{modus operandi} most resembles that of a character in \textit{The Big Sleep}.
	
	As it happens, I live no far from 7000 Romaine, \& I make a point of driving past it every now \& then, I suppose in the same spirit that Arthurian scholars visit the Cornish coast. I am interested in the folklore of Howard Hughes $\ldots$
\end{quotation}
What is pulling us into this article -- toward, we hope, some glimpse of how Hughes operates, some hint of the riddle of the Sphinx -- is the steady accumulation of facts that have pathos \& faded glamour. Knowing Jean Harlow's manicurist is such a minimal link to glory, the unwelcoming welcome mat such a queer relic of a golden age when Hollywood's windows weren't paned with chicken-wire glass \& the roost was ruled by giants like Mayer \& DeMille \& Zanuck, who could actually be seen exercising their mighty power. We want to know more; we read on.

Another approach is to just tell a story. It's such a simple solution, so obvious \& unsophisticated, that we often forget that it's available to us. But narrative is the oldest \& most compelling method of holding someone's attention; everybody wants to be told a story. Always look for ways to convey your information in narrative form. What follows is the lead of Edmund Wilson's account of the discovery of the Dead Sea Scrolls, 1 of the most astonishing relics of antiquity to turn up in modern times. Wilson doesn't spend any time setting the stage. This is not the ``breakfast-to-bed'' format used by inexperienced writers, in which a fishing trip begins with the ringing of an alarm clock before daylight. Wilson starts right in -- whap! -- \& we are caught:
\begin{quotation}
	At some point rather early in the spring of 1947, a Bedouin boy called Muhammed the Wolf was minding some goats near a cliff on the western shore of the Dead Sea. Climbing up after one that had strayed, he noticed a cave that he had not seen before, \& he idly threw a stone into it. There was an unfamiliar sound of breakage. The boy was frightened \& ran way. But he later came back with another boy, \& together they explored the cave. Inside were several tall clay jars, among fragments of other jars. When they took off the bowl-like lids, a very bad smell arose, which came from dark oblong lumps that were found inside all the jars. When they got these lumps out of the cave, they saw that they were wrapped up in lengths of linen \& coated with a black layer of what seemed to be pitch or wax. They unrolled them \& found long manuscripts, inscribed in parallel columns on thin sheets that had been sewn together. Though these manuscripts had faded \& crumbled in places, they were in general remarkably clear. The character, they saw, was not Arabic. They wondered at the scrolls \& kept them, carrying them along when they moved.
	
	These Bedouin boys belonged to a party of contrabanders, who had been smuggling their goats \& other goods out of Transjordan into Palestine. They had detoured so far to the south in order to circumvent the Jordan bridge, which the customs officers guarded with guns, \& had floated their commodities across the stream. They were now on their way to Bethlehem to sell their stuff in the black market $\ldots$
\end{quotation}
Yet there can be no firm rules for how to write a lead. Within the broad rule of not letting the reader get away, all writers must approach their subject in a manner that most naturally suits what they are writing about \& who they are. Sometimes you can tell your whole story in the 1st sentence. Here's the opening sentence of 7 memorable nonfiction books:
\begin{quotation}
	``In the beginning God created heaven \& earth.'' -- \textit{The Bible}
	
	``In the summer of the Roman year 699, now described as the year 55 before the birth of Christ, the Proconsul of Gaul, Gaius Julius Caesar, turned his gaze upon Britain.'' -- Winston S. Churchill, \textit{A History of The English-Speaking Peoples}
	
	``Put this puzzle together \& you will find milk, cheese \& eggs, meat, fish, beans \& cereals, greens, fruits \& root vegetables -- foods that contain our essential daily needs.'' -- Irma S. Rombauer, \textit{Joy of Cooking}
	
	``To the Manus native the world is a great platter, curving upwards on all sides, from his flat lagoon village where the pile-houses stand like long-legged birds, placid \& unstirred by the changing tides.'' -- Margaret Mead, \textit{Growing Up in New Guinea}
	
	``The problem lay buried, unspoken, for many years in the minds of American women.'' -- Betty Friedan, \textit{The Feminine Mystique}
	
	``Within 5 minutes, or 10 minutes, no more than that, 3 of the others had called her on the telephone to ask her if she had heard that something had happened out there.'' -- Tom Wolfe, \textit{The Right Stuff}
	
	``You know more than you think you do.'' -- Benjamin Spock, \textit{Baby \& Child Care}
\end{quotation}
Those are some suggestions on how to get started. Now I want to tell you how to stop. Knowing when to end an article is far more important than most writers realize. You should give as much thought to choosing your last sentence as you did to your 1st. Well, almost as much.

That may seem hard to believe. If your readers have stuck with you from the beginning, trailing you around blind corners \& over bumpy terrain, surely they don't leave when the end is in sight. Surely they will, because the end that's in sight turns out to be a mirage. Like the minister's sermon that builds to a series of perfect conclusions that never conclude, an article that doesn't stop where it should stop becomes a drag \& therefore a failure.

Most of us are still prisoners of the lesson pounded into us by the composition teachers of our youth: that every story must have a beginning, a middle, \& an end. We can still visualize the outline, with its Roman numerals (I, II, \& III), which staked out the road we would faithfully trudge, \& its subnumerals (IIa \& IIb) denoting lesser paths down which we would briefly poke. But we always promised to get back to III \& summarize our journey.

That's all right for elementary \& high school students uncertain of their ground. It forces them to see that every piece of writing should have a logical design. It's a lesson worth knowing at any age -- even professional writers are adrift more often than they would like to admit. But if you're going to write good nonfiction you must wriggle out of III's dread grip.

You'll know you have arrived at III when you see emerging on your screen a sentence that begins, ``In sum, it can be noted that $\ldots$'' Or a question that asks, ``What insights, then, have we been able to glean from $\ldots$?'' These are signals that you are about to repeat in compressed form what you have already said in detail. The reader's interest begins to falter; the tension you have built begins to sag. Yet you will be true to Miss Potter, your teacher, who made you swear fealty to the holy outline. You remind the reader of what can, in sum, be noted. You go gleaning 1 more time in insights you have already adduced.

But your readers hear the laborious sound of cranking. They notice what you are doing \& how bored you are by it. They feel the stirrings of resentment. Why didn't you give more thought to how you were going to wind this thing up? Or are you summarizing because you think they're too dumb to get the point? Still, you keep cranking. But the readers have another option. They quit.

That's the negative reason for remembering the importance of the last sentence. Failure to know where that sentence should occur can wreck an article that until its final stage has been tightly constructed. The positive reason for ending well is that a good last sentence -- or last paragraph -- is a joy in itself. It gives the reader a lift, \& it lingers when the article is over.

The perfect ending should take your readers slightly by surprise \& yet seem exactly right. They didn't expect the article to end so soon, or so abruptly, or to say what it said. But they know it when they see it. Like a good lead, it works. It's like the curtain line in a theatrical comedy. We are in the middle of a scene (we think), when suddenly 1 of the actors says something funny, or outrageous, or epigrammatic, \& the lights go out. We are startled to find the scene over, \& then delighted by the aptness of how it ended. What delights us is the playwright's perfect control.

For the nonfiction writer, the simplest way of putting this into a rule is: when you're ready to stop, stop. If you have presented all the facts \& made the point you want to make, look for the nearest exit.

Often it takes just a few sentences to wrap things up. Ideally they should encapsulate the idea of the piece \& conclude with a sentence that jolts us with its fitness \& unexpectedness. Here's how H. L. Mencken ends his appraisal of President Calvin Coolidge, whose appeal to the ``customers'' was that his ``governments governed hardly at all; thus the ideal of Jefferson was realized at last, \& the Jeffersonians were delighted'':
\begin{quotation}
	We suffer most, not when the White House is a peaceful dormitory, but when it [has] a tin-pot Paul bawling from the roof. Counting out Harding as a cipher only, Dr. Coolidge was preceded by 1 World Saver \& followed by 2 more. What enlightened American, having to choose between any of them \& another Coolidge, would hesitate for an instant? There were no thrills while he reigned, but neither were there any headaches. He had no ideas, \& he was not a nuisance.
\end{quotation}
The 5 short sentences send the reader on his way quickly \& with an arresting thought to take along. The notion of Coolidge having no ideas \& not being a nuisance can't help leaving a residue of enjoyment. It works.

Something I often do in my writing is to bring the story full circle -- to strike at the end an echo of a note that was sounded at the beginning. It gratifies my sense of symmetry, \& it also pleases the reader, completing with its resonance the journey we set out on together.

But what usually works best is a quotation. Go back through your notes to find some remark that has a sense of finality, or that's funny, or that adds an unexpected closing detail. Sometimes it will jump out at you during the interview -- I've often thought, ``That's my ending!'' -- or during the process of writing. In the mid-1960s, when Woody Allen was just becoming established as America's resident neurotic, doing nightclub monologues, I wrote the 1st long magazine piece that took note of his arrival. It ended like this:
\begin{quotation}
	``If people come away relating to me as a person,'' Allen says, ``rather than just enjoying my jokes; if they come away wanting to hear me again, no matter what I might talk about, then I'm succeeding.'' Judging by the returns, he is. Woody Allen is Mr. Related-To, \& he seems a good bet to hold the franchise for many years.
	
	Yet he does have a problem all his own, unshared by, unrelated to, the rest of America. ``I'm obsessed,'' he says, ``by the fact that my mother genuinely resembles Groucho Marx.''
\end{quotation}
There's a remark from so far out in left field that nobody could see it coming. The surprise it carries is tremendous. How could it not be a perfect ending? Surprise is the most refreshing element in nonfiction writing. If something surprises you it will also surprise -- \& delight -- the people you are writing for, especially as you conclude your story \& send them on their way.'' -- \cite[pp. 57--67]{Zinsser2016}

%------------------------------------------------------------------------------%

\section{Bits \& Pieces}
``This is a chapter of scraps \& morsels -- small admonitions on many points that I have collected under one, as they say, umbrella.

\subsection{Verbs}
Use active verbs unless there is no comfortable way to get around using a passive verb. The difference between an active-verb style \& a passive-verb style -- in clarity \& vigor -- is the difference between life \& death for a writer.

``Joe saw him'' is strong. ``He was seen by Joe'' is weak. The 1st is short \& precise; it leaves no doubt about who did what. The 2nd is necessarily longer \& it has an insipid quality: something was done by somebody to someone else. It's also ambiguous. How often was he seen by Joe? Once? Every day? Once a week? A style that consists of passive constructions will sap the reader's energy. Nobody ever quite knows what is being perpetrated by whom \& on whom.

I use ``perpetrated'' because it's the kind of word that passive-voice writers are fond of. They prefer long words of Latin origin to short Anglo-Saxon words -- which compounds their trouble \& makes their sentences still more glutinous. Short is better than long. Of the 701 words in Lincoln's 2nd Inaugural Address, a marvel of economy in itself, 505 are words of 1 syllable \& 122 are words of 2 syllables.

Verbs are the most important of all your tools. They push the sentence toward \& give it momentum. Active verbs push hard; passive verbs tug fitfully. Active verbs also enable us to visualize an activity because they require a pronoun (``he''), or a noun (``the boy''), or a person (``Mrs. Scott'') to put them in motion. Many verbs also carry in their imagery or in their sound a suggestion of what they mean: glitter, dazzle, twirl, beguile, scatter, swagger, poke, pamper, vex. Probably no other language has such a vast supply of verbs so bright with color. Don't choose one that is dull or merely serviceable. Make active verbs activate your sentences, \& avoid the kind that need an appended preposition to complete their work. Don't set up a business that you can start or launch. Don't say that the president of the company stepped down. Did he resign? Did he retire? Did he get fired? Be precise. Use precise verbs.

If you want to see how active verbs give vitality to the written word, don't just go back to Hemingway or Thurber or Thoreau. I commend the King James Bible \& William Shakespeare.

\subsection{Adverbs}
``Most adverbs are unnecessary. You will clutter your sentence \& annoy the reader if you choose a verb that has a specific meaning \& then add an adverb that carries the same meaning. Don't tell us that the radio blared loudly; ``blare'' connotes loudness. Don't write that someone clenched his teeth tightly; there's no other way to clench teeth. Again \& again in careless writing, strong verbs are weakened by redundant adverbs. So are adjective \& other parts of speech: ``effortlessly easy,'' ``slightly spartan,'' ``totally flabbergasted.'' The beauty of ``flabbergasted'' is that it implies an astonishment that is total; I can't picture someone being partly flabbergasted. If an action is so easy as to be effortless, use ``effortless.'' \& what is ``slightly spartan''? Perhaps a monk's cell with wall-to-wall carpeting. Don't use adverbs unless they do necessary work. Spare us the news that the winning athlete grinned widely.

\& while we're at it, let's retire ``decidedly'' \& all its slippery cousins. Every day I see in the paper that some situations are decidedly better \& others are decidedly worse, but I never know how decided the improvement is, or who did the deciding, just as I never know how eminent a result is that's eminently fair, or whether to believe a fact that's arguably true. ``He's arguably the best pitcher on the Mets,'' the preening sportswriter writes, aspiring to Parnassus, which Red Smith reached by never using words like ``arguably.'' Or is he \textit{perhaps} -- the opinion is open to argument -- the best pitcher? Admittedly I don't know. It's virtually a toss-up.

\subsection{Adjectives}
Most adjectives are also unnecessary. Like adverbs, they are sprinkled into sentences by writers who don't stop to think that the concept is already in the noun. This kind of prose is littered with precipitous cliffs \& lacy spiderwebs, or which adjectives denoting the color of an object whose color is well known: yellow daffodils \& brownish dirt. If you want to make a value judgment about daffodils, choose an adjective like ``garish.'' If you're in a part of the country where the dirt is red, feel free to mention the red dirt. Those adjectives would do a job that the noun alone wouldn't be doing.

Most writers sow adjectives most unconsciously into the soil of their prose to make it more lush \& pretty, \& the sentences become longer \& longer as they fill up with stately elms \& frisky kittens \& hard-bitten detectives \& sleepy lagoons. This is adjective-by-habit -- a habit you should get rid of. Not every oak has to be gnarled. The adjective that exists soely as decoration is a self-indulgence for the writer \& a burden for the reader.

Again, the rule is simple: make your adjectives do work that needs to be done. ``He looked at the gray sky \& the black clouds \& decided to sail back to the harbor.'' The darkness of the sky \& the clouds is the reason for the decision. If it's important to tell the reader that a house was drab or a girl was beautiful, by all means use ``drab'' \& ``beautiful.'' They will have their proper power because you have learned to use adjectives sparsely.

\subsection{Little Qualifiers}
Prune out the small words that qualify how you feel \& how you think \& what you saw: ``a bit,'' ``a little,'' ``sort of,'' ``kind of,'' ``rather,'' ``quite,'' ``very,'' ``too,'' ``pretty much,'' ``in a sense'', \& dozens more. They dilute your style \& your persuasiveness.

Don't say you were a bit confused \& sort of tired \& a little depressed \& somewhat annoyed. Be confused. Be tired. Be depressed. Be annoyed. Don't hedge your prose with little timidities. Good writing is lean \& confident.

Don't say you weren't too happy because the hotel was pretty expensive. Say you weren't happy because the hotel was expensive. Don't tell us you were quite fortunate. How fortunate is that? Don't describe an event as rather spectacular or very awesome. Words like ``spectacular'' \& ``awesome'' don't submit to measurement. ``Very'' is a useful word to achieve emphasis, but far more often it's clutter. There's no need to call someone very methodical. Either he is methodical or he isn't.

The larger point is 1 of authority. Every little qualifier whittles away some fraction of the reader's trust. Readers want a writer who believes in himself \& in what he is saying. Don't diminish that belief. Don't be kind of bold. Be bold.

\subsection{Punctuation}
These are brief thoughts on punctuation, in no way intended as a primer. If you don't know how to punctuate -- \& many college students still don't -- get a grammar book.

\textbf{The Period.} There's not much to be said about the period except that most writers don't reach it soon enough. If you find yourself hopelessly mired in a long sentence, it's probably because you're trying to make the sentence do more than it can reasonably do -- perhaps express 2 dissimilar thoughts. The quickest way out is to break the long sentence into 2 short sentences, or even 3. There is no minimum length for a sentence that's acceptable in the eyes of God. Among good writers it is the short sentence that predominates, \& don't tell me about Norman Mailer -- he's a genius. If you want to write long sentences, be a genius. Or at least make sure that the sentence is under control from beginning to end, in syntax \& punctuation, so that the reader knows where he is at every step of the winding trail.

\textbf{The Exclamation Point.} Don't use it unless you must to achieve a certain effect. It has a gushy aura, the breathless excitement of a debutante commenting on an event that was exciting only to her: ``Daddy says I must have had too much champagne!'' ``But honestly, I could have danced all night!'' We have all suffered more than our share of these sentences in which an exclamation point knocks us over the head with how cute or wonderful something was. Instead, construct your sentence so that the order of the words will put the emphasis where you want it. Also resist using an exclamation point to notify the reader that you are making a joke or being ironic. ``It never occurred to me that the water pistol might be loaded!'' Readers are annoyed by your reminder that this was a comical moment. They are also robbed of the pleasure of finding it funny on their own. Humor is best achieved by understatement, \& there's nothing subtle about an exclamation point.

\textbf{The Semicolon.} There is a 19th-century mustiness that hangs over the semicolon. We associate it with the carefully balanced sentences, the judicious weighing of ``on the 1 hand'' \& ``on the other hand,'' of Conrad \& Thackeray \& Hardy. Therefore it should be used sparingly by modern writers of nonfiction. Yet I notice that it turns up quite often in the passages I've quoted in this book \& that I use it often myself -- usually to add a related thought to the 1st half of a sentence. Still, the semicolon brings the reader, if not to a halt, at least to a pause. So use it with discretion, remembering that it will slow to a Victorian pace the early-21st-century momentum you're striving for, \& rely instead on the period \& the dash.

\textbf{The Dash.} Somehow this invaluable tool is widely regarded as not quite proper -- a bumpkin at the genteel dinner table of good English. But it has full membership \& will get you out of many tight corners. The dash is used in 2 ways. One is to amplify or justify in the 2nd part of the sentence a thought you stated in the 1st part. ``We decided to keep going -- it was only 100 miles more \& we could get there in time for dinner.'' By its very shape the dash pushes the sentence ahead \& explains why they decided to keep going. The other use involves 2 dashes, which set apart a parenthetical thought within a longer sentence. ``She told me to get in the car -- she had been after me all summer to have a haircut -- \& we drove silently into town.'' An explanatory detail that might otherwise have required a separate sentence is neatly dispatched along the way.

\textbf{The Colon.} The colon has begun to look even more antique than the semicolon, \& many of its functions have been taken over by the dash. But it still serves well its pure role of bringing your sentence to a brief halt before you plunge into, say, an itemized list. ``The brochure said the ship would stop at the following ports: Oran, Algiers, Naples, Brindisi, Piraeus, Istanbul, \& Beirut.'' You can't beat the colon for work like that.

\subsection{Mood Changers}
Learn to alert the reader as soon as possible to any change in mood from the previous sentence. At least a dozen words will do this job for you: ``but,'' ``yet,'' ``however,'' ``nevertheless,'' ``still,'' ``instead,'' ``thus,'' ``therefore,'' ``meanwhile,'' ``now,'' ``later,'' ``today,'' ``subsequently,'' \& several more. I can't overstate how much easier it is for readers to process a sentence if you start with ``but'' when you're shifting direction. Or, conversely, how much harder it is if they must wait until the end to realize that you have shifted.

Many of us were taught that no sentence should begin with ``but.'' If that's what you learned, unlearn it -- there's no stronger word at the start. It announces total contrast with what has gone before, \& the reader is thereby primed for the change. If you need relief from too many sentences beginning with ``but,'' switch to ``however.'' It is, however, a weaker word \& needs careful placement. Don't start a sentence with ``however'' -- it hangs there like a wet dishrag. \& don't end with ``however'' -- by that time it has lost its howeverness. Put it as early as you reasonably can, as I did 3 sentences ago. Its abruptness then becomes a virtue.

``Yet'' does almost the same job as ``but,'' though its meaning is closer to ``nevertheless.'' Either of those words at the beginning of a sentence -- ``Yet he decided to go'' or ``Nevertheless he decided to go'' -- can replace a whole long phrase that summarizes what the reader has just been told: \textit{``Despite the fact that all these dangers had been pointed out to him}, he decided to go.'' Look for all the places where 1 of these short words will instantly convey the same meaning as a long \& dismal clause. ``Instead I took the train.'' ``Still I had to admire him.'' ``Thus I learned how to smoke.'' ``It was therefore easy to meet him.'' ``Meanwhile I had talked to John.'' What a vast amount of huffing \& puffing these pivotal words save! (The exclamation point is to show that I really mean it.)

As for ``meanwhile,'' ``now,'' ``today'', \& ``later,'' what they also save is confusion, for careless writers often change their time frame without remembering to tip the reader off. ``Now I know better.'' ``Today you can't find such an item.'' ``Later I found out why.'' Always make sure your readers are oriented. Always ask yourself where you left them in the previous sentence.

\subsection{Contractions}
Your style will be warmer \& truer to your personality if you use contractions like ``I'll'' \& ``won't'' \& ``can't'' when they fit comfortably into what you're writing. ``I'll be glad to see them if they don't get mad'' is less stiff than ``I will be glad to see them if they do not get mad.'' (Read that aloud \& hear how stilted it sounds.) There's no rule against such informality -- trust your ear \& your instincts. I only suggest avoiding 1 form -- ``I'd,'' ``he'd,'' ``we'd,'' etc. -- because ``I'd'' can mean both ``I had'' \& ``I would,'' \& readers can get well into a sentence before learning which meaning it is. Often it's not the one they thought it was. Also, don't invent contractions like ``could've.'' They cheapen your style. Stick with the ones you can find in the dictionary.

\subsection{That \& Which}
Anyone who tries to explain ``that'' \& ``which'' in $< 1$ hour is asking for trouble. Fowler, in his \textit{Modern English Usage}, takes 25 columns of type. I'm going for 2 minutes, perhaps the world record. Here (I hope) is much of what you need to bear in mind:


Always use ``that'' unless it makes your meaning ambiguous. Notice that in carefully edited magazines, such as \textit{The New Yorker}, ``that'' is by far the predominant usage. I mention this because it is still widely believed -- a residue from school \& college -- that ``which'' is more correct, more acceptable, more literary. It's not. In most situations, ``that'' is what you would naturally say \& therefore what you should write.

If your sentence needs a comma to achieve its precise meaning, it probably needs ``which.'' ``Which'' serves a particular identifying function, different from ``that.'' (A) ``Take the shoes that are in the closet.'' This means: take the shoes that are in the closet, not he ones under the bed. (B) ``Take the shoes, which are in the closet.'' Only 1 pair of shoes is under discussion; the ``which'' usage tells you where they are. Note that the common is necessary in B, but not in A.

A high proportion of ``which'' usages narrowly describe, or identify, or locate, or explain, or otherwise qualify the phrase that preceded the comma:
\begin{quote}
	The house, which has a red roof,\\The store, which is called Bob's Hardware,\\The Rhine, which is in Germany,\\The monsoon, which is a seasonal wind,\\The moon, which I saw from the porch,
\end{quote}
That's all I'm going to say that I think you initially need to know to write good nonfiction, which is a form that requires exact marshaling of information.

\subsection{Concept Nouns}
Nouns that express a concept are commonly used in bad writing instead of verbs that tell what somebody did. Here are 3 typical dead sentences:
\begin{quote}
	The common reaction is incredulous laughter.\\Bemused cynicism isn't the only response to the old system.\\The current campus hostility is a symptom of the change.
\end{quote}
What is so eerie about these sentences is that they have no people in them. They also have no working verbs -- only ``is'' or ``isn't.'' The reader can't visualize anybody performing some activity; all the meaning lies in impersonal nouns that embody a vague concept: ``reaction,'' ``cynicism,'' ``response,'' ``hostility.'' Turn these cold sentences around. Get people doing things:
\begin{quote}
	Most people just laugh with disbelief.\\Some people respond to the old system by turning cynical; others say $\ldots$\\It's easy to notice the change -- you can see how angry all the students are.
\end{quote}
My revised sentences aren't jumping with vigor, partly because the material I'm trying to knead into shape is shapeless dough. But at least they have real people \& real verbs. Don't get caught holding a bag full of abstract nouns. You'll sink to the bottom of the lake \& never be seen again.

\subsection{Creeping Nounism}
This is a new American disease that strings 2 or 3 nouns together where 1 noun -- or, better yet, 1 verb -- will do. Nobody goes broke now; we have money problem areas. It no longer rains; we have precipitation activity or a thunderstorm probability situation. Please, let it rain.

Today as many as 4 or 5 concepts nouns will attach themselves to each other, like a molecule chain. Here's a brilliant specimen I recently found: ``Communication facilitation skills development intervention.'' Not a person in sight, or a working verb. I think it's a program to help students write better.

\subsection{Overstatement}
``The living room looked as if an atomic bomb had gone off there,'' writes the novice writer, describing what he saw on Sunday morning after a party that got out of hand. Well, we all know he's exaggerating to make a droll point, but we also know that an atomic bomb \textit{didn't} go off there, or any other bomb except maybe a water bomb. ``I felt as if 10 747 jets were flying through my brain,'' he writes, ``\& I seriously considered jumping out the window \& killing myself.'' These verbal high jinks can get just so high -- \& this writer is already well over the limit -- before the reader feels an overpowering drowsiness. It's like being trapped with a man who can't stop reciting limericks. Don't overstate. You didn't really consider jumping out the window. Life has more than enough truly horrible funny situations. Let the humor sneak up so we hardly hear it coming.

\subsection{Credibility}
Credibility is just as fragile for a writer as for a President. Don't inflate an incident to make it more outlandish than it actually was. If the reader catches you in just 1 bogus statement that you are trying to pass off as true, everything you write thereafter will be suspect. It's too great a risk, \& not worth taking.

\subsection{Dictation}
Much of the ``writing'' done in America is done by dictation. Administrators, executives, mangers, educators, \& other officials think in terms of using their time efficiently. They think the quickest way of getting something ``written'' is to dictate it to a secretary \& never look at it. This is false economy -- they save a few hours \& blow their whole personality. Dictated sentences tend to be pompous, sloppy, \& redundant. Executives who are so busy that they can't avoid dictating should at least find time to edit what they have dictated, crossing words out \& putting words in, making sure that what they finally write is a true reflection of who they are, especially if it's a document that will go to customers who will judge their personality \& their company on the basis of their style.

\subsection{Writing Is Not A Contest}
Every writer is starting from a different point \& is bound for a different destination. Yet many writers are paralyzed by the thought that they are competing with everybody else who is trying to write \& presumably doing it better. This can often happen in a writing class. Inexperienced students are chilled to find themselves in the same class with students whose byline has appeared in the college newspaper. But writing for the college paper is no great credential; I've often found that the hares who write for the paper are overtaken by the tortoises who move studiously toward the goal of mastering the craft. The same fear hobbles freelance writers, who see the work of other writers appearing in magazines while their own keeps returning in the mail. Forget the competition \& go at your own pace. Your only contest is with yourself.

\subsection{The Subconscious Mind}
Your subconscious mind does more writing than you think. Often you'll spend a whole day trying to fight your way out of some verbal thicket in which you seem to be tangled beyond salvation. Frequently a solution will occur to you the next morning when you plunge back in. While you slept, your writer's mind didn't. A writer is always working. Stay alert to the currents around you. Much of what you see \& hear will come back, having percolated for days or months or even years through your subconscious mind, just when your conscious mind, laboring to write, needs it.

\subsection{The Quickest Fix}
Surprisingly often a difficult problem in a sentence can be solved by simply getting rid of it. Unfortunately, this solution is usually the last one that occurs to writers in a jam. 1st they will put the troublesome phrase through all kinds of exertions -- moving it to some other part of the sentence, trying to rephrase it, adding new words to clarify the thought or to oil whatever is stuck. These efforts only make the situation worse, \& the writer is left to conclude that there \textit{is} no solution to the problem -- not a comforting thought. When you find yourself at such an impasse, look at the troublesome element \& ask, ``Do I need it at all?'' Probably you don't. It was trying to do an unnecessary job all along -- that's why it was giving you so much grief. Remove it \& watch the afflicted sentence spring to life \& breathe normally. It's the quickest cure \& often the best.

\subsection{Paragraphs}
Keep your paragraphs short. Writing is visual -- it catches the eye before it has a chance to catch the brain. Short paragraphs put air around what you write \& make it look inviting, whereas a long chunk of type can discourage a reader from even starting to read.

Newspaper paragraphs should be only 2 or 3 sentences long; newspaper type is set in a narrow width, \& the inches quickly add up. You may think such frequent paragraphing will damage the development of your point. Obviously \textit{The New Yorker} is obsessed by this fear -- a reader can go for miles without relief. Don't worry; the gains far outweigh the hazards.

But don't go berserk. A succession of tiny paragraphs is as annoying as a paragraph that's too long. I'm thinking of all those midget paragraphs -- verbless wonders -- written by modern journalists to make their articles quick `n' easy. Actually they make the reader's job harder by chopping up a natural train of thought. Compare the following 2 arrangements of the same article -- how they look at a glance \& how they read:
\begin{quotation}
	The No. 2 lawyer at the White House left work early on Tuesday, drove to an isolated park overlooking the Potomac River \& took his life.
	
	A revolver in his hand, slumped against a Civil War -- era cannon, he left behind no note, no explanation.
	
	Only friends, family \& colleagues in stunned sorrow.
	
	\& a life story that until Tuesday had read like any man's fantasy.
	
	The No. 2 lawyer at the White House left work early on Tuesday, drove to an isolated park overlooking the Potomac River \& took his life. A revolver in his hand, slumped against a Civil War -- era cannon, he left behind no note, no explanation. Only friends, family \& colleagues in stunned sorrow. \& a life story that until Tuesday had read like any man's fantasy.
\end{quotation}
The Associated Press version (left), with its breezy paragraphing \& verbless 3rd \& 4th sentences, is disruptive \& condescending. ``Yoo-hoo! Look how simple I'm making this for you!'' the reporter is calling to us. My version (right) gives the reporter the dignity of writing good English \& building 3 sentences into a logical unit.

Paragraphing is a subtle but important element in writing nonfiction articles \& books -- a road map constantly telling your reader how you have organized your ideas. Study good nonfiction writers to see how they do it. You'll find that almost all of them think in paragraph units, not in sentence units. Each paragraph has its own integrity of content \& structure.

\subsection{Sexism}
1 of the most vexing new questions for writers is what to do about sexist language, especially the ``he-she'' pronoun. The feminist movement helpfully revealed how much sexism lurks in our language, not only in the offensive ``he'' but in the hundreds of words that carry an invidious meaning or some overtone of judgment. They are words that patronize (``gal''), or that imply 2nd-class status (``poetess''), or a 2nd-woman to do a certain kind of job (``lady lawyer''), or that are deliberately prurient (``divorce\'e,'' ``coed,'' ``blonde'') \& are seldom applied to men. Men get mugged; a woman who gets mugged is a shapely stewardess or a pert brunette.

More damaging -- \& more subtle -- are all the usages that treat woman as possessions of the family made, not as people with their own identity who played an equal part in the family saga: ``Early settlers pushed west with their wives \& children.'' Turn those settlers into pioneer families, or pioneer couples who went est with their sons \& daughters, or men \& women who settled the West. Today there are very few roles  that aren't open to both sexes. Don't use constructions that suggest that only men can be settlers or farmers or cops or firefighters.

A thornier problem is raised by the feminists' annoyance with words that contain ``man,'' such as ``chairman'' \& ``spokesman.'' Their point is that women can chair a committee as well as a man \& are equally good at spoking. Hence the flurry of new words like ``chairperson'' \& ``spokeswoman.'' Those makeshift words from the 1960s raised our consciousness about sex discrimination, both in words \& in attitudes. But in the end they are makeshift words, sometimes hurting the cause more than helping it. 1 solution is to find another term: ``chair'' for ``chairman,'' ``company representative'' for ``spokesman.'' You can also convert the noun into a verb: ``Speaking for the company, Ms. Jones said $\ldots$'' Where a certain occupation has both a masculine \& a feminine form, look for a generic substitute. Actors \& actresses can become performers.

This still leaves the bothersome pronoun. ``He'' \& ``him'' \& ``his'' are words that rankle. ``Everyone employee should decide what he thinks is best for him \& his dependents.'' What are we to do about these countless sentences? 1 solution is to turn them into the plural: ``All employees should decide what they think is best for them \& their dependents.'' But this is good only in small doses. A style that converts every ``he'' into a ``they'' will quickly turn to mush.

Another common solution is to use ``or'': Every employee should decide what he or she thinks is best for him or her.'' But again, it should be used sparingly. Often a writer will find several situations in an article where he or she can use ``he or she,'' or ``him or her,'' if it seems natural. By ``natural'' I mean that the writer is serving notice that he (or she) has the problem in mind \& is trying his (or her) best within reasonable limits. But let's face it: the English language is stuck with the generic masculine (``Man shall not live by bread alone''). To turn every ``he'' into a ``he or she,'' \& every ``his'' into a ``his or her,'' would clog the language.

In early editions of \textit{On Writing Well} I used ``he'' \& ``him'' to refer to ``the reader,'' ``the writer,'' ``the critic,'' ``the humorist,'' etc. I felt that the book would be harder to read if I used ``he or she'' with every such mention. (I reject ``he{\tt/}she'' altogether; the slant has no place in good English.) Over the years, however, many women wrote to nudge me about this. They said that as writers \& readers themselves they resent always having to visualize a man doing the writing \& reading, \& they're right; I stand nudged. Most of the nudgers urged me to adopt the plural: to use ``readers'' \& ``writers,'' followed thereafter by ``they.'' I don't like plurals; they weaken writing because they are less specific than the singular, less easy to visualize. I'd like every writer to visualize \textit{1} reader struggling to read what he or she has written. Nevertheless I found 300 or 400 places where I could eliminate ``he,'' ``him,'' ``his,'' ``himself'' or ``man,'' mainly by switching to the plural, with no harm done; the sky didn't fall in. Where the male pronoun remains in this edition I felt it was the only solution that wasn't cumbersome.

The best solutions simply eliminate ``he'' \& its connotations of male ownership by using other pronouns or by altering some other component of the sentence. ``We'' is a handy replacement for ``he.'' ``Our'' \& ``the'' can often replace ``his.'' (A) ``1st \textit{he} notices what's happening to \textit{his} kids \& he blames it on \textit{his} neighborhood.'' (B) ``1st \textit{we} notice what's happening to \textit{our} kids \& we blame it on \textit{the} neighborhood.'' General nouns can replace specific nouns. (A) ``Doctors often neglect their wives \& children.'' (B) ``Doctors often neglect their families.'' Countless sins can be erased by such small changes.

1 other pronoun that helped me in my repairs was ``you.'' Instead of talking about what ``the writer'' does \& the trouble \textit{he} gets into, I found more places where I could address the writer directly (``You'll often find $\ldots$''). It doesn't work for every kind of writing, but it's a godsend to anyone writing an instructional book or a self-help book. The voice of a Dr. Spock talking to the mother of a child with a fever, or the voice of a Julia Child talking to the cook stalled in mid-recipe, is 1 of the most reassuring sounds a reader can hear. Always look for ways to make yourself available to the people you're trying to reach.

\subsection{Rewriting}
Rewriting is the essence of writing well: it's where the game is won or lost. That idea is hard to accept. We all have an emotional equity in our 1st draft; we can't believe that it wasn't born perfect. But the odds are close to 100\% that it wasn't. Most writers don't initially say what they want to say, or say it as well as they could. The newly hatched sentence almost always has something wrong with it. It's not clear. It's not logical. It's verbose. It's klunky. It's pretentious. It's boring. It's full of clutter. It's full of clich\'es. It lacks rhythm. It can be read in several different ways. It doesn't lead out of the previous sentence. It doesn't $\ldots$ The point is that clear writing is the result of a lot of tinkering.

Many people assume that professional writers don't need to rewrite; the words just fall into place. On the contrary, careful writers can't stop fiddling. I've never thought of rewriting as an unfair burden; I'm grateful for every chance to keep improving my work. Writing is like a good watch -- it should run smoothly \& have no extra parts. Students don't share my love of rewriting. They think of it as punishment: extra homework or extra infield practice. Please -- if you're such a student -- think of it as a gift. You won't write well until you understand that writing is an evolving \textit{process}, not a finished \textit{product}. Nobody expects you to get it right the 1st time, or even the 2nd time.

What do I mean by ``rewriting''? I don't mean writing 1 draft \& then writing a different 2nd version, \& then a 3rd. Most rewriting consists of reshaping \& tightening \& refining the raw material you wrote on your 1st try. Much of it consists of making sure you've given the reader a narrative flow he can follow with no trouble from beginning to end. Keep putting yourself in the reader's place. Is there something he should have been told early in the sentence that you put near the end? Does he know when he starts sentence B that you've made a shift -- of subject, tense, tone, emphasis -- from sentence A?

Let's look at a typical paragraph \& imagine that it's the writer's 1st draft. There's nothing really wrong with it; it's clear \& it's grammatical. But it's full of ragged edges: failures of the writer to keep the reader notified of changes in time, place, \& mood, or to vary \& animate the style. What I've done is to add, in bracketed italics after each sentence, some of the thoughts that might occur to an editor taking a 1st look at this draft. After that you'll find my revised paragraph, which incorporates those corrective thoughts.
\begin{quotation}
	There used to be a time when neighbors took care of one another, he remembered. [\textit{Put ``he remembered'' 1st to establish reflective tone}.] It no longer seemed to happen that way, however. [\textit{The contrast supplied by ``however'' must come 1st. Start with ``But.'' Also establish America locale}.] He wondered if it was because everyone in the modern world was so busy. [\textit{All these sentences are the same length \& have the same soporific rhythm; turn this one into a question?}] It occurred to him that people today have so many things to do that they don't have time for old-fashioned friendship. [\textit{Sentence essentially repeats previous sentence; kill it or warm it up with specific detail}.] Things didn't work that way in America in previous eras. [\textit{Reader is still in the present; reverse the sentence to tell him he's now in the past. ``America'' no longer needed if inserted earlier.}] \& he knew that the situation was very different in other countries, as he recalled from the years when he lived in villages in Spain \& Italy. [\textit{Reader is still in America. Use a negative transition word to get him to Europe. Sentence is also too flabby. Break it into 2 sentences?}] It almost seemed to him that as people got richer \& built their houses farther apart they isolated themselves from the essentials of life. [\textit{Irony deferred too long. Plant irony early. Sharpen the paradox about richness}.] \& there was another thought that troubled him. [\textit{This is the real point of the paragraph; signal the reader that it's important. Avoid weak ``there was'' construction}.] His friends had deserted him when he needed them most during his recent illness. [\textit{Reshape to end with ``most''; the last word is the one that stays in the reader's ear \& gives the sentence its punch. Hold sickness for next sentence; it's a separate thought}.] It was almost as if they found him guilty of doing something shameful. [\textit{Introduce sickness here as the reason for the same. Omit ``guilty''; it's implicit}.] He recalled reading somewhere about societies in primitive parts of the world in which sick people were shunned, though he had never heard of any such ritual in America. [\textit{Sentence starts slowly \& stays sluggish \& dull. Break it into shorter units. Snap off the ironic point}.]
	
	He remembered that neighbors used to take care of one another. But that no longer seemed to happen in America. Was it because everyone was so busy? Were people really so preoccupied with their television sets \& their cars \& their fitness programs that they had no time for friendship? In previous eras that was never true. Nor was it how families lived in other parts of the world. Even in the poorest villages of Spain \& Italy, he recalled, people would drop in with a loaf of bread. An ironic idea struck him: as people got richer they cut themselves off from the richness of life. But what really troubled him was an even more shocking fact. The time when his friends deserted him was the time when he needed them most. By getting sick he almost seemed to have done something shameful. He knew that other societies had a custom of ``shunning'' people who were very ill. But that ritual only existed in primitive cultures. Or did it?
\end{quotation}
My revisions aren't the best ones that could be made, or the only ones. They're mainly matters of carpentry: altering the sequence, tightening the flow, sharpening the point. Much could still be done in such areas as cadence, detail, \& freshness of language. The total construction is equally important. Read your article aloud from beginning to end, always remembering where you left the reader in the previous sentence. You might find you had written 2 sentences like this:
\begin{quotation}
	The tragic hero of the play is Othello. Small \& malevolent, Iago feeds his jealous suspicions.
\end{quotation}
In itself there's nothing wrong with the Iago sentence. But as a sequel to the previous sentence it's very wrong. The name lingering in the reader's ear is Othello; the reader naturally assumes that Othello is small \& malevolent.

When you read your writing aloud with these connecting links in mind you'll hear a dismaying number of places where you lost the reader, or confused the reader, or failed to tell him the 1 fact he needed to know, or told him the same thing twice: the inevitable loose ends of every early draft. What you must do is make an arrangement -- one that holds together from start to finish \& that moves with economy \& warmth.

Learn to enjoy this tidying process. I don't like to write; I like to have written. But I love to rewrite. I especially like to cut: to press the \textsc{delete} key \& see an unnecessary word or phrase or sentence vanish into the electricity. I like to replace a humdrum word with one that has more precision or color. I like to strengthen the transition between 1 sentence \& another. I like to rephrase a drab sentence to give it a more pleasing rhythm or a more graceful musical line. With every small refinement I feel that I'm coming nearer to where I would like to arrive, \& when I finally get there I know it was the rewriting, not the writing, that won the game.

\subsection{Wring on A Computer}
The computer is God's gift, or technology's gift, to rewriting \& reorganizing. It puts your words right in front of your eyes for your instant consideration -- \& reconsideration; you can play with your sentences until you get them right. The paragraphs \& pages will keep rearranging themselves, no matter how much you cut \& change, \& then your printer will type everything neatly while you go \& have a beer. Sweeter music could hardly be sung to writers than the sound of their article being retyped with all its improvements -- but not by them.

It's no longer necessary for this book to explain, as earlier editions did, how to operate the wonderful new machine called a word processor that had come into our lives \& how to put its wonders to use in writing, rewriting, \& organizing. That's now common knowledge. I'll just remind you (if you're still not a believer) that the savings in time \& drudgery are enormous. With a computer I sit down to write more willingly than I did when I used a typewriter, especially if I'm facing a complex task of organization, \& I finish the task sooner \& with far less fatigue. These are crucial gains for a writer: time, output, energy, enjoyment, \& control.

\subsection{Trust Your Material}
The longer I work at the craft of writing, the more I realize that there's nothing more interesting than the truth. What people do -- \& what people say -- continues to take me by surprise me with its wonderfulness, or its quirkiness, or its drama, or its humor, or its pain. Who could invent all the astonishing things that really happen? I increasingly find myself saying to writers \& students, ``Trust your material.'' It seems to be hard advice to follow.

Recently I spent some time as a writing coach at a newspaper in a small American city. I noticed that many reporters had fallen into the habit of trying to make the news more palatable by writing in a feature style. Their leads consisted of a series of snippets that went something like this:
\begin{quotation}
	Whoosh!
	
	It was incredible.
	
	Ed Barnes wondered if he was seeing things.
	
	Or maybe it was just spring fever. Funny how April can do that to a guy.
	
	It wasn't as if he hadn't checked his car before leaving the house.
	
	But then again, he didn't remember to tell Linda.
	
	Which was odd, because he always remembered to tell Linda. Ever since they started going together back in junior high.
	
	Was that really 20 years ago?
	
	\& now there was also little Scooter to worry about.
	
	Come to think of it, the dog was acting kind of suspicious.
\end{quotation}
The articles often began on page 1, \& I would read as far as ``Continued on page 9'' \& still have no idea of what they were about. Then I would dutifully turn to page 9 \& find myself in an interesting story, full of specific details. I'd say to the reporter, ``That was a good story when I finally got over here to page 9. Why didn't you put that stuff in the lead?'' The reporter would say, ``Well, in the lead I was writing color.'' The assumption is that fact \& color are 2 separate ingredients. They're not; color is organic to the fact. Your job is to present  the colorful fact.

In 1988 I wrote a baseball book called \textit{Spring Training}. It combined my lifelong vocation with my lifelong addiction -- which is 1 of the best things that can happen to a writer; people will write better \& with more enjoyment if they write about what they care about. I chose spring training as my small corner of the large subject of baseball because it's a time of renewal, both for the players \& for the fans. The game is given back to us in its original purity: it's played outside, in the sun, on grass, without organ music, by young men who are almost near enough to touch \& whose salaries \& grievances are mercifully put aside for 6 weeks. Above all, it's a time to teaching \& learning. I chose the Pittsburgh Pirates as the team I would cover because they trained in an old-time ballpark in Bradenton, Florida, \& were a young club just starting to rebuild, with a manager, Jim Leyland, who was committed to teaching.

I didn't want to romanticize the game. I don't like baseball movies that go into slow motion when the batter hits a home run, to notify me that it's a pregnant moment. I \textit{know} that about home runs, especially if they're hit with 2 out in the bottom of the 9th to win the game. I resolved not to let my writing go into slow motion -- not to nudge the reader with significance -- or to claim baseball as a metaphor for life, death, middle age, lost youth or a more innocent America. My premise was that baseball is a job -- honorable work -- \& I wanted to know how that job gets taught \& learned.

So I went to Jim Leyland \& his coaches \& I said, ``You're a teacher. I'm a teacher. Tell me: How do you teach hitting? How do you teach pitching? How do you teach fielding? How do you teach baserunning? How do you keep these young men \textit{up} for such a brutally long schedule?'' All of them responded generously \& told me in detail how they do what they do. So did the players \& all the other men \& women who had information I wanted: umpires, scouts, ticket sellers, local boosters.

1 day I climbed up into the stands behind home plate to look for a scout. Spring training is baseball's ultimate talent show, \& the camps are infested with laconic men who have spent a lifetime appraising talent. I spotted an empty seat next to a weathered man in his 60s who was using a stopwatch \& taking notes. When the inning was over I asked him what he was timing. He said he was Nick Kamzic, Northern Scouting Coordinator of the California Angels, \& he was timing runners on the base paths. I asked him what kind of information he was looking for.
\begin{quotation}
	``Well, it takes a right-handed batter 4.3 seconds to reach 1st base,'' he said, ``\& a left-handed batter 4.1 or 5.2 seconds. Naturally that varies a little -- you've got to take the human element into consideration.''

	``What do those numbers tell you?'' I asked.

	``Well, of course the average double play takes 4.3 seconds,'' he said. He said it as if it was common knowledge. I had never given any thought to the elapsed time of a double play.

	``So that means $\ldots$''

	``If you see a player who gets to 1st base in $< 4.3$ seconds you're interested in him.''
\end{quotation}
As a fact that's self-sufficient. There's no need to add a sentence pointing out that 4.3 seconds is remarkably little time to execute a play that involves 1 batted ball, 2 thrown balls, \& 3 infielders. Given 4.3 seconds, readers can do their own marveling. They will also enjoy being allowed to think for themselves. The reader plays a major role in the act of writing \& must be given room to play it. Don't annoy your readers by over-explaining -- by telling them something they already know or can figure out. Try not to use words like ``surprisingly,'' ``predictably'', \& ``of course,'' which put a value on a fact before the reader encounters the fact. Trust your material.

\subsection{Go With Your Interests}
There's no subject you don't have permission to write about. Students often avoid subjects close to their heart -- skateboarding, cheerleading, rock music, cars -- because they assume that their teachers will regard those topics as ``stupid.'' No area of life is stupid to someone who takes it seriously. If you follow your affections you will write well \& will engage your readers.

I've read elegant books on fishing \& poker, billiards \& rodeos, mountain climbing \& giant sea  turtles \& many other subjects I didn't think I was interested in. Write about your hobbies: cooking, gardening, photography, knitting, antiques, jogging, sailing, scuba diving, tropical birds, tropical fish. Write about your work: teaching, nursing, running a business, running a store. Write about a field you enjoyed in college \& always meant to get back to: history, biography, art, archaeology. No subject is too specialized or too quirky if you make an honest connection with it when you write about it.'' -- \cite[pp. 69--89]{Zinsser2016}

%------------------------------------------------------------------------------%

\begin{center}\huge
	Part III: Forms
\end{center}

%------------------------------------------------------------------------------%

\section{Nonfiction as Literature}
``1 weekend a few years ago I went to Buffalo to talk at a writers' conference that had been organized by a group of women writers in that city. The women were serious about their craft, \& the books \& articles they had written were solid \& useful. They asked me if I would take part in a radio talk show earlier in the week to publicize the conference -- they would be with the host in the studio \& I would be on a telephone hookup from my apartment in New York.

The appointed evening arrived, \& my phone rang, \& the host came on \& greeted me with the strenuous joviality of his trade. He said he had 3 lovely ladies in the studio with him \& he was eager to find out what we all thought of the present state of literature \& what advice we had for all his listeners who were members of the literati \& had literary ambitions themselves. This hearty introduction dropped like a stone in our midst, \& none of the 3 lovely ladies said anything, which I thought was the proper response.

The silence lengthened, \& finally I said, ``I think we should banish all further mention of the words `literature' \& `literary' \& `literati.''' I knew that the host had been briefed about what kind of writers we were \& what we wanted to discuss. But he had no other frame of reference. ``Tell me,'' he said, ``what insights do you all have about the literary experience in America today?'' Silence also greeted this question. Finally I said, ``We're here to talk about the craft of writing.''

He didn't know what to make of that, \& he began to invoke the names of authors like Ernest Hemingway \& Saul Bellow \& William Styron, whom we surely regarded as literary giants. We said those writers didn't happen to be our models, \& we mentioned people like Lewis Thomas \& Joan Didion \& Gary Wills. He had never heard of them. 1 of the women mentioned Tom Wolfe's \textit{The Right Stuff}, \& he hadn't heard of that. We explained that these were writers we admired for their ability to harness the issues \& concerns of the day.

``But don't you want to write anything literary?'' our host said. The 3 women said they felt they were already doing satisfying work. That brought the program to another halt, \& the host began to accept phone calls from his listeners, all of whom were interested in the craft of writing \& wanted to know how we went about it. ``\& yet, in the stillness of the night,'' the host said to several callers, ``don't you ever dream of writing the great American novel?'' They didn't. They had no such dreams -- in the stillness of the night or at any other time. It was 1 of the all-time lousy radio talk shows.

The story sums up a situation that any practitioner of nonfiction will recognize. Those of us who are trying to write well about the world we live in, or to teach students to write well about the world \textit{they} live in, are caught in a time warp, where literature by definition still consists of forms that were certified as ``literary'' in the 19th century: novels \& short stories \& poems. But the great preponderance of what writers now write \& shell, what book \& magazine publishers publish \& what readers demand is nonfiction.

The shift can be documented by all kinds of examples. One is the history of the Book-of-the-Month Club. When the club was founded in 1926 by Harry Scherman, Americans had little access to good new literature \& were mainly reading junk like \textit{Ben-Hur}. Scherman's idea was that any town that had a post office had the equivalent of a bookstore, \& he began sending the best new books to his newly recruited readers all over the country.

Much of what he sent was fiction. The list of main selections chosen by the club from 1926 through 1941 is heavily laced with novelists: Ellen Glasgow, Sinclair Lewis, Virginia Woolf, John Galsworthy, Elinor Wylie, Ignazio Silone, Rosamond Lehmann, Edith Wharton, Somerset Maugham, Willa Cather, Booth Tarkington, Isak Dinesen, James Gould Cozzens, Thornton Wilder, Sigrid Undset, Ernest Hemingway, William Saroyan, John P. Marquand, John Steinbeck, \& many others. That was the high tide of ``literature'' in America. Members of the Book-of-the-Month Club hardly heard the approach of World War II. Not until 1940 was it brought home to them in a book, \textit{Mrs. Miniver}, a stiff-upper-lip novel about the early days of the Battle of Britain.

All of this changed with Pearl Harbor. World War II sent 7 million Americans overseas \& opened their eyes to reality: to new places \& issues \& events. After the war that trend was reinforced by the advent of television. People who saw reality every evening in their living room lost patience with the slower rhythms \& glancing allusions of the novelist. Overnight, America became a fact-minded nation. After 1946 the Book-of-the-Month Club's members predominantly demanded -- \& therefore received -- nonfiction.

Magazines were swept along on the same tide. The \textit{Saturday Evening Post}, which had long spoon-fed its readers a heavy diet of short stories by writers who all seemed to have 3 names -- Clarence Budington Kelland, Octavus Roy Cohen -- reversed the ratio in the early 1960s. 90\% of the magazine was now allotted to nonfiction articles, with just 1 short story by a 3-named author to keep the faithful from feeling abandoned. It was the beginning of a golden era of nonfiction, especially in \textit{Life}, which ran finely crafted articles every week; in \textit{The New Yorker}, which elevated the form by originating such landmarks of modern American writing as Rachel Carson's \textit{Silent Spring} \& Truman Capote's \textit{In Cold Blood}; \& in \textit{Harper's}, which commissioned such remarkable pieces as Norman Mailer's \textit{Armies of the Night}. Nonfiction became the new American literature.

Today there's no area of life -- present of past -- that isn't being made accessible to ordinary readers by men \& women writing with high seriousness \& grace. Add to this literature of fact all the disciplines that were once regarded as academic, like anthropology \& economics \& social history, that have become the domain of nonfiction writers \& of broadly curious readers. Add all the books combining history \& biography that have distinguished American letters in recent years: David McCullough's \textit{Truman} \& \textit{The Path Between the Seas}; Robert A. Caro's \textit{The Power Broker: Robert Moses \& the Fall of New York}; Taylor Branch's \textit{Parting the Waters: America in the King Years, 1954--63}; Richard Kluger's \textit{The Paper: The Life \& Death of the New York Herald Tribune}; Richard Rhodes's \textit{The Making of the Atomic Bomb}; Thomas L. Friedman's \textit{From Beirut to Jerusalem}; J. Anthony Lukas's \textit{Common Ground: A Turbulent Decade in the Lives of American Families}; Edmund Morris's \textit{Theodore Rex}; Nicholas Lemann's \textit{The Promised Land: The Great Black Migration \& How It Changed America}; Adam Hochschild's \textit{King Leopold's Ghost: A Story of Greed, Terror \& Heroism in Colonial Africa}; Ronald Steel's \textit{Walter Lippmann \& the American Century}; Marion Elizabeth Rodgers's \textit{Mencken: The American Iconoclast}; David Remnick's \textit{Lenin's Tomb: The Last Days of the Soviet Empire}; Andrew Delbanco's \textit{Melville}; Mark Stevens's \& Annalyn Swan's \textit{de Kooning: An American Master}. My roster of the new literature of nonfiction, in short, would include all the writers who come bearing information \& who present it with vigor, clarity \& humanity.

I'm not saying that fiction is dead. Obviously the novelist can take us into places where no other writer can go: into the deep emotions \& the interior life. What I'm saying is that I have no patience with the snobbery that says nonfiction is only journalism by another name \& that journalism by any name is a dirty word. While we're redefining literature, let's also redefine journalism. Journalism is writing that 1st appears in any periodic journal, whatever its constituency. Lewis Thomas's 1st 2 books, \textit{Lives of a Cell} \& \textit{The Medusa \& the Snail}, were 1st written as essays for the \textit{New England Journal of Medicine}. Historically, in America, good journalism becomes good literature. H. L. Mencken, Ring Lardner, Joseph Mitchell, Edmund Wilson \& dozens of other major American writers were working journalists before they were canonized in the church of literature. They just did what they did best \& never worried about how it was defined.

Ultimately every writer must follow the path that feels most comfortable. For most people learning to write, that path is nonfiction. It enables them to write about what they know or can observe or can find out. This is especially true of young people \& students. They will write far more willingly about subjects that touch their own lives or that they have an aptitude for. Motivation is at the heart of writing. If nonfiction is where you do your best writing, or your best teaching of writing, don't be buffaloed into the idea that it's an inferior species. The only important distinction is between good writing \& bad writing. Good writing is good writing, whatever form it takes \& whatever we call it.'' -- \cite[pp. 93--96]{Zinsser2016}

%------------------------------------------------------------------------------%

\section{Writing About People: The Interview}
``Get people talking. Learn to ask questions that will elicit answers about what is most interesting or vivid in their lives. Nothing so animates writing as someone telling what he thinks or what he does -- in his own words.

His own words will always be better than your words, even if you are the most elegant stylist in the land. They carry the inflection of his speaking voice \& the idiosyncrasies of how he puts a sentence together. They contain the regionalisms of his conversation \& the lingo of his trade. They convey his enthusiasms. This is a person talking to the reader directly, not through the filter of a writer. As soon as a writer steps in, everyone else's experience becomes secondhand.

Therefor learn how to conduct an interview. Whatever form of nonfiction you write, it will come alive in proportion to the number of ``quotes'' you can weave into it as you go along. Often you'll find yourself embarking on an article so apparently lifeless -- the history of an institution, or some local issue such as storm sewers -- that you will quail at the prospect of keeping your readers, or even yourself, awake.

Take heart. You'll find the solution if you look for the human element. Somewhere in every drab institution are men \& women who have a fierce attachment to what they are doing \& are rich repositories of lore. Somewhere behind every storm sewer is a politician whose future hangs on getting it installed \& a widow who has always lived on the block \& is outraged that some damn-fool legislator thinks it will wash away. Find these people to tell your story \& it won't be drab.

I've proved this to myself often. Many years ago I was invited to write a small book for the New York Public Literary to celebrate the 50th anniversary of its main building on 5th Avenue. On the surface it seemed to be just the story of a marble building \& millions of musty volumes. But behind the facade I found that the library had 19 research divisions, each with a curator supervising a hoard of treasures \& oddities, from Washington's handwritten Farewell Address to 750,000 movie stills. I decided to interview all those curators to learn what was in their collections, what they were adding to keep up with new areas of knowledge, \& how their rooms were being used.

I found that the Science \& Technology division had a collection of patents 2nd only to that of the United States Patent Office \& was therefore a 2nd home to the city's patent lawyers. But it also had a daily stream of men \& women who thought they were on the verge of discovering perpetual motion. ``Everybody's got something to invent,'' the curator explained, ``but they won't tell us what they're looking for -- maybe because they think we'll patent it ourselves.'' The whole building turned out to be just such a mixture of scholars \& searchers \& crackpots, \& my story, though ostensibly the chronicle of an institution, was really a story about people.

I used the same approach in a long article about Sotheby's, the London auction film. Sotheby's was also divided into various domains, such as silver \& porcelain \& art, each with an expert in charge, \&, like the Library, it subsisted on the whims of a capricious public. The experts were like department heads in a small college, \& all of them had anecdotes that were unique both in substance \& in the manner of telling:
\begin{quotation}
	``We just sit here like Micawber waiting for things to come in,'' said R. S. Timewell, head of the furniture department. ``Recently an old lady near Cambridge wrote that she wanted to raise 2000 pounds \& asked if I would go through her house \& see if her furniture would fetch that much. I did, \& there was absolutely nothing of value. As I was about to leave I said, `Have I seen \textit{every}thing?' She said I had, except for a maid's room that she hadn't bothered to show me. The room had a very fine 18th-century chest that the old lady was using to store blankets in. `Your worries are over,' I told her, `if you sell that chest.' She said, `But that's quite impossible -- where will I store my blankets?'''
\end{quotation}
My worries were over, too. By listening to the quizzical scholars who ran the business \& to the men \& women who flocked there every morning bearing unloved objects found in British attics (``I'm afraid it \textit{isn't} Queen Anne, madam -- much nearer Queen Victoria, unfortunately''), I got as much human detail as a writer could want.

Again, when I was asked in 1966 to write a history of the Book-of-the-Month Club to mark its 40th birthday, I thought I might encounter nothing but inert matter. But I found a peppery human element on both sides of the fence, for the books had always been selected by a panel of strong-minded judges \& sent to equally stubborn subscribers, who never hesitated to wrap up a book they didn't like \& send it right back. I was given more than 1000 pages of transcribed interviews with the 5 original judges (Heywood Broun, Henry Seidel Canby, Dorothy Canfield, Christopher Morley, \& William Allen White), to which I added my own interviews with the club's founder, Harry Scherman, \& with the judges who were then active. The result was 4 decades' worth of personal memories on how America's reading tastes had changed, \& even the books took on a life of their own \& became characters in my story:
\begin{quotation}
	``Probably it's difficult for anyone who remembers the prodigious success of \textit{Gone With the Wind},'' Dorothy Canfield said, ``to think how it would have seemed to people who encountered it simply as a very, very long \& detailed book about the Civil War \& its aftermath. We had never heard of the author \& didn't have anybody else's opinion on it. It was chosen with a little difficulty, because some of the characterization was not very authentic or convincing. But as a narrative it had the quality which the French call \textit{attention}: it made you want to turn over the page to see what happens next. I remember that someone commented, `Well, people may not like it very much, but nobody can deny that it gives a lot of reading for your money.' Its tremendous success was, I must say, about as surprising to us as to anybody else.''
\end{quotation}
Those 3 examples are typical of the kind of information that is locked inside people's heads, which a good nonfiction writer must unlock. The bests way to practice is to go out \& interview people. The interview itself is 1 of the most popular nonfiction forms, so you should master it early.

How should you start? 1st, decide what person you want to interview. If you are a college student, don't interview your roommate. With all due respect for what terrific roommates you've got, they probably don't have much to say that the rest of us want to hear. To learn the craft of nonfiction you must push yourself out into the real world -- your town or your city or your country -- \& pretend that you're writing for a real publication. If it helps, decide which publication you are hypthetically writing for. Choose as your subject someone whose job is so important, or so interesting, or so unusual that the average reader would want to read about that person.

That doesn't mean he or she to be president of the bank. It can be the owner of the local pizza parlor or supermarket or hairdressing academy. It can be the fisherman who puts out to sea every morning, or the Little League manager, or the nurse. It can be the butcher, the baker or -- better yet, if you can find him -- the candlestick maker. Look for the women in your community who are unraveling the old myths about what the 2 sexes were foreordained to do. Choose, in short, someone who touches some corner of the reader's life.

Interviewing is 1 of those skills you can only get better at. You will never again feel so ill at ease as when you try it for the 1st time, \& probably you'll never feel entirely comfortable prodding another person for answers he or she may be too shy or too inarticulate to reveal. But much of the skill is mechanical. The rest is instinct -- knowing how to make the other person relax, when to push, when to listen, when to stop. This can all be learned with experience.

The basic tools for an interview are paper \& some well-sharpened pencils. Is that insultingly obvious advice? You'd be surprised how many writers venture forth to stalk their quarry with no pencil, or with one that breaks, or with a pen that doesn't work, \& with nothing to write on. ``Be prepared'' is as apt a motto for the nonfiction writer on his rounds as it is for the Boy Scout.

But keep your notebook out of sight until you need it. There's nothing less likely to relax a person than the arrival of a stranger with a stenographer's pad. Both of you need time to get to know each other. Take a while just to chat, gauging what sort of person you're dealing with, getting him or her to trust you.

Never go into an interview without doing whatever homework you can. If you are interviewing a town official, know his or her voting record. If it's an actress, know what plays or movies she has been in. You will be resented if you inquire about facts you could have learned in advance.

Make a list of likely questions -- it will save you the vast embarrassment of going dry in mid-interview. Perhaps you won't need the list; better questions will occur to you, or the people being interviewed will veer off at an angle you couldn't have foreseen. Here you can only go by intuition. If they stray hopelessly off the subject, drag them back. If you like the new direction, follow along \& forget the questions you intended to ask.

Many beginning interviews are inhibited by the fear that they are imposing on other people \& have no right to invade their privacy. This fear is almost wholly unfounded. The so-called man in the street is delighted that somebody wants to interview him. Most men \& women lead lives, if not of quiet desperation, at least of desperate quietness, \& they jump at a chance to talk about their work to an outsider who seems eager to listen.

This doesn't necessarily mean it will go well. Often you will be talking to people who have never been interviewed before, \& they will warm to the process awkwardly, self-consciously, perhaps not giving you anything you can use. Come back another day; it will go better. You will both even begin to enjoy it -- proof that you aren't forcing your victims to do something they really don't want to do.

Speaking of tools, is it all right (you ask) to use a tape recorder? Why not just take one along, start it going, \& forget all that business of pencil \& paper?

Obviously the tape recorder is a superb machine for capturing what people have to say -- especially people who, for reasons of their culture of temperament, would never get around to writing it down. In such areas as social history \& anthropology it's invaluable. I admire the books of Studs Terkel, such as \textit{Hard Times: An Oral History of the Great Depression}, which he ``wrote'' by recording interviews with ordinary people \& patching the results into coherent shape. I also like the question-\&-answer interviews, obtained by tape recorder, that are published in certain magazines. They have the sound of spontaneity, the refreshing absence of a writer hovering over the product \& burnishing it to a high gloss.

Strictly, however, this isn't writing. It's a process of asking questions \& then pruning \& splicing \& editing the transcribed answers, \& it takes a tremendous amount of time \& labor. Educated people who you think have been talking into your tape recorder with linear precision turn out to have been stumbling so aimlessly over the sands of language that they haven't completed a single decent sentence. The ear makes allowances for missing grammar, syntax, \& transitions that the eye wouldn't tolerate in print. The seemingly simple use of a tape recorder isn't simple; infinite stitchery is required.

By my main reasons for warning you off it are practical. 1 hazard is that you don't usually have a tape recorder with you; you are more likely to have a pencil. Another is that tape recorders malfunction. Few moments in journalism are as glum as the return of a reporter with ``a really great story,'' followed by his pushing of the \textsc{Play} button \& silence. But above all, a writer should be able to see his materials. If your interview is on tape you become a listener, forever fussing with the machine, running it backward to find the brilliant remark you can never quite find, running it forward, stopping, starting, driving yourself crazy. Be a writer. Write things down.

I do my interviewing by hand, with a sharp No. 1 pencil. I like the transaction with another person. I like the fact that that person can see me \textit{working} -- doing a job, not just sitting there letting a machine do it for me. Only once did I use a tape recorder extensively: for my book, \textit{Mitchell \& Ruff}, about the jazz musicians Willie Ruff \& Dwike Mitchell. Although I knew both men well, I felt that a white writer who presumes to write about the black experience has an obligation to get the tonalities right. It's not that Ruff \& Mitchell speak a different kind of English; they speak good \& often eloquent English. But as Southern blacks they use certain words \& idioms that are distinctive to their heritage, adding richness \& humor to what they say. I didn't want to miss any of those usages. My tape recorder caught them all, \& readers of the book can hear that I got the 2 men right. Consider using a tape recorder in situations where you might violate the cultural integrity of the people you're interviewing.

Taking notes, however, has 1 big problem: the person you're interviewing often starts talking faster than you can write. You are still scribbling Sentence A when he zooms into Sentence B. You drop Sentence \& pursue him into Sentence B, meanwhile trying to hold the rest of Sentence A in your inner ear \& hoping Sentence C will be a dud that you can skip altogether, using the time to catch up. Unfortunately, you now have your subject going at high speed. He is finally saying all the things you have been trying to cajole out of him for an hour, \& saying them with what seems to be Churchillian eloquence. Your inner ear is clogging up with sentences you want to grab before they slip away.

Tell him to stop. Just say, ``Hold it a minute, please,'' \& write until you catch up. What you are trying to do with your feverish scribbling is to quote him correctly, \& nobody wants to be misquoted.

With practice you will write faster \& develop some form of shorthand. You'll find yourself devising abbreviations for often-used words \& also omitting the small connective syntax. As soon as the interview is over, fill in all the missing words you can remember. Complete the uncompleted sentences. Most of them will still be lingering just within the bounds of recall.

When you get home, type out your notes -- probably an almost illegible scrawl -- so that you can read them easily. This not only makes the interview accessible, along with the clippings \& other materials you have assembled. It enables you to review tranquility a torrent of words you wrote in haste, \& thereby discover what the person really said.

You'll find that he said much that's not interesting, or not pertinent, or that's repetitive. Single out the sentences that are most important or colorful. You'll be tempted ot use all the words that are in your notes because you performed the laborious chore of getting them all down. But that's a self-indulgence -- no excuse for putting the reader to the same effort. Your job is to distill the essence.

What about your obligation to the person you interviewed? To what extent can you cut or juggle his words? This question vexes every writer returning from a 1st interview -- \& it should. But the answer isn't hard if you keep in mind 2 standards: brevity \& fair play.

Your ethical duty to the person being interviewed is to present his position accurately. If he carefully weighed 2 sides of an issue \& you only quote his views of 1 side, making him seen to favor that position, you will misrepresent what he told you. Or you might misrepresent him by quoting him out of context, or by choosing only some flashy remark without adding the serious afterthought. You are dealing with a person's honor \& reputation -- \& also with your own.

But after that your duty is to the reader. He or she deserves the tightest package. Most people meander in their conversation, filling it with irrelevant tales \& trivia. Much of it is delightful, but it's still trivia. Your interview will be strong to the extent that you get the main points made without waste. Therefore if you find on page 5 of your notes a comment that perfectly amplifies a point on page 2 -- a point made earlier in the interview -- you will do everyone a favor if you link the 2 thoughts, letting the 2nd sentence follow \& illustrate the 1st. This may violate the truth of how the interview actually progressed, but you will be true to the intent of what was said. Play with the quotes by all means -- selecting, rejecting, thinning, transposing their order, saving a good one for the end. Just make sure the play is fair. Don't change any words or let the cutting of a sentence distort the proper context of what remains.

Do I literally mean ``don't change any words?'' Yes \& no. If a speaker chooses his words carefully you should make it a point of professional pride to quote him verbatim. Most interviewers are sloppy about this; they think that if they achieve a rough approximation it's good enough. It's not good enough: nobody wants to see himself in print using words or phrases he would never use. But if the speaker's conversation is ragged -- if his sentences trail off, if his thoughts are disorderly, if his language is so tangled that it would embarrass him -- the writer has no choice but to clean up the English \& provide the missing links.

Sometimes you can fall into a trap by trying to be too true to the speaker. As you write your article, you type his words exactly as you took them down. You even allow yourself a moment of satisfaction at being such a faithful scribe. Later, editing what you've written, you realize that several of the quotes don't quite make sense. When you 1st heard them they sounded so felicitous that you didn't give them a 2nd thought. Now, on 2nd thought, there's a hole in the language or the logic. To leave the hole is no favor to the reader or the speaker -- \& no credit to the writer. Often you only need to add 1 or 2 clarifying words. Or you might find another quote in your notes that makes the same point clearly. But also remember that you can call the person you interviewed. Tell him you want to check a few of the things he said. Get him to rephrase his points until you understand them. Don't become the prisoner of your quotes -- so lulled by how wonderful they sound that you don't stop to analyze them. Never let anything go out into the world that \textit{you} don't understand.

As for how to organize the interview, the lead should obviously tell the reader why the person is worth reading about. What is his claim to our time \& attention? Thereafter, try to achieve a balance between what the subject is saying in \textit{his} words \& what you are writing in \textit{your} words. If you quote a person for 3 or 4 consecutive paragraphs it becomes monotonous. Quotes are livelier when you break them up, making periodic appearances in your role as guide. You are still the writer -- don't relinquish control. But make your appearances useful; don't just insert 1 of those dreary sentences that shout to the reader that your sole purpose is to break up a string of quotations (``He tapped his pipe on a nearby ashtray \& I noticed that his fingers were quite long.'' ``She toyed idly with her arugula salad'').

When you use a quotation, start the sentence with it. Don't lead up to it with a vapid phrase saying what the man said.
\begin{quotation}
	\textsc{Bad}: Mr. Smith said that he liked to ``go downtown once a week \& have lunch with some of my old friends.''
	\textsc{Good}: ``I usually like to go downtown once a week,'' Mr. Smith said, ``\& have lunch with some of my old friends.''
\end{quotation}
The 2nd sentence has vitality, the 1st one is dead. Nothing is deader than to start a sentence with a ``Mr. Smith said'' construction -- it's where many readers stop reading. If the man said it, let him say it \& get the sentence off to a warm, human start.

But be careful where you break the quotation. Do it as soon as you naturally can, so that the reader knows who is talking, but not where the break will destroy the rhythm or the meaning. Notice how the following 3 variants all inflict some kind of damage:
\begin{quotation}
	``I usually like,'' Mr. Smith said, ``to go downtown once a week \& have lunch with some of my old friends.''
	
	``I usually like to go downtown,'' Mr. Smith said, ``once a week \& have lunch with some of my old friends.''
	
	``I usually like to go downtown once a week \& have lunch,'' Mr. Smith said, ``with some of my old friends.''
\end{quotation}
Finally, don't strain to find synonyms for ``he said.'' Don't make your man assert, aver \& expostulate just to avoid repeating ``he said,'' \& please -- please! -- don't write ``he smiled'' or ``he grinned.'' I've never heard anybody smile. The reader's eye skips over ``he said'' anyway, so it's not worth a lot of fuss. If you crave variety, choose synonyms that catch the shifting nature of the conversation. ``He pointed out,'' ``he explained,'' ``he replied,'' ``he added'' -- these all carry a particular meaning. But don't use ``he added'' if the main is merely averring \& not putting a postscript on what he just said.

All these technical skills, however, can take you just so far. Conducting a good interview is finally related to the character \& personality of the writer, because the person you're interviewing with always know more about the subject than you do. Some ideas on how to overcome your anxiety in this uneven situation, learning to trust your general intelligence, are offered in Chap. 21, ``Enjoyment, fear, \& Confidence.'' The proper \& improper use of quotations has been much in the news, dragged there by some highly visible events. One was the libel \& defamation trial of Janet Malcolm, whom a jury found guilty of ``fabricating'' certain quotes in her \textit{New Yorker} profile of the psychiatrist Jeffrey M. Masson. The other was the revelation by Joe McGinniss that in his biography of Senator Edward M. Kennedy, \textit{The Last Brother}, he had ``written certain scenes \& described certain events from what I have inferred to be his point of view,'' though he never interviewed Kennedy himself. Such blurring of fact \& fiction is a trend that bothers careful writers of nonfiction -- an assault on the craft. Yet even for the conscientious reporter this is uncertain terrain. Let me invoke the work of Joseph Mitchell to suggest some guidelines. The seamless weaving of quotes through his prose was a hallmark of Mitchell's achievement in the brilliant articles he wrote for \textit{The New Yorker} from 1938 to 1965, many of them dealing with people who worked around the New York waterfront. Those articles were hugely influential on nonfiction writers of my generation -- a primary textbook.

The 6 Mitchell pieces that would eventually constitute his book, \textit{The Bottom of the Harbor}, a classic American nonfiction, ran with maddening infrequency in \textit{The New Yorker} during the late 1940s \& early `50s, often several years apart. Sometimes I would ask friends who worked at the magazine when I might expect a new one, but they never knew or even presumed to guess. This was mosaic work, they reminded me, \& the mosaicist was finicky about fitting the pieces together until he got them right. When at last a new article did appear I saw why it had taken so long; it \textit{was} exactly right. I still remember the excitement of reading ``Mr. Hunter's Grave,'' my favorite Mitchell piece. It's about an 87-year-old elder of the African Methodist Church, who was 1 of the last survivors of a 19th-century village of negro oystermen of Staten Island called Sandy Ground. With \textit{The Bottom of the Harbor} the past become a major character in Mitchell's work, giving it a tone both elegiac \& historical. The old men who were his main subject were custodians of memory, a living link with an earlier New York.

The following paragraph, quoting George H. Hunter on the subject of pokeweed, is typical of many very long quotes in ``Mr. Hunter's Grave'' in its leisurely accretion of enjoyable detail:
\begin{quotation}
	``In the spring, when it 1st comes up, the young shoots above the root are good to eat. They taste like asparagus. The old women in Sandy Ground used to believe in eating pokeweed shoots, the old Southern women. They said it renewed your blood. My mother believed it. Every spring she used to send me out in the woods to pick pokeweed shoots. \& I believe it. So every spring, if I think about it, I go pick some \& cook them. It's not that I like them so much -- in fact, they give me gas -- but they remind me of the days gone by, they remind me of my mother. Now, away down here in the woods in this part of Staten Island, you might think you were 15 miles on the other side of nowhere, but just a little ways up Arthur Kill Road, up near Arden Avenue, there's a bend in the road where you can sometimes see the tops of the skyscrapers in New York. Just the tallest skyscrapers, \& just the tops of them. It has to be an extremely clear day. Even then, you might be able to see them 1 moment \& the next moment they're gone. Right beside this bend in the road there's a little swamp, \& the edge of this swamp is the best place I know to pick pokeweed. I went up there 1 morning this spring to pick some, but we had a late spring, if you remember, \& the pokeweed hadn't come up. The fiddleheads were up, \& golden club, \& spring beauty, \& skunk cabbage, \& bluets, but no pokeweed. So I was looking here \& looking there, \& not noticing where I was stepping, \& I made a misstep, \& the next thing I knew I was up to my knees in mud. I floundered around in the mud a minute, getting my bearings, \& then I happened to raise my head \& look up, \& suddenly I saw, away off in the distance, miles \& miles away, the tops of the skyscrapers in New York shining in the morning sun. I wasn't expecting it, \& it was amazing. It was like a vision in the Bible.''
\end{quotation}
Now, nobody thinks Mr. Hunter really said all that in 1 spurt; Mitchell did a heap of splicing. Yet I have no doubt that Mr. Hunter did say it at 1 moment or another -- that all the words \& turns of phrase are hits. It sounds like him; Mitchell didn't write the scene from what he ``inferred'' to be his subject's point of view. He made a literary arrangement, pretending to have spent 1 afternoon being shown around the cemetery, whereas I would guess, knowing his famously patient \& courteous manner \& his lapidary methods, that the article took at least a year of strolling, chatting, writing, \& rewriting. I've seldom read a piece so rich in texture; Mitchell's ``afternoon'' has the unhurried quality of an actual afternoon. By the time it's over, Mr. Hunter, reflecting on the history of oyster fishing in New York harbor, on the passing of generations in Sandy Ground, on families \& family names, planting \& cooking, wildflowers \& fruit, birds \& trees, churches \& funerals, change \& decay, has touched on much of what living is all about.

I have no problem calling ``Mr. Hunter's Grave'' nonfiction. Although Mitchell altered the truth about elapsed time, he used a dramatist's prerogative to compress \& focus his story, thereby giving the reader a manageable framework. If he had told the story in real time, strung across all the days \& months he did spend on Staten Island, he would have achieved the numbing truth of Andy Warhol's 8-hour film of a man having an 8-hour sleep. By careful manipulation he raised the craft of nonfiction to art. But he never manipulated Mr. Hunter's truth; there has been no ``inferring,'' no ``fabricating.'' He has played fair.

That, finally, is my standard. I know that it's just not possible to write a competent interview without some juggling \& eliding of quotes; don't believe any writer who claims he never does it. But many shades of opinion exist on both sides of mine. Purists would say that Joseph Mitchell has taken a novelist's wand to the facts. Progressives would say that Mitchell was a pioneer -- that he anticipated by several decades the ``new journalism'' that writers like Gay Talese \& Tom Wolfe were hailed for inventing in the 1960s, using fictional techniques of imagined dialogue \& emotion to give narrative flair to works whose facts they had punctiliously researched. Both views are partly right.

What's wrong, I believe, is to fabricate quotes or to surmise what someone might have said. Writing is a public trust. The nonfiction writer's rare privilege is to have the whole wonderful of real people to write about. When you get people talking, handle what they say as you would handle a valuable gift.'' -- \cite[pp. 98--111]{Zinsser2016}

%------------------------------------------------------------------------------%

\section{Writing About Places: The Travel Article}

%------------------------------------------------------------------------------%

\section{Writing About Yourself: The Memoir}

%------------------------------------------------------------------------------%

\section{Science \& Technology}

%------------------------------------------------------------------------------%

\section{Business Writing: Writing in Your Job}

%------------------------------------------------------------------------------%

\section{Sports}

%------------------------------------------------------------------------------%

\section{Writing About the Arts: Critics \& Columnists}

%------------------------------------------------------------------------------%

\section{Humor}

%------------------------------------------------------------------------------%

\section{The Sound of Your Voice}

%------------------------------------------------------------------------------%

\section{Enjoyment, Fear, \& Confidence}

%------------------------------------------------------------------------------%

\section{The Tyranny of the Final Product}

%------------------------------------------------------------------------------%

\section{A Writer's Decisions}

%------------------------------------------------------------------------------%

\section{Write as Well as You Can}

%------------------------------------------------------------------------------%

\printbibliography[heading=bibintoc]

\end{document}