\documentclass{article}
\usepackage[backend=biber,natbib=true,style=alphabetic]{biblatex}
\addbibresource{/home/nqbh/reference/bib.bib}
\usepackage{tocloft}
\renewcommand{\cftsecleader}{\cftdotfill{\cftdotsep}}
\usepackage[colorlinks=true,linkcolor=blue,urlcolor=red,citecolor=magenta]{hyperref}
\usepackage{algorithm,algpseudocode,amsmath,amssymb,amsthm,float,graphicx,mathtools}
\usepackage{enumitem}
\setlist{leftmargin=4mm}
\allowdisplaybreaks
\usepackage[left=1cm,right=1cm,top=5mm,bottom=5mm,footskip=4mm]{geometry}
\def\labelitemii{$\circ$}

\title{On Writing Well: The Classic Guide to Writing Nonfiction}
\author{William Zinsser}
\date{\today}

\begin{document}
\maketitle
\tableofcontents
\subsection*{Reference{\tt/}Writing Skills}

\begin{quotation}
	``\textit{On Writing Well} \cite{Zinsser2001, Zinsser2016} is a bible for a generation of writers looking for clues to clean, compelling prose.'' -- \textit{New York Times}
\end{quotation}
``\textit{On Writing Well} has been praised for its sound avice, its clarity \& the warmth of its style. It is a book for everybody who wants to learn how to write or who needs to do some writing to get through the day, as almost everybody does in the age of e-mail \& the Internet.

Whether you want to write about people or places, science \& technology, business, sports, the arts or about yourself in the increasingly popular memoir genre, \textit{On Writing Well} offers you fundamental principles as well as the insights of a distinguished writer \& teacher. With $>10^6$ copies sold, this volume has stood the test of time \& remains a valuable resource for writers \& would-be writers.''
\begin{quotation}
	``Not since \textit{The Elements of Style} has there been a guide to writing as well presented \& readable as this one. A love \& respect for the language is evident on every page.'' -- \textit{Library Journal}
\end{quotation}
\textsc{Books by William Zinsser.} \textit{Any Old Place With You. Seen Any Good Movies Lately? The City Dwellers. Weekend Guests. The Haircurl Papers. Pop Goes America. The Paradise Bit. The Lunacy Boom. On Writing Well. Writing With a Word Processor. Willie \& Dwike} (republished as \textit{Mitchell \& Ruff}). \textit{Writing to Learn. Spring Training. American Places. Speaking of Journalism. Easy to Remember}.

\noindent\textsc{Audio Books by William Zinsser.} \textit{On Writing Well. How to Write a Memoir}.

\noindent\textsc{Books Edited by William Zinsser.} \textit{Extraordinary Lives: The Art \& Craft of American Biography. Inventing the Truth: The Art \& Craft of Memoir. Spiritual Quests: The Art \& Craft of Religious Writing. Paths of Resistance: The Art \& Craft of the Political Novel. Worlds of Childhood: The Art \& Craft of Writing for Children. They Went: The Art \& Craft of Travel Writing. Going on Faith: Writing as a Spiritual Quest}.

%------------------------------------------------------------------------------%

\section*{Introduction}
``When I 1st wrote this book, in 1976, the readers I had in mind were a relatively small segment of the population: students, writers, editors, \& people who wanted to learn to write. I wrote it on a typewriter, the highest technology then available. I had no inkling of the electronic marvels just around the corner that were about to revolutionize the act of writing. 1st came the word processor, in the 1980s, which made the computer an everyday tool for people who had never thought of themselves as writers. Then came the Internet \& e-mail, in the 1990s, which completed the revolution. Today everybody in the world is writing to everybody else, keeping in touch \& doing business across every border \& time zone.

To me this is nothing less than a miracle, curing overnight what appeared to be a deep American disorder. I've been repeated told by people in nonwriting occupations -- especially people in science, technology, medicine, business, \& finance -- that they hate writing \& can't write \& don't want to be made to write. 1 thing they particularly didn't want to write was letters. Just getting started on a letter loomed as a chore with so many formalities -- Where's the stationery? Where's the envelope? Where's the stamp? -- that they would keep putting it off, \& when they finally did sit down to write they would spend the entire 1st paragraph explaining why they hadn't written sooner. In the 2nd paragraph they would describe the weather in their part of the country -- a subject of no interest anywhere else. Only in the 3rd paragraph would they begin to relax \& say what they wanted to say.

Then along came e-mail \& all the formalities went away. E-mail has no etiquette. It doesn't require stationery, or neatness, or proper spelling, or preliminary chitchat. E-mail writers are like people who stop a friend on the sidewalk \& say, ``Did you see the game last night?'' WHAP! No amenities. They just start typing at full speed. So here's the miracle: All those people who said they hate writing \& can't write \& don't want to write \textit{can} write \& \textit{do} want to write. In fact, they can't be turned off. Never have so many Americans written so profusely \& with so few inhibitions. Which means that it wasn't a cognitive problem after all. It was a cultural problem, rooted in that old bugaboo of American education: fear.

Fear of writing gets planted in American schoolchildren at an early age, especially children of scientific or technical or mechanical bent. They are led to believe that writing is a special language owned by the English teacher, available only to the humanistic few who have ``a gift for words.'' But writing isn't a skill that some people are born with \& others aren't, like a gift for art or music. Writing is talking to someone else on paper. Anybody who can think clearly can write clearly, about any subject at all. That has always been the central premise of this book.

On 1 level, therefore, the new fluency created by e-mail is terrific news. Any invention that eliminates the fear of writing is up there with air conditioning \& the lightbulb. But, as always, there's a catch. Nobody told all the new e-mail writers that the essence of writing is rewriting. Just because they are writing with ease \& enjoyment doesn't mean they are writing well.

That condition was 1st revealed in the 1980s, when people began writing on word processors. 2 opposite things happened. The word processor made good writers better \& bad writers worse. Good writers know that very few sentences come out right the 1st time or even the 3rd time or the 5th time. For them the word processor was a rare gift, enabling them to fuss endlessly with their sentences -- cutting \& revising \& reshaping -- without the drudgery of retyping. Bad writers became even more verbose because writing was suddenly so easy \& their sentences looked so pretty on the screen. How could such beautiful sentences not be perfect?

E-mail pushed that verbosity to a new extreme: chatter unlimited. It's a spontaneous medium, not conductive to slowing down or looking back. That makes it ideal for the never-ending upkeep of personal life: maintaining contact with far-flung children \& grandchildren \& friends \& long-lost classmates. If the writing is often garrulous or disorganized or not quite clear, no real harm is done.

But e-mail is also where much of the world's business is now conducted. Millions of e-mail messages every day give people the information they need to do their job, \& a badly written message can cause a lot of damage. Employers have begun to realize that they literally cannot afford to hire men \& women who can't write sentences that are tight \& logical \& clear. The new information age, for all its high-tech gadgetry, is, finally, writing-based. E-mail, the Internet \& the fax are all forms of writing, \& writing is, finally, a craft, with its own set of tools, which are words. Like all tools, they have to be used right.

\textit{On Writing Well} is a craft book. That's what I set out to write 25 years ago -- a book that would teach the craft of writing warmly \& clearly -- \& its principles have never changed; they are as valid in the digital age as they were in the age of the typewriter. I don't mean that the book itself hasn't changed. I've revised \& expanded it 5 times since 1976 to keep pace with new trends in the language \& in society: a far greater interest in memoir-writing, e.g., \& in writing about business \& science \& sports, \& in nonfiction writing by women \& by newcomers to the United States from other cultural traditions.

I'm also not the same person I was 25 years ago. Books that teach, if they have a long life, should reflect who the writer has become at later stages of his own long life -- what he has been doing \& thinking about. \textit{On Writing Well} \& I have grown older \& wiser together. In each of the 5 new editions the new material consisted of things I had learned since the previous edition by continuing to wrestle with the craft as a writer. As a teacher, I've become far more preoccupied with the intangibles of the craft -- the attitudes \& values, like enjoyment \& confidence \& intention, that keep us going \& produce our best work. But it wasn't until the 6th edition that I knew enough to write the 2 chapters (21 \& 22) that deal at proper length with those attitudes \& values.

Ultimately, however, good writing rests on craft \& always will. I don't know what still newer electronic marvels are waiting just around the corner to make writing twice as easy \& twice as fast in the next 25 years. But I do know they won't make writing twice as good. That will still require plain old hard work -- clear thinking -- \& the plain old tools of the English language. \textsc{William Zinsser}, Sep 2001.'' -- \cite[pp. ix--xii]{Zinsser2001}

``1 of the pictures hanging in my office in mid-Manhattan is a photograph of the writer E. B. White. It was taken by Jill Krementz when White was 77 years old, at his home in North Brooklin, Maine. A white-haired man is sitting on a plain wooden bench at a plain wooden table -- 3 boards nailed to 4 legs -- in a small boathouse. The window is open to a view across the water. White is typing on a manual typewriter, \& the only other objects are an ashtray \& a nail keg. The keg, I don't have to be told, is his wastebasket.

Many people from many corners of my life -- writers \& aspiring writers, students \& former students -- have seen that picture. They come to talk through a writing problem or to catch me up on their lives. But usually it doesn't take more than a few minutes for their eye to be drawn to the old man sitting at the typewriter. What gets their attention is the simplicity of the process. White has everything he needs: a writing implement, a piece of paper, \& a receptable for all the sentences that didn't come out the way he wanted them to.

Since then writing has gone electronic. Computers have replaced the typewriter, the delete key has replaced the wastebasket, \& various other keys insert, move, \& rearrange whole chunks of text. But nothing has replaced the writer. He or she is still stuck with the same old job of saying something that other people will want to read. That's the point of the photograph, \& it's still the point -- 30 years later -- of this book.

I 1st wrote \textit{On Writing Well} in an outbuilding in Connecticut that was as small \& as crude as White's boathouse. My tools were a dangling lightbulb, an Underwood standard typewriter, a ream of yellow copy paper \& a wire wastebasket. I had then been teaching my nonfiction writing course at Yale for 5 years, \& I wanted to use the summer of 1975 to try to put the course into a book.

E. B. White, as it happened, was very much on my mind. I had long considered him my model as a writer. His was the seemingly effortless style -- achieved, I knew, with great effort -- that I wanted to emulate, \& whenever I began a new project I would 1st read some White to get his cadences into my ear. But now I also had a pedagogical interest: White was the reigning champ of the arena I was trying to enter. \textit{The Elements of Style}, his updating of the book that had most influenced \textit{him}, written in 1919 by his English professor at Cornell, William Strunk, Jr., was the dominant how-to manual for writers. Tough competition.

Instead of competing with the Strunk \& White book I decided to complement it. \textit{The Elements of Style} was a book of pointers \& admonitions: do this, don't do that. What it \textit{didn't} address was how to apply those principles to the various forms that nonfiction writing \& journalism can take. That's what I taught in my course, \& it's what I would teach in my book: how to write about people \& places, science \& technology, history \& medicine, business \& education, sports \& that arts \& everything else under the sun that's waiting to be written about.

So \textit{On Writing Well} was born, in 1976, \& it's now in its 3rd generation of readers, its sales well over a million. Today I often meet young newspaper reporters who were given the book by the editor who hired them, just as those editors were 1st given the book by the editor who hired \textit{them}. I also often meet gray-haired matrons who remember being assigned the book in college \& not finding it the horrible medicine they expected. Sometimes they bring that early edition for me to sign, its sentences highlighted in yellow. They apologize for the mess. I love the mess.

As America has steadily changed in 30 years, so has the book. I've revised it 6 times to keep pace with new social trends (more interest in memoir, business, science, \& sports), new literary trends (more women writing nonfiction), new demographic patterns (more writers from other cultural traditions), new technologies (the computer) \& new words \& usages. I've also incorporated lessons I learned by continuing to wrestle with the craft myself, writing books on subjects I hadn't tried before: baseball \& music \& American history. My purpose is to make myself \& my experience available. If readers connect with my book it's because they don't think they're hearing from an English professor. They're hearing from a working writer.

My concerns as a teacher have also shifted. I'm more interested in the intangibles that produce good writing -- confidence, enjoyment, intention, integrity -- \& I've written new chapters on those values. Since the 1990s I've also taught an adult course on memoir \& family history at the New School. My students are men \& women who want to use writing to try to understand who they are \& what heritage they were born into. Year after year their stories take me deeply into their lives \& into their yearning to leave a record of what they have done \& thought \& felt. Half the people in America, it seems, are writing a memoir.

The bad news is that most of them are paralyzed by the size of the task. How can they begin to impose a coherent shape on the past -- that vast sprawl of half-remembered people \& events \& emotions? Many are near despair. To offer some help \& comfort I wrote a book in 2004 called \textit{Writing About Your Life}. It's a memoir of various events in my own life, but it's also a teaching book: along the way I explain the writing decisions I made. They are the same decisions that confront every writer going in search of his or her past: matters of selection, reduction, organization, \& tone. Now, for this 17th edition, I've put the lessons I learned into a new chapter called ``Writing Family History \& Memoir.''

When I 1st wrote \textit{On Writing Well}, the readers I had in mind were a small segment of the population: students, writers, editors, teachers, \& people who wanted to learn how to write. I had no inkling of the electronic marvels that would soon revolutionize the act of writing. 1st came the word processor, in the 1980s, which made the computer an everyday  tool for people who had never thought of themselves as writers. Then came the Internet \& e-mail, in the 1990s, which continued the revolution. Today everybody in the world is writing to everybody else, making instant contact across every border \& across every time zone. Bloggers are saturating the globe.

On 1 level the new torrent is good news. Any invention that reduces the fear of writing is up there with air-conditioning \& the lightbulb. But, as always, there's a catch. Nobody told all the new computer writers that the essence of writing is rewriting. Just because they're writing fluently doesn't mean they're writing well.

That condition was 1st revealed with the arrival of the word processor. 2 opposite things happened: good writers go better \& bad writers got worse. Good writers welcomed the gift of being able to fuss endlessly with their sentences -- pruning \& revising \& reshaping -- without the drudgery of retyping. Bad writers became even more verbose because writing was suddenly so easy \& their sentences looked so pretty on the screen. How could such beautiful sentences not be perfect?

E-mail is an impromptu medium, not conducive to slowing down or looking back. It's ideal for the never-ending upkeep of daily life. If the writing is disorderly, no real harm is done. But e-mail is also where much of the world's business is now conducted. Millions of e-mail messages every day give people the information they need to do their job, \& a badly written message can do a lot of damage. So can a badly written Web site. The new age, for all its electronic wizardry, is still writing-based.

\textit{On Writing Well} is a craft book, \& its principles haven't changed since it was written 30 years ago. I don't know what still newer marvels will make writing twice as easy in the next 30 years. But I do know they won't make writing twice as good. That will still require plain old hard thinking -- what E. B. White was doing in his boathouse -- \& the plain old tools of the English language. \textsc{William Zinsser}, Apr 2006.'' -- \cite[pp. 5--8]{Zinsser2016}

%------------------------------------------------------------------------------%

\begin{center}\huge
	Part I: Principles
\end{center}

%------------------------------------------------------------------------------%

\section{The Transaction}
``A school in Connecticut once held ``a day devoted to the arts,'' \& I was asked if I would come \& talk about writing as a vocation. When I arrived I found that a 2nd speaker had been invited -- Dr. Brock (as I'll call him), a surgeon who had recently begun to write \& had sold some stories to magazines. He was going to talk about writing as an avocation. That made us a panel, \& we sat down to face a crowd of students \& teachers \& parents, all eager to learn the secrets of our glamorous work.

Dr. Brock was dressed in a bright red jacket, looking vaguely bohemian, as authors are supposed to look, \& the 1st question went to him. What was it like to be a writer?

He said it was tremendous fun. Coming home from an arduous day at the hospital, he would go straight to his yellow pad \& write his tensions away. The words just flowed. It was easy. I then said that writing wasn't easy \& wasn't fun. It was hard \& lonely, \& the words seldom just flowed.

Next Dr. Brock was asked if it was important to rewrite. Absolutely not, he said. ``Let it all hang out,'' he told us, \& whatever form the sentences take will reflect the writer at his most natural. I then said that rewriting is the essence of writing. I pointed out that professional writers rewrite their sentences over \& over \& then rewrite what they have rewritten.

``What do you do on days when it isn't going well?'' Dr. Brock was asked. He said he just stopped writing \& put the work aside for a day when it would go better. I then said that the professional writer must establish a daily schedule \& stick to it. I said that writing is a craft, not an art, \& that the man who runs away from his craft because he lacks inspiration is fooling himself. He is also going broke.

``What if you're feeling depressed or unhappy?'' a student asked. ``Won't that affect your writing?''

Probably it will, Dr. Brock replied. Go fishing. Take a walk. Probably it won't, I said. If your job is to write every day, you learn to do it like any other job.

A student asked if we found it useful to circulate in the literary world. Dr. Brock said he was greatly enjoying his new life as a man of letters, \& he told several stories of being taken to lunch by his publisher \& his agent at Manhattan restaurants where writers \& editors gather. I said that professional writers are solitary drudges who seldom see other writers.

``Do you put symbolism in your writing?'' a student asked me.

``Not if I can help it,'' I replied. I have an unbroken record of missing the deeper meaning in any story, play or movie, \& as for dance \& mime, I have never had any idea of what is being conveyed.

``I \textit{love} symbols!'' Dr. Brock exclaimed, \& he described with gusto the joys of weaving them through his work.

So the morning went, \& it was a revelation to all of us. At the end Dr. Brock told me he was enormously interested in my answers -- it had never occurred to him that writing could be hard. I told him I was just as interested in \textit{his} answers -- it had never occurred to me that writing could be easy. Maybe I should take up surgery on the side.

As for the students, anyone might think we left them bewildered. But in fact we gave them a broader glimpse of the writing process than if only 1 of us had talked. For there isn't any ``right'' way to do such personal work. There are all kinds of writers \& all kinds of methods, \& any method that helps you to say what you want to say is the right method for you. Some people write by day, others by night. Some people need silence, others turn on the radio. Some write by hand, some by computer, some by talking into a tape recorder. Some people write their 1st draft in 1 long burst \& then revise; others can't write the 2nd paragraph until they have fiddled endlessly with the 1st.

But all of them are vulnerable \& all of them are tense. They are driven by a compulsion to put some part of themselves on paper, \& yet they don't just write what comes naturally. They sit down to commit an act of literature, \& the self who emerges on paper is far stiffer than the person who sat down to write. The problem is to find the real man or woman behind the tension.

Ultimately the product that any writer has to sell is not the subject being written about, but who he or she is. I often find myself reading with interest about a topic I never thought would interest me -- some scientific quest, perhaps. What holds me is the enthusiasm of the writer for his field. How was he drawn into it? What emotional baggage did he bring along? How did it change his life? It's not necessary to want to spend a year alone at Walden Pond to become involved with a writer who did.

This is the personal transaction that's at the heart of good nonfiction writing. Out of it come 2 of the most important qualities that this book will go in search of: humanity \& warmth. Good writing has an aliveness that keeps the reader reading from 1 paragraph to the next, \& it's not a question of gimmicks to ``personalize'' the author. It's a question of using the English language in a way that will achieve the greatest clarity \& strength.

Can such principles be taught? Maybe not. But most of them can be learned.'' -- \cite[pp. 12--13]{Zinsser2016}

%------------------------------------------------------------------------------%

\section{Simplicity}
``Clutter is the disease of American writing. We are a society strangling in unnecessary words, circular constructions, pompous frills \& meaningless jargon.

Who can understand the clotted language of everyday American commerce: the memo, the corporation report, the business letter, the notice from the bank explaining its latest ``simplified'' statement? What member of an insurance or medical plan can decipher the brochure explaining his costs \& benefits? What father or mother can put together a child's toy from the instructions on the box? Our national tendency is to inflate \& thereby sound important. The airline pilot who announces that he is presently anticipating experiencing considerable precipitation wouldn't think of saying it may rain. The sentence is too simple -- there must be something wrong with it.

But the secret of good writing is to strip every sentence to its cleanest components. Every word that serves no function, every long word that could be a short word, every adverb that carries the same meaning that's already in the verb, every passive construction that leaves the reader unsure of who is doing what -- these are the 1001 adulterants that weaken the strength of a sentence. \& they usually occur in proportion to education \& rank.

During the 1960s the president of my university wrote a letter to mollify the alumni after a spell of campus unrest. ``You are probably aware,'' he began, ``that we have been experiencing very considerable potentially explosive expressions of dissatisfaction on issues only partially related.'' He meant that the students had been hassling them about different things. I was far more upset by the president's English than by the students' potentially explosive expressions of dissatisfaction. I would have preferred the presidential approach taken by Franklin D. Roosevelt when he tried to convert into English his own government's memos, such as this blackout order of 1942:
\begin{quotation}
	Such preparations shall be made as will completely obscure all Federal buildings \& non-Federal buildings occupied by the Federal government during an air raid for any period of time from visibility by reason of internal or external illumination.
\end{quotation}
``Tell them,'' Roosevelt said, ``that in buildings where they have to keep the work going to put something across the windows.''

Simplify, simplify. Thoreau said it, as we are so often reminded, \& no American writer more consistently practiced what he preached. Open \textit{Walden} to any page \& you will find a man saying in a plain \& orderly way what is on his mind:
\begin{quotation}
	I went to the woods because I wished to live deliberately, to front only the essential facts of life, \& see if I could not learn what it had to teach, \& not, when I came to die, discover that I had not lived.
\end{quotation}
How can the rest of us achieve such enviable freedom from clutter? The answer is to clear our heads of clutter. Clear thinking becomes clear writing; one can't exist without the other. It's impossible for a muddy thinker to write good English. He may get away with it for a paragraph or 2, but soon the reader will be lost, \& there's no sin so grave, for the reader will not easily be lured back.

Who is this elusive creative, the reader? The reader is someone with an attention span of about 30 s -- a person assailed by many forces competing for attention. At 1 time those forces were relatively few: newspapers, magazines, radio, spouse, children, pets. Today they also include a galaxy of electronic devices for receiving entertainment \& information -- television, VCRs, DVDs, CDs, video games, the Internet, e-mail, cell phones, BlackBerries, iPods -- as well as a fitness program, a pool, a lawn \& that most potent of competitors, sleep. The man or woman snoozing in a chair with a magazine or a book is a person who was being given too much unnecessary trouble by the writer. 

It won't do to say that the reader is too dumb or too lazy to keep pace with the train of thought. If the reader is lost, it's usually because the writer hasn't been careful enough. That carelessness can take any number of forms. Perhaps a sentence is so excessively cluttered that the reader, hacking through the verbiage, simply doesn't know what it means. Perhaps a sentence has been so shoddily constructed that the reader could read it in several ways. Perhaps the writer has switched pronouns in midsentence, or has switched tenses, so the reader loses track of who is talking or when the action took place. Perhaps Sentence B is not a logical sequel to Sentence A; the writer, in whose head the connection is clear, hasn't bothered to provide the missing link. Perhaps the writer has used a word incorrectly by not taking the trouble to look it up.

Faced with such obstacles, readers are at 1st tenacious. They blame themselves -- they obviously missed something, \& they go back over the mystifying sentence, or over the whole paragraph, piecing it out like an ancient rule, making guesses \& moving on. But they won't do that for long. The writer is making them work too hard, \& they will look for one who is better at the craft.

Writers must therefore constantly ask: what am I trying to say? Surprising often they don't know. Then they must look at what they have written \& ask: have I said it? Is it clear to someone encountering the subject for the 1st time? If it's not, some fuzz worked its way into the machinery. The clear writer is someone clearheaded enough to see this stuff for what it is: fuzz.

I don't mean that some people are born clearheaded \& are therefore natural writers, whereas others are naturally fuzzy \& will never write well. Thinking clearly is a conscious act that writers must force on themselves, as if they were working an any other project that requires logic: making a shopping list or doing an algebra problem. Good writing doesn't come naturally, though most people seem to think it does. Professional writers are constantly bearded by people who say they'd like to ``try a little writing someone'' -- meaning when they retire from their real profession, like insurance or real estate, which is hard. Or they say, ``I could write a book about that.'' I doubt it.

Writing is hard work. A clear sentence is no accident. Very few sentences come out right the 1st time, or even the 3rd time. Remember this in moments of despair. If you find that writing is hard, it's because it \textit{is} hard.

2 pages of the final manuscript of this chapter from the 1st Edition of \textit{On Writing Well}. Although they look like a 1st draft, they had already been rewritten \& retyped -- like almost every other page -- 4 or 5 times. With each rewrite I try to make what I have written tighter, stronger \& more precise, eliminating every element that's not doing useful work. Then I go over it once more, reading it aloud, \& am always amazed at how much clutter can still be cut. (In later editions I eliminated the sexist pronoun ``he'' denoting ``the writer'' \& ``the reader.'')'' -- \cite[pp. 15--17]{Zinsser2016}

%------------------------------------------------------------------------------%

\section{Clutter}
``Fighting clutter is like fighting weeds -- the writer is always slightly behind. New varieties sprout overnight, \& by noon they are part of American speech. Consider what President Nixon's aide John Dean accomplished in just 1 day of testimony on television during the Watergate hearings. The next day everyone in America was saying ``at this point in time'' instead of ``now.''

Consider all the prepositions that are draped onto verbs that don't need any help. We no longer head committees. We head them up. We don't face problems anymore. We face up to them when we can free up a few minutes. A small detail, you may say -- not worth bothering about. It \textit{is} worth bothering about. Writing improves in direct ratio to the number of things we can keep out of it that shouldn't be there. ``Up'' in ``free up'' shouldn't be there. Examine every word you put on paper. You'll find a surprising number that don't serve any purpose.

Take the adjective ``personal,'' as in ``a personal friend of mine,'' ``his personal feeling'' or ``her personal physician.'' It's typical of hundreds of words that can be eliminated. The personal friends has come into the language to distinguish him or her from the business friend, thereby debasing both language \& friendship. Someone's feeling \textit{is} that person's personal feeling -- that's what ``his'' means. As for the personal physician, that's the man or woman summoned to the dressing room of a stricken actress so she won't have to be treated by the impersonal physician assigned to the theater. Someday I'd like to see that person identified as ``her doctor.'' Physicians are physicians, friends are friends. The rest is clutter.

Clutter is the laborious phrase that has pushed out the short word that means the same thing. Even before John Dean, people \& businesses had stopped saying ``now.'' They were saying ``currently'' (``all our operators are currently assisting other customers''), or ``at the present time,'' or ``presently'' (which means ``soon'). Yet the idea can always be expressed by ``now'' to mean the immediate moment (``Now I can see him''), or by ``today'' to mean the historical present (``Today prices are high''), or simply by the verb ``to be'' (``It is raining''). There's no need to say, ``At the present time we are experiencing precipitation.''

``Experiencing'' is 1 of the worst clutterers. Even your dentist will ask if you are experiencing any pain. If he had his own kid in the chair he would say, ``Does it hurt?'' He would, in short, be himself. By using a more pompous phrase in his professional role he not only sounds more important; he blunts the painful edge of truth. It's the language of the flight attendant demonstrating the oxygen mask that will drop down if the plane should run out of air. ``In the unlikely possibility that the aircraft should experience such an eventuality,'' she begins -- a phrase so oxygen-depriving in itself that we are prepared for any disaster.

Clutter is the ponderous euphemism that turns a slum into a depressed socioeconomic area, garbage collectors into waste-disposal personnel \& the town dump into the volume reduction unit. I think of Bill Mauldin's cartoon of 2 hoboes riding a freight car. 1 of them says, ``I started as a simple bum, but now I'm hard-core unemployed.'' Clutter is political correctness gone amok. I saw an ad for a boys' camp designed to provide ``individual attention for the minimally exceptional.''

Clutter is the official language used by corporations to hide their mistakes. When the Digital Equipment Corporation eliminated 3000 jobs its statement didn't mention layoffs; those were ``involuntary methodologies.'' When an Air Force missile crashed, it ``impacted with the ground prematurely.'' When General Motors had a plant shutdown, that was a ``volume-related production-schedule adjustment.'' Companies that go belly-up have ``a negative cash-flow position.''

Clutter is the language of the Pentagon calling an invasion a ``reinforced protective reaction strike'' \& justifying its vast budgets on the need for ``counterforce deterrence.'' As George Orwell pointed out in ``Politics \& the English Language,'' an essay written in 1946 but often cited during the wars in Cambodia, Vietnam, \& Iraq, ``political speech \& writing are largely the defense of the indefensible $\ldots$ Thus political language has to consist largely of euphemism, question-begging \& sheer cloudy vagueness.'' Orwell's warning that clutter is not just a nuisance but a deadly tool has come true in the recent decades of American military adventurism. It was during George W. Bush's presidency that ``civilian casualities'' in Iraq became ``collateral damage.''

Verbal camouflage reached new heights during General Alexander Haig's tenure as President Reagan's secretary of state. Before Haig nobody had thought of saying ``at this juncture of maturization'' to mean ``now.'' He told the American people that terrorism could be fought with ``meaningful sanctionary teeth'' \& that intermediate nuclear missiles were ``at the vortex of cruciality.'' As for any worries that the public might harbor, his message was ``leave it to AI,'' though what he actually said was: ``We must push this to a lower decibel of public fixation. I don't think there's much of a learning curve to be achieved in this area of content.''

I could go on quoting examples from various fields -- every profession has its growing arsenal of jargon to throw dust in the eyes of the populace. But the list would be tedious. The point of raising it now is to serve notice that clutter is the enemy. Beware, then, of the long word that's no better than the short word: ``assistance'' (help), ``numerous'' (many), ``facilitate'' (ease), ``individual'' (man or woman), ``remainder'' (rest), ``initial'' (1st), ``implement'' (do), ``sufficient'' (enough), ``attempt'' (try), ``referred to as'' (called) \& hundreds more. Beware of all the slippery new fad words: paradigm \& parameter, prioritize \& potentialize. They are all weeds that will smother what you write. Don't dialogue with someone you can talk to. Don't interface with anybody.

Just as insidious are all the word clusters with which we explain how we propose to go about our explaining: ``I might add,'' ``It should be pointed out,'' ``It is interesting to note.'' If you might add, add it. If it should be pointed out, point it out. If it is interesting to note, \textit{make} it interesting; are we not all stupefied by what follows when someone says, ``This will interest you''? Don't inflate what needs no inflating: ``with the possible exception of'' (except), ``due to the fact that'' (because), ``he totally lacked the ability to'' (he couldn't), ``until such time as'' (until), ``for the purpose of'' (for).

Is there any way to recognize clutter at a glance? Here's a device my students at Yale found helpful. I would put brackets around every component in a piece of writing that wasn't doing useful work. Often just 1 word got bracketed: the unnecessary preposition appended to a verb (``order up''), or the adverb that carries the same meaning as the verb (``smile happily''), or the adjective that states a known fact (``tall skyscraper''). Often my brackets surrounded the little qualifiers that weaken any sentence they inhabit (``a bit,'' ``sort of''), or phrases like ``in a sense,'' which don't mean anything. Sometimes my brackets surrounded an entire sentence -- the one that essentially repeats what the previous sentence said, or that says something readers don't  need to know or can figure out for themselves. Most 1st drafts can be cut by 50\% without losing any information or losing the author's voice.

My reason for bracketing the students' superfluous words, instead of crossing them out, was to avoid violating their sacred prose. I wanted to leave the sentence intact for them to analyze. I was saying, ``I may be wrong, but I think this can be deleted \& the meaning won't be affected. But \textit{you} decide. Read the sentence without the bracketed material \& see if it works.'' In early weeks of the term I handed back papers that were festooned with brackets. Entire paragraphs were bracketed. But soon the students learned to put mental brackets around their own clutter, \& by the end of the term their papers were almost clean. Today many of those students are professional writers, \& they tell me, ``I still see your brackets -- they're following me through life.''

You can develop the same eye. Look for the clutter in your writing \& prune it ruthlessly. Be grateful for everything you can throw away. Reexamine each sentence you put on paper. Is every word doing new work? Can any thought be expressed with more economy? Is anything pompous or pretentious or faddish? Are you hanging on to something unless just because you think it's beautiful?

Simplify, simplify.'' -- \cite[pp. 19--22]{Zinsser2016}

%------------------------------------------------------------------------------%

\section{Style}
``So much for early warnings about the bloated monsters that lie in ambush for the writer trying to put together a clean English sentence.

``But,'' you may say, ``if I eliminate everything you think is clutter \& if I strip every sentence to its barest bones, will there by anything left of me?'' The question is a fair one; simplicity carried to an extreme might seem to point to a style little more sophisticated than ``Dick likes Jane'' \& ``See Spot run.''

I'll answer the question 1st on the level of carpentry. Then I'll get to the larger issue of who the writer is \& how to preserve his or her identity.

Few people realize how badly they write. Nobody has shown them how much excess or murkiness has crept into their style \& how it obstructs what they are trying to say. If you give me an 8-page article \& I tell you to cut it to 4 pages, you'll howl \& say it can't be done. Then you'll go home \& do it, \& it will be much better. After that comes the hard part: cutting it to 3.

The point is that you have to strip your writing down before you can build it back up. You must know what the essential tools are \& what job they were designed to do. Extending the metaphor of carpentry, it's 1st necessary to be able to saw wood neatly \& to drive nails. Later you can bevel the edges or add elegant finials, if that's your taste. But you can never forget that you are practicing a craft that's based on certain principles. If the nails are weak, your house will collapse. If your verbs are weak \& your syntax is rickety, your sentences will fall apart.

I'll admit that certain nonfiction writers, like Tom Wolfe \& Norman Mailer, have built some remarkable houses. But these are writers who spent years learning their crafts, \& when at last they raised their fanciful turrets \& hanging gardens, to the surprise of all of us who never dreamed of such ornamentation, they knew what they were doing. Nobody becomes Tom Wolfe overnight, not even Tom Wolfe.

1st, then, learn to hammer the nails, \& if what you build is sturdy \& serviceable, take satisfaction in its plain strength.

But you will be impatient to find a ``style'' -- to embellish the plain words so that readers will recognize you as someone special. You will reach for gaudy similes \& tinseled adjectives, as if ``style'' were something you could buy at the style store \& drape onto your words in bright decorator colors. (Decorator colors are the colors that decorators come in.) There is no style store; style is organic to the person doing the writing, as much a part of him as his hair, or, if he is bald, his lack of it. Trying to add style is like adding a toupee. At 1st glance the formerly bald man looks young \& even handsome. But at 2nd glance -- \& with a toupee there's always a 2nd glance -- he doesn't look quite right. The problem is not that he doesn't look well groomed; he does, \& we can only admire the wigmaker's skill. The point is that he doesn't look like himself.

This is the problem of writers who set out deliberately to garnish their prose. You lose whatever it is that makes you unique. The reader will notice if you are putting on airs. Readers want the person who is talking to them to sound genuine. Therefore a fundamental rule is: be yourself.

No rule, however, is harder to follow. It requires writers to do 2 things that by their metabolism are impossible. They must relax, \& they must have confidence.

Telling a writer to relax is like telling a man to relax while being examined for a hernia, \& as for confidence, see how stiffly he sits, glaring at the screen that awaits his words. See how often he gets up to look for something to eat or drink. A writer will do anything to avoid the act of writing. I can testify from my newspaper days that the number of trips to the water cooler per reporter-hour far exceeds the body's need for fluids.

What can be done to put the writer out of these miseries? Unfortunately, no cure has been found. I can only offer the consoling thought that you are not alone. Some days will go better than others. Some will go so badly that you'll despair of ever writing again. We have all had many of those days \& will have many more.

Still, it would be nice to keep the bad days to a minimum, which brings me back to the problem of trying to relax.

Assume that you are the writer sitting down to write. You think your article must be of a certain length or it won't seem important. You think how august it will look in print. You think of all the people who will read it. You think that it must have the solid weight of authority. You think that its style must dazzle. No wonder you tighten; you are so busy thinking of your awesome responsibility to the finished article that you can't even start. Yet you vow to be worthy of the task, \&, casting about for grand phrases that wouldn't occur to you if you weren't trying so hard to make an impression, you plunge in.

Paragraph 1 is a disaster -- a tissue of generalities that seem to have come out of a machine. No \textit{person} could have written them. Paragraphs 2 isn't much better. But Paragraph 3 begins to have a somewhat human quality, \& by Paragraph 4 you begin to sound like yourself. You've started to relax. It's amazing how often an editor can throw away the 1st 3 or 4 paragraphs of an article, or even the 1st few pages, \& start with the paragraph where the writer begins to sound like himself or herself. Not only are those 1st paragraphs impersonal \& ornate; they don't say anything -- they are a self-conscious attempt at a fancy prologue. What I'm always looking for as an editor is a sentence that says something like ``I'll never forget the day when I $\ldots$'' I think, ``Aha! A person!''

Writers are obviously at their most natural when they write in the 1st person. Writing is an intimate transaction between 2 people, conducted on paper, \& it will go well to the extent that it retains its humanity. Therefore I urge people to write in the 1st person: to use ``I'' \& ``me'' \& ``we'' \& ``us.'' They put up a fight.

``Who am I to say what \textit{I} think?'' they ask. ``Or what \textit{I} fee?''

``Who are you \textit{not} to say what you think?'' I tell them. ``There's only 1 you. Nobody else thinks or feels in exactly the same way.''

``But nobody cares about my opinions,'' they say. ``It would make me feel conspicuous.''

``They'll care if you tell them something interesting,'' I say, ``\& tell them in words that come naturally.''

Nevertheless, getting writers to use ``I'' is seldom easy. They think they must earn the right to reveal their emotions or their thoughts. Or that it's egotistical. Or that it's undignified -- a fear that afflicts the academic world. Hence the professorial use of ``one'' (``One finds oneself not wholly in accord with Dr. Maltby's view of the human condition''), or of the impersonal ``it is'' (``It is to be hoped that Prof. Felt's monograph will find the wider audience it most assuredly deserves''). I don't want to meet ``one'' -- he's a boring guy. I want a professor with a passion for his subject to tell me why it fascinates \textit{him}.

I realize that there are vast regions of writing where ``I'' isn't allowed. Newspapers don't want ``I'' in their news stories; many magazines don't want it in their articles; businesses \& institutions don't want it in the reports they send so profusely into the American home; colleges don't want ``I'' in their term papers or dissertations, \& English teachers discourage any 1st-person pronoun except the literary ``we'' (``We see in Melville's symbolic use of the white whale $\ldots$''). Many of those prohibitions are valid; newspaper articles should consist of news, reported objectively. I also sympathize with teachers who don't want to give students an easy escape into opinion -- ``I think Hamlet was stupid'' -- before they have grappled with the discipline of assessing a work on its merits \& on external sources. ``I'' can be a self-indulgence \& a cop-out.

Still, we have become a society fearful of revealing who we are. The institutions that seek our support by sending us their brochures sound remarkably alike, though surely all of them -- hospitals, schools, libraries, museums, zoos -- were founded \& are still sustained by men \& women with different dreams \& visions. Where are these people? It's hard to glimpse them among all the impersonal passive sentences that say ``initiatives were undertaken'' \& ``priorities have been identified.''

Even when ``I'' isn't permitted, it's still possible to convey a sense of I-ness. The political columnist James Reston didn't use ``I'' in his columns; yet I had a good idea of what kind of person he was, \& I could say the same of many other essayists \& reporters. Good writers are visible just behind their words. If you aren't allowed to use ``I,'' at least think ``I'' while you write, or write the 1st draft in the 1st person \& they take the ``I''s out. It will warm up your impersonal style.

Style is tied to the psyche, \& writing has deep psychological roots. The reasons why we express ourselves as we do, or fail to express ourselves because of ``writer's block,'' are partly buried in the subconscious mind. There are as many kinds of writer's block as there are kinds of writers, \& I have no intention of trying to untangle them. This is a short book, \& my name isn't Sigmund Freud.

But I've also noticed a new reason for avoiding ``I'': Americans are unwilling to go out on a limb. A generalization ago our leaders told us where they stood \& what they believed. Today they perform strenuous verbal feats to escape that fate. Watch them wriggle through TV interviews without committing themselves. I remember President Ford assuring a group of visiting businessmen that his fiscal policies would work. He said: ``We see nothing but increasingly brighter clouds every month.'' I took this to mean that the clouds were still fairly dark. Ford's sentence was just vague enough to say nothing \& still sedate his constituents.

Later administrations brought no relief. Defense Secretary Caspar Weinberger, assessing a Polish crisis in 1984, said: ``There's continuing ground for serious concern \& the situation remains serious. The longer it remains serious, the more ground there is for serious concern.'' The 1st President Bush, questioned about his stand on assault rifles, said: ``There are various groups that think you can ban certain kinds of guns. I am not in that mode. I am in the mode of being deeply concerned.''

But my all-time champ is Elliot Richardson, who held 4 major cabinet positions in the 1970s. It's hard to know where to begin picking from his trove of equivocal statements, but consider this one: ``\& yet, on balance, affirmative action has, I think, been a qualified success.'' A 13-word sentence with 5 hedging words. I give it 1st prize as the most wishy-washy sentence in modern public discourse, though a rival would be his analysis of how to ease boredom among assembly-line workers: ``\& so, at last, I come to the 1 firm conviction that I mentioned at the beginning: it is that the subject is too new for final judgments.''

That's a firm conviction? Leaders who bob \& weave like aging boxers don't inspire confidence -- or deserve it. The same thing is true of writers. Sell yourself, \& your subject will exert its own appeal. Believe in your own identity \& your own opinions. Writing is an act of ego, \& you might as well admit it. Use its energy to keep yourself going.'' -- \cite[pp. 24--28]{Zinsser2016}

%------------------------------------------------------------------------------%

\section{The Audience}
``Soon after you confront the matter of preserving your identity, another question will occur to you: ``Who am I writing for?''

It's a fundamental question, \& it has a fundamental answer: You are writing for yourself. Don't try to visualize the great mass audience. There is no such audience -- every reader is a different person. Don't try to guess what sort of thing editors want to publish or what you think the country is in a mood to read. Editors \& readers don't know what they want to read until they read it. Besides, they're always looking for something new.

Don't worry about whether the reader will ``get it'' if you indulge a sudden impulse for humor. If it amuses you in the act of writing, put it in. (It can always be taken out, but only you can put it in.) You are writing primarily to please yourself, \& if you go about it with enjoyment you will also entertain the readers who are worth writing for. If you lose the dullards back in the dust, you don't want them anyway.

This may seem to be a paradox. Earlier I warned that the reader is an impatient bird, perched on the thin edge of distraction or sleep. Now I'm saying you must write for yourself \& not be gnawed by worry over whether the reader is tagging along.

I'm talking about 2 different issues. One is craft, the other is attitude. The 1st is a question of mastering a precise skill. The 2nd is a question of how you use that skill to express your personality.

In terms of craft, there's no excuse for losing readers through sloppy workmanship. If they doze off in the middle of your article because you have been careless about a technical detail, the fault is yours. But on the larger issue of whether the reader likes you, or likes what you are saying or how you are saying it, or agrees with it, or feels an affinity for your sense of humor or your vision of life, don't give him a moment's worry. You are who you are, he is who he is, \& either you'll get along or you won't.

Perhaps this still seems like a paradox. How can you think carefully about not losing the reader \& still be carefree about his opinion? I assure you that they are separate processes.

1st, work hard to master the tools. Simplify, prune, \& strive for order. Think of this as a mechanical act, \& soon your sentences will become cleaner. The act will never become as mechanical as, say, shaving or shampooing; you will always have to think about the various ways in which the tools can be used. But at least your sentences will be grounded in solid principles, \& your chances of losing the reader will be smaller.

Think of the other as a creative at: the expressing of who you are. Relax \& say what you want to say. \& since style is who you are, you only need to be true to yourself to find it gradually emerging from under the accumulated clutter \& debris, growing more distinctive every day. Perhaps the style won't solidify for years as \textit{your} style, \textit{your} voice. Just as it takes time to find yourself as a person, it takes time to find yourself as a stylist, \& even then your style will change as you grow older.

But whatever your age, be yourself when you write. Many old men still write with the zest they had in their 20s or 30s; obviously their ideas are still young. Other old writers ramble \& repeat themselves; their style is the tip-off that they have turned into garrulous bores. Many college students write as if they were desiccated alumni 30 years out. Never say anything in writing that you wouldn't comfortably say in conversation. If you're not a person who says ``indeed'' or ``moreover,'' or who calls someone an individual (``he's a fine individual''), \textit{please} don't write it.

Let's look at a few writers to see the pleasure with which they put on paper their passions \& their crotchets, not caring whether the reader shares them or not. The 1st excerpt is from ``The Hen (An Appreciation),'' written by E. B. White in 1944, at the height of World War II:
\begin{quotation}
	Chickens do not always enjoy an honorable position among city-bred people, although the egg, I notice, goes on \& on. Right now the hen is in favor. The var has deified her \& she is the darling of the home front, feted at conference tables, praised in every smoking car, her girlish ways \& curious habits the topic of many an excited husbandryman to whom yesterday she was a stranger without honor or allure.
	
	My own attachment to the hen dates from 1907, \& I have been faithful to her in good times \& bad. Ours has not always been an easy relationship to maintain. At 1st, as a boy in a carefully zoned suburb, I had neighbors \& police to reckon with; my chickens had to be as closely guarded as an underground newspaper. Later, as a man in the country, I had my old friends in town to reckon with, most of whom regarded the hen as a comic prop straight out of vaudeville $\ldots$ Their scorn only increased my devotion to the hen. I remained loyal, as a man would to a bride whom his family received with open ridicule. Now it is my turn to wear the smile, as I listen to the enthusiastic cackling of urbanites, who have suddenly taken up the hen socially \& who fill the air with their newfound ecstasy \& knowledge \& the relative charms of the New Hampshire Red \& the Laced Wyandotte. You would think, from their nervous cries of wonder \& praise, that the hen was hatched yesterday in the suburbs of New York, instead of in the remote past in the jungles of India.
	
	To a man who keeps hens, all poultry lore is exciting \& endlessly fascinating. Every spring I settle down with my farm journal \& read, with the same glazed expression on my face, the age-old story of how to prepare a brooder house $\ldots$
\end{quotation}
There's a man writing about a subject I have absolutely no interest in. Yet I enjoy this piece thoroughly. I like the simple beauty of its style. I like the rhythms, the unexpected but refreshing words (``deified,'' ``allure,'' ``cackling''), the specific details like the Laced Wyandotte \& the brooder house. But mainly what I like is that this is a man telling me unabashedly about a love affair with poultry that goes back to 1907. It's written with humanity \& warmth, \& after 3 paragraphs I know quite a lot about what sort of man this hen-lover is.

Or take a writer who is almost White's opposite in terms of style, who relishes the opulent word for its opulence \& doesn't deify the simple sentence. Yet they are brothers in holding firm opinions \& saying what they think. This is H. L. Mencken reporting on the notorious ``Monkey Trial'' -- the trial of John Scopes, a young teacher who taught the theory of evolution in his Tennessee classroom -- in the summer of 1925:
\begin{quotation}
	It was hot weather when they tried the infidel Scopes at Dayton, Tenn., but I went down there very willingly, for I was eager to see something of evangelical Christianity as a going concern. In the big cities of the Republic, despite the endless efforts of consecrated men, it is laid up with a wasting disease. They very Sunday-school superintendents, taking jazz from the stealthy radio, shake their fire-proof legs; their pupils, moving into adolescence, no longer respond to the proliferating hormones by enlisting for missionary service in Africa, but resort to necking instead. Even in Dayton, I found, though the mob was up to do execution on Scopes, there was a strong smell of antinomianism. The 9 churches of the village were all half empty on Sunday, \& weeds choked their yards. Only 2 or 3 of the resident pastors managed to sustain themselves by their ghostly science; the rest had to take orders for mail-order pantaloons or work in the adjacent strawberry fields; one, I heard, was a barber $\ldots$ Exactly 12 minutes after I reached the village I was taken in tow by a Christian man \& introduced to the favorite tipple of the Cumberland Range; half corn liquor \& half Coca-Cola. It seemed a dreadful dose to me, but I found that the Dayton illuminati got it down with gusto, rubbing their tummies \& rolling their eyes. They were all hot for Genesis, but their faces were too florid to belong to teetotalers, \& when a pretty girl came tripping down the main street, they reached for the places where their neckties should have been with all the amorous enterprise of movie stars $\ldots$
\end{quotation}
This is pure Mencken in its surging momentum \& its irreverence. At almost any page where you open his books he is saying something sure to outrage the professed pieties of his countrymen. The sanctity in which Americans bathed their heroes, their churches \& their edifying laws -- especially Prohibition -- was a well of hypocrisy for him that never dried up. Some of his heaviest ammunition he hurled at politicians \& Presidents -- his portrait of ``The Archangel Woodrow'' still scorches the pages -- \& as for Christian believers \& clerical folk, they turn up unfailingly as mountebanks \& boobs.

It may seem a miracle that Mencken could get away with such heresies in the 1920s, when hero worship was an American religion \& the self-righteous wrath of the Bible Belt oozed from coast to coast. Not only did he get away with it; he was the most revered \& influential journalist of his generation. The impact he made on subsequent writers of nonfiction is beyond measuring, \& even now his topical pieces seem as fresh as if they were written yesterday.

The secret of his popularity -- aside from his pyrotechnical use of the American language -- was that he was writing for himself \& didn't give a damn what the reader might think. It wasn't necessary to share his prejudices to enjoy seeing them expressed with such mirthful abandon. Mencken was never timid or evasive; he didn't kowtow to the reader or curry anyone's favor. It takes courage to be such a writer, but it is out of such courage that revered \& influential journalists are born.

Moving forward to our own time, here's an excerpt from \textit{How to Survive in Your Native Land}, a book by James Herndon describing his experiences as a teacher in a California junior high school. Of all the earnest books on education that have sprouted in America, Herndon's is -- for me -- the one that best captures how it really is in the classroom. His style is not quite like anybody else;s, but his voice is true. Here's how the book starts:
\begin{quotation}
	I might as well begin with Piston. Piston was, as a matter of description, a red-headed medium-sized chubby 8th-grader; his definitive characteristic was, however, stubbornness. Without going into a lot of detail, it became clear right away that what Piston didn't want to do, Piston didn't do; what Piston wanted to do, Piston did.
	
	It really wasn't much of a problem. Piston wanted mainly to paint, draw monsters, scratch designs on mimeograph blanks \& print them up, write an occasional horror story -- some kids referred to him as The Ghoul -- \& when he didn't want to do any of those, he wanted to roam the halls \& on occasion (we heard) investigate the girls' bathrooms.
	
	We had minor confrontations. Once I wanted everyone to sit down \& listen to what I had to say -- something about the way they had been acting in the halls. I was letting them come \& go freely \& it was up to them (I planned to point out) not to raise hell so that I had to hear about it from other teachers. Sitting down was the issue -- I was determined everyone was going to do it 1st, then I'd talk. Piston remained standing. I reordered. He paid no intention. I pointed out that I was talking to him. He indicated he heard me. I inquired then why in hell didn't he sit down. He said he didn't want to. I said I did want him to. He said that didn't matter to him. I said do it anyway. He said why? I said because I said so. He said he wouldn't. I said Look I want you to sit down \& listen to what I'm going to say. He said he \textit{was} listening. I'll listen but I won't sit down.
	
	Well, that's the way it goes sometimes in schools. You as teacher become obsessed with an issue -- I was the injured party, conferring, as usual, unheard-of freedoms, \& here they were as usual taking advantage. It ain't pleasant coming in the teachers' room for coffee \& having to hear somebody say that so-\&-so \& so-\&-so from \textit{your} class were out in the halls \textit{without a pass} \& \textit{making faces} \& \textit{giving the finger} to kids in \textit{my} class during the most \textit{important} part of \textit{my} lesson about \textit{Egypt} -- \& you ought to be allowed your tendentious speech, \& most everyone will allow it, sit down for it, but occasionally someone wises you up by refusing to submit where it isn't necessary $\ldots$ How did any of us get into this? we ought to be asking ourselves.
\end{quotation}
Any writer who uses ``ain't'' \& ``tendentious'' in the same sentences, who quotes without using quotation marks, knows what he's doing. This seemingly artless style, so full of art, is ideal for Herndon's purpose. It avoids the pretentiousness that infects so much writing by people doing worthy work, \& it allows for a rich vein of humor \& common sense. Herndon sounds like a good teacher \& a man whose company I would enjoy. But ultimately he is writing for himself: an audience of one.

``Who am I writing for?'' The question that begins this chapter has irked some readers. They want me to say ``Whom am I writing for?'' But I can't bring myself to say it. It's just not me.'' -- \cite[pp. 30--35]{Zinsser2016}

%------------------------------------------------------------------------------%

\section{Words}
``There is a kind of writing that might be called journalese, \& it's the death of freshness in anybody's style. It's the common currency of newspapers \& of magazines like \textit{People} -- a mixture of cheap words, made-up words \& clich\'es that have become so pervasive that a writer can hardly help using them. You must fight these phrases or you'll sound like every hack. You'll never make you mark as a writer unless you develop a respect for words \& a curiosity about their shades of meaning that is almost obsessive. The English language is rich in strong \& supple words. Take the time to root around \& find the ones you want.

What is ``journalese''? It's a quilt of instant words patched together out of other parts of speech. Adjectives are used as nouns (``greats,'' ``notables''). Nouns are used as verbs (``to host''), or they are chopped off to form verbs (``enthuse,'' ``emote''), or they are padded to form verbs (``beef up,'' ``put teeth into''). This is a world where eminent people are ``famed'' \& their associates are ``staffers,'' where the future is always ``up-coming'' \& someone is forever ``firing off'' a note. Nobody in America has sent a note or a memo or a telegram in years. Famed diplomat Condoleezza Rice, who hosts foreign notables to beef up the morale of top State Department staffers, sits down \& fires off a lot of notes. Notes that are fired off are always fired in anger \& from a sitting position. What the weapon is I've never found out.

Here's an article from a famed newsmagazine that is hard to match for fatigue:
\begin{quotation}
	Last Feb, Plainclothes Patrolman Frank Serpico knocked at the door of a suspected Brooklyn heroin pusher. When the door opened a crack, Serpico shouldered his way in only to be met by a .22-cal. pistol slug crashing into his face. Somehow he survived, although there are still buzzing fragments in his head, causing dizziness \& permanent deafness in his left ear. Almost as painful is the suspicion that he may well have been set up for the shooting by other policemen. For Serpico, 35, has been waging a lonely, 4-year war against the routine \& endemic corruption that he \& others claim is rife in the New York City police department. His efforts are now sending shock waves through the ranks of New York's finest $\ldots$ Though the impact of the commission's upcoming report has yet to be felt, Serpico has little hope that $\ldots$
\end{quotation}
The upcoming report has yet to be felt because it's still upcoming, \& as for the permanent deafness, it's a little early to tell. \& what makes those buzzing fragments buzz? By now only Serpico's head should be buzzing. But apart from these lazinesses of logic, what makes the story so tired is the failure of the writer to reach for anything but the nearest clich\'e. ``Shouldered his way,'' ``only to be met,'' ``crashing into his face,'' ``waging a lonely war,'' ``corruption that is rife,'' ``sending shock waves,'' ``New York's finest'' -- these dreary phrases constitute writing at its most banal. We know just what to expect. No surprise awaits us in the form of an unusual word, an oblique look. We are in the hands of a hack, \& we know it right away. We stop reading.

Don't let yourself get in this position. The only way to avoid it is to care deeply about words. If you find yourself writing that someone recently enjoyed a spell of illness, or that a business has been enjoying a slump, ask yourself how much they enjoyed it. Notice the decisions that other writers make in their choice of words \& be finicky about the ones you select from the vast supply. The race in writing is not to the swift but to the original.

Make a habit of reading what is being written today \& what was written by earlier masters. Writing is learned by imitation. If anyone asked me how I learned to write, I'd say I learned by reading the men \& women who were doing the kind of writing \textit{I} wanted to do \& trying to figure out how they did it. But cultivate the best models. Don't assume that because an article is in a newspaper or a magazine it must be good. Sloppy editing is common in newspapers, often for lack of time, \& writers who use clich\'es often work for editors who have seen so many clich\'es that they no longer even recognize them.

Also get in the habit of using dictionaries. My favorite for handy use is \textit{Webster's New World Dictionary}, 2nd College Edition, although, like all word freaks, I own bigger dictionaries that will reward me when I'm on some more specialized search. If you have any doubt of what a word means, look it up. Learn its etymology \& notice what curious branches its original root has put forth. See if it has any meanings you didn't know it had. Master the small gradations between words that seem to be synonyms. What's the difference between ``cajole,'' ``wheedle,'' ``blandish'', \& ``coax''? Get yourself a dictionary of synonyms.

\& don't scorn that bulging grab bag \textit{Roget's Thesaurus}. It's easy to regard the book as hilarious. Look up ``villain,'' e.g., \& you'll be awash in such rascality as only a lexicographer could conjure back from centuries of iniquity, obliquity, depravity, knavery, profligacy, frailty, flagrancy, infamy, immorality, corruption, wickedness, wrongdoing, backsliding, \& sin. You'll find ruffians \& riffraff, miscreants \& malefactors, reprobates \& rapscallions, hooligans \& hoodlums, scamps \& scapegraces, scoundrels \& scalawags, jezebels \& jades. You'll find adjectives to fit them all (foul \& fiendish, devilish \& diabolical), \& adverbs \& verbs to describe how the wrongdoers do their wrong, \& cross-references leading to still other thickets of venality \& vice. Still, there's no better friend to have around to nudge the memory than \textit{Roget}. It saves you the time of rummaging in your brain -- that network of overloaded grooves -- to find the word that's right on the tip of your tongue, where it doesn't do you any good. The \textit{Thesaurus} is to the writer what a rhyming dictionary is to the songwriter -- a reminder of all the choices -- \& you should use it with gratitude. If, having found the scalawag \& the scapegrace, you want to know how they differ, \textit{then} go to the dictionary.

Also bear in mind, when you're choosing words \& stringing them together, how they sound. This may seem absurd: readers read with their eyes. But in fact they hear what they are reading far more than you realize. Therefore such matters as rhythm \& alliteration are vital to every sentence. A typical example -- maybe not the best, but undeniably the nearest -- is the preceding paragraph. Obviously I enjoyed making a certain arrangement of my ruffians \& riffraff, my hooligans \& hoodlums, \& my readers enjoyed it too -- far more than if I had provided a mere list. They enjoyed not only the arrangement but the effort to entertain them. They weren't enjoying it, however, with their eyes. They were hearing the words in their inner ear.

E. B. White makes the case cogently in \textit{The Elements of Style}, a book every writer should read once a year, when he suggests trying to rearrange any phrase that has survived for a century or 2, such as Thomas Paine's ``These are the time that try men's souls'':
\begin{quote}
	Times like these try men's souls.\\How trying it is to live in these times!\\These are trying times for men's souls.\\Soulwise, these are trying times.
\end{quote}
Paine's phrase is like poetry \& the other 4 are like oatmeal -- which is the divine mystery of the creative process. Good writers of prose must be part poet, always listening to what they write. E. B. White is 1 of my favorite stylists because I'm conscious of being with a man who cares about the cadences \& sonorities of the language. I relish (in my ear) the pattern his words make as they fall into a sentence. I try to surmise how in rewriting the sentence he reassembled it to end with a phrase that will momentarily linger, or how he chose 1 word over another because he was after a certain emotional weight. It's the difference between, say, ``serene'' \& ``tranquil'' -- one so soft, the other strangely disturbing because of the unusual \textit{n} \& \textit{q}.

Such considerations of sound \& rhythm should go into everything you write. If all your sentences move at the same plodding gait, which even you recognize as deadly but don't know how to cure, read them aloud. (I write entirely by ear \& read everything aloud before letting it go out into the world.) You'll begin to hear where the trouble lies. See if you can gain variety by reversing the order of a sentence, or by substituting a word that has freshness or oddity, or by altering the length of your sentences so they don't all sound as if they came out of the same machine. An occasional short sentence can carry a tremendous punch. It stays in the reader's ear.

Remember that words are the only tools you've got. Learn to use them with originality \& care. \& also remember: somebody out there is listening.'' -- \cite[pp. 37--40]{Zinsser2016}

%------------------------------------------------------------------------------%

\section{Usage}
``All this talk about good words \& bad words brings us to a gray but important area called ``usage.'' What is good usage? What is good English? What newly minted words is it O.K. to use, \& who is to be the judge? Is it O.K. to use ``O.K.''?

Earlier I mentioned an incident of college students hassling the administration, \& in the last chapter I described myself as a word freak. Here are 2 fairly recent arrivals. ``Hassle'' is both a verb \& a noun, meaning to give somebody a hard time, or the act of being given a hard time, \& anyone who has ever been hassled for not properly filling out Form 35-BX will agree that the word sounds exactly right. ``Freak'' means an enthusiast, \& there's no missing the aura of obsession that goes with calling someone a jazz freak, or a chess freak, or a sun freak, though it would probably be pushing my luck to describe a man who compulsively visits circus sideshows as a freak freak.

Anyway, I accept these 2 usages gladly. I don't consider them slang, or put quotation marks around them to show that I'm mucking about in the argot of the youth culture \& really know better. They're good words \& we need them. But I won't accept ``notables'' \& ``greats'' \& ``upcoming'' \& many other newcomers. They are cheap words \& we \textit{don't} need them.

Why is 1 word good \& another word cheap? I can't give you an answer, because usage has no fixed boundaries. Language is a fabric that changes from 1 week to another, adding new strands \& dropping old ones, \& even word freaks fight over what is allowable, often reaching their decision on a wholly subjective basis such as taste (``notables'' is sleazy). Which still leaves the question of who our tastemakers are.

The question was confronted in the 1960s by the editors of a brand-new dictionary, \textit{The American Heritage Dictionary}. They assembled a ``Usage Panel'' to help them appraise the new words \& dubious constructions that had come knocking at the door. Which ones should be ushered in, which thrown out on their ear? The panel consisted of 104 men \& women -- mostly writers, poets, editors, \& teachers -- who were known for caring about the language \& trying to use it well. I was a member of the panel, \& over the next few years I kept getting questionnaires. Would I accept ``finalize'' \& ``escalate''? How did I feel about ``It's me''? Would I allow ``like'' to be used as a conjunction -- like so many people do? How about ``mighty,'' as in ``mighty fine''?

We were told that in the dictionary our opinions would be tabulated in a separate ``Usage Note,'' so that readers could see how we voted. The questionnaire also left room for any comments we might feel impelled to make -- an opportunity the panelists seized avidly, as we found when the dictionary was published \& our comments were released to the press. Passions ran high. ``Good God, no! Never!'' cried Barbara W. Tuchman, asked about the verb ``to author.'' Scholarship hath no fury like that of a language purist faced with sludge, \& I shared Tuchman's vow that ``author'' should never be authorized, just as I agreed with Lewis Mumford that the adverb ``good'' should be ``left as the exclusive property of Ernest Hemingway.''

But guardians of usage are doing only half their job if they merely keep the language from becoming sloppy. Any dolt can rule that the suffix ``wise,'' as in ``healthwise,'' is doltwise, or that being ``rather unique'' is no more possible than being rather pregnant. The other half of the job is to help the language grow by welcoming any immigrant that will bring strength or color. Therefore I was glad that 97\% of us voted to admit ``dropout,'' which is clean \& vivid, but that only 47\% would accept ``senior citizen,'' which is typical of the pudgy new intruders from the land of sociology, where an illegal alien is now an undocumented resident. I'm glad we accepted ``escalate,'' the kind of verbal contraption I generally dislike but which the Vietnam war endowed with a precise meaning, complete with overtones of blunder.

I'm glad we took into full membership all sorts of robust words that previous dictionaries derided as ``colloquial'': adjectives like ``rambunctious,'' verbs like ``trigger'' \& ``rile,'' nouns like ``shambles'' \& ``tycoon'', \& ``trek,'' the latter approved by 78\% to mean any difficult trip, as in ``the commuter's daily trek to Manhattan.'' Originally it was a Cape Dutch word applied to the Boers' arduous journey by ox wagon. But our panel evidently felt that the Manhattan commuter's daily trek is no less arduous.

Still, 22\% were unwilling to let ``trek'' slip into general usage. That was the virtue of revealing how our panel voted -- it put our opinions on display, \& writers in doubt can conduct themselves accordingly. Thus our 95\% vote against ``myself,'' as in ``He invited Mary \& myself to dinner,'' a word condemned as ``prissy,'' ``horrible'' \& ``genteelism,'' ought to warn off anyone who doesn't want to be prissy, horrible or genteel. As Red Smith put it, ```Myself' is the refuge of idiots taught early that `me' is a dirty word.''

On the other hand, only 66\% of our panel rejected the verb ``to contact,'' once regarded as tacky, \& only half opposed the split infinitive \& the verbs ``to fault'' \& ``to bus.'' So only 50\% of your readers will fault you if you decide to voluntarily call your school board \& to bus your children to another town. If you contact your school board you risk your reputation by another 16\%. Our apparent rule of thumb was stated by Theodore M. Bernstein, author of the excellent \textit{The Careful Writer}: ``We should apply the test of convenience. Does the word fill a real need? If it does, let's give it a franchise.''

All of this confirms what lexicographers have always known: that the laws of usage are relative, bending with the taste of the lawmaker. 1 of our panelists, Katherine Anne Porter, called ``O.K.'' a ``detestable vulgarity'' \& claimed she had never spoken the word in her life, whereas I freely admit that I have spoken the word ``O.K.'' ``Most,'' as in ``most everyone,'' was scorned as ``cute farmer talk'' by Isaac Asimov \& embraced as a ``good English idiom'' by Virgil Thomson. ``Regime,'' meaning any administration, as in ``the Truman regime,'' drew the approval of most everyone on the panel, as did ``dynasty.'' But they drew the wrath of Jacques Barzun, who said, ``These are technical terms, you blasted non-historians!'' Probably I gave my O.K. to ``regime.'' Now, chided by Barzun for imprecision, I think it looks like journalese. 1 of the words \textit{I} railed against was ``personality,'' as in a ``TV personality.'' But now I wonder if it isn't the only word for that vast swarm of people who are famous for being famous -- \& possibly nothing else. What did the Gabor sisters actually \textit{do}?

In the end it comes down to what is ``correct'' usage. We have no king to establish the King's English; we only have the President's English, which we don't want. \textit{Webster}, long a defender of the faith, muddied the waters in 1961 with its permissive 3rd Edition, which argued that almost anything goes as long as somebody uses it, noting that ``ain't'' is ``used orally in most parts of the U.S. by many cultivated speakers.''

Just where \textit{Webster} cultivated those speakers I ain't sure. Nevertheless it's true that the spoken language is looser than the written language, \& \textit{The American Heritage Dictionary} properly put its question to us in both forms. Often we allowed an oral idiom that we forbade in print as too informal, fully realizing, however, that ``the pen must at length comply with the tongue,'' as Samuel Johnson said, \& that today's spoken garbage may be tomorrow's written gold. The growing acceptance of the split infinitive, or of the preposition at the end of a sentence, proves that formal syntax can't hold the fort forever against a speaker's more comfortable way of getting the same thing said -- \& it shouldn't. I think a sentence is a fine thing to put a preposition at the end of.

Our panel recognized that correctness can even vary within a word. We voted heavily against ``cohort'' as a synonym for ``colleague,' except when the tone was jocular. Thus a professor would not be among his cohorts at a faculty meeting, but they would abound at his college reunion, wearing funny hats. We rejected ``too'' as a synonym for ``very,'' as in ``His health is not too good.'' Whose health is? But we approved it in sardonic or humorous use, as in ``He was not too happy when she ignored him.''

These may seem like picayune distinctions. They're not. They are signals to the reader that you are sensitive to the shadings of usage. ``Too'' when substituted for ``very'' is clutter: ``He didn't feel too much like going shopping.'' But the wry example in the previous paragraph is worthy of Ring Lardner. It adds a tinge of sarcasm that otherwise wouldn't be there.

Luckily, a pattern emerged from the deliberations of our panel, \& it offers a guideline that is still useful. We turned out to be liberal in accepting new words \& phrases, but conservative in grammar.

It would be foolish to reject a word as perfect as ``dropout,'' or to pretend that countless words \& phrases are not entering the gates of correct usage every day, borne on the winds of science \& technology, business \& sports \& social change: ``outsource,'' ``blog,'' ``laptop,'' ``mousepad,'' ``geek,'' ``boomer,'' ``Google,'' ``iPod,'' ``hedge fund,'' ``24{\tt/}7,'' ``multi-tasking,'' ``slam dunk'' \& hundreds of others. Nor should we forget all the short words invented by the counterculture in the 1960s as a way of lashing back at the self-important verbiage o the Establishment: ``trip,'' ``rap,'' ``crash,'' ``trash,'' ``funky,'' ``split,'' ``rip-off,'' ``vibes,'' ``downer,'' ``bummer.'' If brevity is a prize, these were winners. The only trouble with accepting words that entered the language overnight is that they often leave just as abruptly. The ``happenings'' of the late 1960s no longer happen, ``out of sight'' is out of sight, \& even ``awesome'' has begun to chill out. The writer who cares about usage must always know the quick from the dead.

As for the area where ur Usage Panel was conservative, we upheld most of the classic distinctions in grammar -- ``can'' \& ``may,'' ``fewer'' \& ``less,'' ``eldest'' \& ``oldest,'' etc. -- \& decried the classic errors, insisting that ``flout'' still doesn't mean ``flaunt,'' no matter how many writers flaunt their ignorance by flouting the rule, \& that ``fortuitous'' still means ``accidental,'' ``disinterested'' still means ``impartial,'' \& ``infer'' doesn't mean ``imply.'' Here we were motivated by our love of the language's beautiful precision. Incorrect usage will lose you the readers you would most like to win. Know the difference between a ``reference'' \& an ``allusion,'' between ``connive'' \& ``conspire,'' between ``compare with'' \& ``compare to.'' If you must use ``comprise,'' use it right. It means ``include''; dinner comprises meat, potatoes, salad, \& dessert.

``I choose always the grammatical form unless it sounds affected,'' Marianne Moore explained, \& that's finally where our panel took its stand. We were not pedants, so hung up on correctness that we didn't want the language to keep refreshing itself with phrases like ``hung up.'' But that didn't mean we had to accept every atrocity that comes lumbering in.

Meanwhile the battle continues. Today I still receive ballots from \textit{The American Heritage Dictionary} soliciting my opinion on new locutions: verbs like ``definitize'' (``Congress definitized a proposal''), nouns like ``affordables,'' colloquialisms like ``the bottom line'' \& strays like ``into'' (``He's into backgammon \& she's into jogging'').

It no longer takes a panel of experts to notice that jargon is flooding our daily life \& language. President Carter signed an executive order directing that federal regulations be written ``simply \& clearly.'' President Clinton's attorney general, Janet Reno, urged the nation's lawyers to replace ``a lot of legalese'' with ``small, old words that all people understand'' -- ``words like ``right'' \& ``wrong'' \& ``justice.'' Corporations have hired consultants to make their prose less opaque, \& even the insurance industry is trying to rewrite its policies to tell us in less disastrous English what redress will be ours when disaster strikes. Whether these efforts will do much good I wouldn't want to bet. Still, there's comfort in the sight of so many watchdogs standing Canute-like on the beach, trying to hold back the tide. That's where all careful writers ought to be -- looking at every new piece of flotsam that washes up \& asking ``Do we need it?''

I remember the 1st time somebody asked me, ``How does that impact you?'' I always thought ``impact'' was a noun, except in dentistry. Then I began to meet ``de-impact,'' usually in connection with programs to de-impact the effects of some adversity. Nouns now turn overnight into verbs. We target goals \& we access facts. Train conductors announce that the train won't platform. A sign on an airport door tells me that the door is alarmed. Companies are downsizing. It's part of an ongoing effort to grow the business. ``Ongoing'' is a jargon word whose main use is to raise morale. We face our daily job with more zest if the boss tells us it's an ongoing project; we give more willingly to institutions if they have targeted our funds for ongoing needs. Otherwise we might fall prey to disincentivization.

I could go on; I have enough examples to fill a book, but it's not a book I would want anyone to read. We're still left with the question: What is good usage? 1 helpful approach is to try to separate usage from jargon.

I would say, e.g., that ``prioritize'' is jargon -- a pompous new verb that sounds more important than ``rank'' -- \& that ``bottom line'' is usage, a metaphor borrowed from the world of bookkeeping that conveys an image we can picture. As every businessman knows, the bottom line is the one that matters. If someone says, ``The bottom line is that we just can't work together,'' we know what he means. I don't much like the phrase, but the bottom line is that it's here to stay.

new usages also arrive with new political events. Just as Vietnam gave us ``escalate,'' Watergate gave us a whole lexicon of words connoting obstruction \& deceit, including ``deep-6,'' ``launder,'' ``enemies list'' \& other ``gate''-suffix scandals (``Irangate''). It's a fitting irony that under Richard Nixon ``launder'' become a dirty word. Today when we hear that someone laundered his funds to hide the origin of the money \& the route it took, the word has a precise meaning. It's short, it's vivid, \& we need it. I accept ``launder'' \& ``stonewall''; I don't accept ``prioritize'' \& ``disincentive.''

I would suggest a similar guideline for separating good English from technical English. It's the difference between, say, ``printout'' \& ``input.'' A printout is a specific object that a computer emits. Before the advent of computers it wasn't needed; now it is. But it ha stayed where it belongs. Not so with ``input,'' which was coined to describe the information that's fed to a computer. Our input is sought on every subject, from diets to philosophical discourse (``I'd like your input on whether God really exists'').

I don't want to give somebody my input \& get his feedback, though I'd be glad to offer my ideas \& hear what he thinks of them. Good usage, to me, consists of using good words if they already exist -- as they almost always do -- to express myself clearly \& simply to someone else. You might say it's how I verbalize the interpersonal.'' -- \cite[pp. 42--48]{Zinsser2016}

%------------------------------------------------------------------------------%

\begin{center}\huge
	Part II: Methods
\end{center}

%------------------------------------------------------------------------------%

\section{Unity}
``You learn to write by writing. It's a truism, but what makes it a truism is that it's true. The only way to learn to write is to force yourself to produce a certain number of words on a regular basis.

If you went to work for a newspaper that required you to write 2 or 3 articles every day, you would be a better writer after 6 months. You wouldn't necessarily be writing well; your style might still be full of clutter \& clich\'es. But you would be exercising your powers of putting the English language on paper, gaining confidence \& identifying the most common problems.

All writing is ultimately a question of solving a problem. It may be a problem of where to obtain the facts or how to organize the material. It may be a problem of approach or attitude, tone or style. Whatever it is, it has to be confronted \& solved. Sometimes you will despair of finding the right solution -- or any solution. You'll think, ``If I live to be 90 I'll never get out of this mess.'' I've often thought it myself. But when I finally do solve the problem it's because I'm like a surgeon removing his 500th appendix; I've been there before.

Unity is the anchor of good writing. So, 1st, get your unities straight. Unity not only keeps the reader from straggling off in all directions; it satisfies your readers' subconscious need for order \& reassures them that all is well at the helm. Therefore choose from among the many variables \& stick to your choice.

1 choice is unity of pronoun. Are you going to write in the 1st person, as a participant, or in the 3rd person, as an observer? Or even in the 2nd person, that darling of sportswriters hung up on Hemingway? (``You knew this had to be the most spine-tingling clash of giants you'd ever seen from a pressbox seat, \& you weren't just some green kid who was still wet behind the ears.'')

Unity of tense is another choice. Most people write mainly in the past tense (``I went up to Boston the other day''), but some people write agreeably in the present (``I'm sitting in the dining car of the Yankee Limited \& we're pulling into Boston''). What is not agreeable is to switch back \& forth. I'm not saying you can't use $> 1$ tense; the whole purpose of tenses is to enable a writer to deal with time in its various gradations, from the past to the hypothetical future (``When I telephoned my mother from the Boston station, I realized that if I had written to tell her I would be coming she would have waited for me''). But you must choose the tense in which you are \textit{principally} going to address the reader, no matter how many glances you may take backward \& forward along the way.

Another choice is unity of mood. You might want to talk to the reader in the casual voice that \textit{The New Yorker} has strenuously refined. Or you might want to approach the reader with a certain formality to describe a serious event or to present a set of important facts. Both tones are acceptable. In fact, \textit{any} tone is acceptable. But don't mix 2 or 3.

Such fatal mixtures are common in writers who haven't learned control. Travel writing is a conspicuous example. ``My wife, Ann, \& I had always wanted to visit Hong Kong,'' the writer begins, his blood astir with reminiscence, ``\& 1 day last spring we found ourselves looking at an airline poster '\& I said, `Let's go!' The kids were grown up,'' he continues, \& he proceeds to describe in genial detail how he \& his wife stopped off in Hawaii \& had such a comical time changing their money at the Hong Kong airport \& finding their hotel. Fine. He is a real person taking us along on a real trip, \& we can identify with him \& Ann.

Suddenly he turns into a travel brochure. ``Hong Kong affords many fascinating experiences to the curious sightseer,'' he writes. ``One can ride the picturesque ferry from Kowloon \& gawk at the myriad sampans as they scuttle across the teeming harbor, or take a day's trip to browse in the alleys of fabled Macao with its colorful history as a den of smuggling \& intrigue. You will want to take the quaint funicular that climbs $\ldots$'' Then we get back to him \& Ann \& their efforts to eat at Chinese restaurants, \& again all is well. Everyone is interested in food, \& we are being told about a personal adventure.

Then suddenly the writer is a guidebook: ``To enter Hong Kong it is necessary to have a valid passport, but no visa is required. You should definitely be immunized against hepatitis \& you would also be well advised to consult your physician with regard to a possible inoculation for typhoid. The climate in Hong Kong is seasonable except in Jul \& Aug when $\ldots$'' Our writer is gone, \& so is Ann, \& so -- very soon -- are we.

It's not that the scuttling sampans \& the hepatitis shots shouldn't be included. What annoys us is that the writer never decided what kind of article he wanted to write or how he wanted to approach us. He comes at us in many guises, depending on what kind of material he is trying to purvey. Instead of controlling his material, his material is controlling him. That wouldn't happen if he took time to establish certain unities.

Therefore ask yourself some basic questions before you start. E.g.: ``In what capacity am I going to address the reader?'' (Reporter? Provider of information? Average man or woman?) ``What pronoun \& tense am I going to use?'' ``What style?'' (Impersonal reportorial? Personal but formal? Personal \& casual?) ``What attitude am I going to take toward the material?'' (Involved? Detached? Judgmental? Ironic? Amused?) ``How much do I want to cover?'' ``What 1 point do I want to make?''

The last 2 questions are especially important. Most nonfiction writers have a definitiveness complex. They feel that they are under some obligation -- to the subject, to their honor, to the gods of writing -- to make their article the last word. It's a commendable impulse, but there is no last word. What you think is definitive today will turn undefinitive by tonight, \& writers who doggedly pursue every last fact will find themselves pursuing the rainbow \& never settling down to write. Nobody can write a book or an article ``about'' something. Tolstoy couldn't write a book about war \& peace, or Melville a book about whaling. They made certain reductive decisions about time \& place \& about individual characters in that time \& place -- 1 man pursuing 1 whale. Every writing project must be reduced before you start to write.

Therefore think small. Decide what corner of your subject you're going to bite off, \& be content to cover it well \& stop. This is also a matter of energy \& morale. An unwieldy writing task is a drain on your enthusiasm. Enthusiasm is the force that keeps you going \& keeps the reader in your grip. When your zest begins to ebb, the reader is the 1st person to know it.

As for what point you want to make, every successful piece of nonfiction should leave the reader with 1 provocative thought that he or she didn't have before. Not 2 thoughts, or 5 -- just 1. So decided what single point you want to leave in the reader's mind. It will not only give you a better idea of what route you should follow \& what destination you hope to reach; it will affect your decision about tone \& attitude. Some points are best made by earnestness, some by dry understatement, some by humor.

Once you have your unities decided, there's no material you can't work into your frame. If the tourist in Hong Kong had chosen to write solely in the conversational vein about what he \& Ann did, he would have found a natural way to weave into his narrative whatever he wanted to tell us about the Kowloon ferry \& the local weather. His personality \& purpose would have been intact, \& his article would have held together.

Now it often happens that you'll make these prior decisions \& then discover that they weren't the right ones. The material begins to lead you in an unexpected direction, where you are more comfortable writing in a different tone. That's normal -- the act of writing generates some cluster of thoughts or memories that you didn't anticipate. Don't fight such a current if it feels right. Trust your material if it's taking you into terrain you didn't intend to enter but where the vibrations are good. Adjust your style accordingly \& proceed to whatever destination you reach. Don't become the prisoner of a preconceived plan. Writing is no respecter of blueprints.

If this happens, the 2nd part of your article will be badly out of joint with the 1st. But at least you know which part is truest to your instincts. Then it's just a matter of making repairs. Go back to the beginning \& rewrite it so that your mood \& your style are consistent from start to finish.

There's nothing in such a method to be ashamed of. Scissors \& paste -- or their equivalent on a computer -- are honorable writers' tools. Just remember that all the unities must be fitted into the edifice you finally put together, however backwardly they may be assembled, or it will soon come tumbling down.'' -- \cite[pp. 52--55]{Zinsser2016}

%------------------------------------------------------------------------------%

\section{The Lead \& the Ending}

%------------------------------------------------------------------------------%

\section{Bits \& Pieces}

%------------------------------------------------------------------------------%

\begin{center}\huge
	Part III: Forms
\end{center}

%------------------------------------------------------------------------------%

\section{Nonfiction as Literature}

%------------------------------------------------------------------------------%

\section{Writing About People: The Interview}

%------------------------------------------------------------------------------%

\section{Writing About Places: The Travel Article}

%------------------------------------------------------------------------------%

\section{Writing About Yourself: The Memoir}

%------------------------------------------------------------------------------%

\section{Science \& Technology}

%------------------------------------------------------------------------------%

\section{Business Writing: Writing in Your Job}

%------------------------------------------------------------------------------%

\section{Sports}

%------------------------------------------------------------------------------%

\section{Writing About the Arts: Critics \& Columnists}

%------------------------------------------------------------------------------%

\section{Humor}

%------------------------------------------------------------------------------%

\section{The Sound of Your Voice}

%------------------------------------------------------------------------------%

\section{Enjoyment, Fear, \& Confidence}

%------------------------------------------------------------------------------%

\section{The Tyranny of the Final Product}

%------------------------------------------------------------------------------%

\section{A Writer's Decisions}

%------------------------------------------------------------------------------%

\section{Write as Well as You Can}

%------------------------------------------------------------------------------%

\printbibliography[heading=bibintoc]

\end{document}